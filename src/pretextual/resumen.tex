% ══════════════════════════════════════════════════════════════════
% RESUMEN (Parte Pre-Textual)
% ══════════════════════════════════════════════════════════════════

\thispagestyle{plain}

\phantomsection
\sectionplain{Resumen}
\addcontentsline{toc}{section}{Resumen}

La sistematización del proceso de pre-escritura de guiones constituyó una práctica fundamental en el desarrollo de proyectos audiovisuales, pero permaneció escasamente documentada en el contexto paraguayo. El presente estudio descriptivo de enfoque cualitativo tuvo como objetivo describir los métodos de sistematización del proceso de pre-escritura de guiones utilizados por los guionistas del Departamento Central durante el año 2025.

Se empleó una muestra intencional de tres guionistas profesionales con experiencia entre 2 y 10 años, aplicándose entrevistas semiestructuradas como instrumento de recolección de datos. Los hallazgos revelaron que los guionistas desarrollaron sistemas personalizados de organización de elementos narrativos que priorizaron aspectos psicológicos sobre estandarizaciones técnicas, rechazando fichas tradicionales prefabricadas.

Se identificó el paradigma de tres actos como base estructural universal, implementado mediante proceso bifásico de exploración creativa seguida de estructuración rigurosa. La secuencia metodológica Logline → Sinopsis → Escaleta emergió como fundamento documental consistente.

Se evidenció preferencia unánime por métodos híbridos análogo-digitales, destacándose el uso de post-its físicos en pared para visualización estructural, justificado cognitivamente por la necesidad de manipulación táctil que favorece creatividad. Las tres dificultades principales identificadas fueron: falta de visualización integrada de elementos, dispersión de información en herramientas desconectadas, y ausencia de vinculación dinámica entre elementos narrativos y guion.

Los resultados validaron empíricamente la vigencia de paradigmas teóricos de Field, McKee y Koestler en el contexto local, proporcionando evidencia que fundamentó el desarrollo de DREAMINK como sistema integrado de estructuración dramática diseñado para resolver necesidades articuladas por profesionales locales.

\noindent Palabras clave: sistematización, pre-escritura de guiones, guionistas, Departamento Central, métodos de organización narrativa, estructuración dramática, herramientas de guionismo.

\clearpage
