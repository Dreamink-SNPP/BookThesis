% ══════════════════════════════════════════════════════════════════
% ABSTRACT (Parte Pre-Textual)
% ══════════════════════════════════════════════════════════════════

\thispagestyle{plain}

\sectionplain{Abstract}
\addcontentsline{toc}{section}{Abstract}

The systematization of the screenwriting pre-writing process constituted a fundamental practice in the development of audiovisual projects, but remained scarcely documented in the Paraguayan context. This descriptive study with a qualitative approach aimed to describe the systematization methods of the screenwriting pre-writing process used by screenwriters in the Central Department during 2025.

An intentional sample of three professional screenwriters with 2 to 10 years of experience was employed, applying semi-structured interviews as the data collection instrument. The findings revealed that screenwriters developed personalized systems for organizing narrative elements that prioritized psychological aspects over technical standardizations, rejecting traditional prefabricated templates.

The three-act paradigm was identified as a universal structural foundation, implemented through a biphasic process of creative exploration followed by rigorous structuring. The methodological sequence Logline → Synopsis → Step Outline emerged as a consistent documentary foundation. Unanimous preference for analog-digital hybrid methods was evidenced, highlighting the use of physical sticky notes on walls for structural visualization, cognitively justified by the need for tactile manipulation that fosters creativity.

The three main difficulties identified were: lack of integrated visualization of elements, information dispersion across disconnected tools, and absence of dynamic linkage between narrative elements and screenplay.

The results empirically validated the validity of theoretical paradigms from Field, McKee, and Koestler in the local context, providing evidence that supported the development of DREAMINK as an integrated dramatic structuring system designed to address needs articulated by local professionals.

\noindent Keywords: systematization, screenwriting pre-writing, screenwriters, Central Department, narrative organization methods, dramatic structuring, screenwriting tools.

\clearpage
