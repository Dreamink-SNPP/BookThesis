% ══════════════════════════════════════════════════════════════════
% CAPÍTULO I - MARCO INTRODUCTORIO
% ══════════════════════════════════════════════════════════════════

\clearpage
\thispagestyle{empty}
\vspace*{\fill}
\begin{center}
    {\Large\bfseries CAPÍTULO I - MARCO INTRODUCTORIO}
\end{center}
\vspace*{\fill}
\clearpage

\section*{CAPÍTULO I - MARCO INTRODUCTORIO}
\addcontentsline{toc}{section}{CAPÍTULO I - MARCO INTRODUCTORIO}
\setcounter{section}{1}

\subsection{Planteamiento del Problema}

En el mundo entero, los guionistas enfrentan desafíos durante la pre-escritura, especialmente por la naturaleza no lineal de su trabajo creativo que transita sin orden específico entre conceptualización, desarrollo de personajes y estructuración narrativa \parencite{chi2025_screenwriters}. La investigación internacional ha documentado que los principales desafíos incluyen falta de inspiración, coherencia insuficiente y profundidad de contenido inadecuada \parencite{screenwriting_research_network}.

En América Latina, los guionistas se forman mayormente de manera autodidacta: en Chile, 32.8\% aprenden por sí mismos \parencite{chile_guion_cine}, mientras que en Argentina los estudios revelan la necesidad de analizar cómo organizan su trabajo creativo \parencite{argentina_telenovelas_guionistas}. Los centros y programas de escritura en la región son recientes y las prácticas organizativas de los guionistas constituyen un área escasamente documentada \parencite{centros_escritura_latam}.

En la República del Paraguay, el sector audiovisual experimenta una institucionalización reciente con iniciativas como el Premio Nacional de Guion del INAP \parencite{inap_premio_guion} y la Ley N° 6106 de Fomento al Audiovisual que reconoce al guionista como autor \parencite{ley_6106_paraguay}. Existen espacios como la Asociación de Guionistas del Paraguay (KUATIA) y la Residencia del Lago \parencite{residencia_lago_paraguay}, pero los métodos que usan los guionistas locales durante la pre-escritura no han sido estudiados académicamente.

La sistematización del proceso de pre-escritura de guiones que emplean los guionistas del Departamento Central no ha sido estudiada en la literatura académica.

La causa principal es que la investigación académica ha priorizado analizar guiones terminados en lugar de estudiar los procesos creativos que los originan, y la naturaleza personalizada del trabajo de pre-escritura dificulta su observación sistemática.

La consecuencia para los guionistas, especialmente quienes se inician en la profesión, es que deben aprender de manera autodidacta sin referencias sobre métodos efectivos de organización, lo que limita su desarrollo profesional y prolonga innecesariamente su curva de aprendizaje.

El presente estudio describe los métodos de sistematización del proceso de pre-escritura de guiones utilizados por los guionistas del Departamento Central durante 2025, mediante entrevistas semiestructuradas que permitan conocer sus prácticas reales de organización, herramientas empleadas y dificultades enfrentadas.

\newpage

\subsection{Preguntas de Investigación}

Aquí se presentarán las preguntas de investigación, como la pregunta general y las específicas.

\subsubsection{Pregunta General}
¿Cuáles son los métodos de sistematización del proceso de pre-escritura de guiones utilizados por los guionistas del Departamento Central durante el año 2025?

\subsubsection{Preguntas Específicas}
\begin{enumerate}
	\item ¿Cómo organizan los guionistas los elementos narrativos (personajes, locaciones e ideas) durante la fase de pre-escritura?
	\item ¿Qué herramientas y métodos utilizan los guionistas para estructurar dramáticamente sus obras antes del guion literario?
	\item ¿Cuáles son las principales dificultades que enfrentan los guionistas en la sistematización del proceso de pre-escritura?
\end{enumerate}

\newpage

\subsection{Objetivos}

Aquí se demostrarán los objetivos tanto general como específicos.

\subsubsection{Objetivo General}
Describir los métodos de sistematización del proceso de pre-escritura de guiones utilizados por los guionistas del Departamento Central durante el año 2025.

\subsubsection{Objetivos Específicos}
\begin{enumerate}
	\item Identificar las formas de organización de elementos narrativos (personajes, locaciones e ideas) empleadas por los guionistas durante la fase de pre-escritura.
	\item Caracterizar las herramientas y métodos utilizados por los guionistas para estructurar dramáticamente sus obras antes del guion literario.
	\item Describir las principales dificultades que enfrentan los guionistas en la sistematización del proceso de pre-escritura.
\end{enumerate}

\newpage

\subsection{Justificación}

Desde el punto de vista teórico, este estudio aporta conocimiento sobre una variable escasamente investigada en un contexto geográfico específico. La sistematización del proceso de pre-escritura de guiones no ha sido caracterizada en el Departamento Central de Paraguay, constituyendo un vacío de conocimiento sobre cómo los guionistas locales organizan efectivamente su trabajo creativo. Al describir estos métodos, se genera evidencia empírica que contribuye al desarrollo de marcos teóricos sobre procesos creativos en guionismo y prácticas profesionales en contextos latinoamericanos.

Desde el punto de vista práctico, el estudio permite prevenir que los guionistas, especialmente quienes se inician en la profesión, continúen aprendiendo exclusivamente de forma autodidacta sin referencias metodológicas. Al documentar las prácticas reales de sistematización empleadas por guionistas activos, se generan referentes concretos que pueden orientar la formación de nuevos profesionales y reducir la curva de aprendizaje innecesariamente prolongada que caracteriza al sector.

Desde el punto de vista social, los beneficiarios directos son los guionistas del Departamento Central, cuyas prácticas profesionales serán visibilizadas y legitimadas académicamente. Asimismo, se benefician las instituciones educativas que ofrecen formación en guionismo, las organizaciones gremiales como KUATIA, los programas de desarrollo audiovisual como la Residencia del Lago, y los futuros estudiantes de narrativa audiovisual que accederán a conocimiento sistematizado sobre métodos de trabajo profesional.

Desde el punto de vista económico, el estudio puede contribuir a optimizar la inversión pública y privada en políticas de fomento al sector audiovisual. Al identificar las prácticas efectivas de sistematización y las principales dificultades enfrentadas por los guionistas, las instituciones como el INAP podrán diseñar programas de capacitación, talleres y recursos de apoyo más pertinentes a las necesidades reales de los profesionales, maximizando el impacto de los recursos destinados al desarrollo del sector.

Desde el punto de vista metodológico, el estudio aporta una operacionalización validada de la variable «sistematización del proceso de pre-escritura de guiones» con dimensiones, indicadores e instrumento de recolección de datos específicamente diseñados para este fenómeno. Esta contribución metodológica puede ser empleada por otros investigadores interesados en estudiar procesos creativos de guionistas en diferentes contextos geográficos, facilitando la réplica y comparación de hallazgos en futuras investigaciones sobre prácticas profesionales en el sector audiovisual.

\newpage

\subsection{Limitaciones}

Este sistema presentará las siguientes limitaciones, de acuerdo con su alcance técnico y funcional:

\begin{enumerate}[label=\arabic*., leftmargin=1.5cm, itemsep=0.2em]
	\item Funcionará únicamente en entornos web.
	\item Será de uso exclusivamente local.
	\item No permitirá trabajo colaborativo en modalidad multiusuario.
	\item Cada usuario gestiona su propio proyecto.
	\item Ningún usuario podrá gestionar el proyecto de otros.
	\item Permitirá únicamente la exportación al formato \textit{Fountain}.
	\item No admitirá la importación de formatos privativos (como \textit{FadeIn}, \textit{Final Draft}, \textit{WriterDuet}, etc.).
	\item Se limitará a la estructuración previa del guion, sin abarcar el desarrollo del guion literario.
	\item No cubrirá la elaboración del guion técnico.
	\item La interfaz estará optimizada para pantallas de computadoras, no para dispositivos móviles.
	\item No se integrará con modelos de lenguaje de gran escala (LLMs) ni con herramientas similares.
\end{enumerate}

\newpage

\subsection{Alcance o Delimitación}

Se expresarán estos siguientes puntos para demostrar el alcance técnico y funcional del estudio:

\begin{enumerate}[label=\arabic*., leftmargin=1.5cm, itemsep=0.2em]
	\item Permitirá el uso de cuentas de usuarios mediante correo y contraseña.
	\item Posee el ingreso de cuentas y registro de nuevos usuarios.
	\item Posee recuperación de contraseñas para cuentas de usuario.
	\item Se podrán dar de alta, baja, modificación y consulta de personajes.
	\item Se podrán dar de alta, baja, modificación y consulta de locaciones.
	\item Se podrán dar de alta, baja, modificación y consulta de ideas.
	\item Permite la organización de la estructura del guion en actos, secuencias y escenas.
	\item Cada proyecto tendrá su propio tratamiento, estructura, personajes, locaciones e ideas.
	\item Posibilidad de realizar reportes en PDF para tratamientos, personajes, locaciones e ideas.
	\item Reportes en PDF individuales para cara personaje, locación o idea.
\end{enumerate}

\newpage

\subsection{Presupuesto}

Aquí se expondrán los presupuestos humanos, hardware y software para la realización del proyecto.

\begin{table}[ht]
	\centering
	\renewcommand{\arraystretch}{1.3}
	\begin{tabularx}{\textwidth}{|>{\centering\arraybackslash}p{2.5cm}|>{\raggedright\arraybackslash}X|>{\raggedleft\arraybackslash}p{3cm}|}
		\hline
		\multicolumn{3}{|c|}{\textbf{PRESUPUESTO PARCIAL}} \\ \hline
		\textbf{RECURSOS} & \textbf{DESCRIPCIÓN} & \textbf{IMPORTE} \\ \hline

		\multirow{2}{*}{\textbf{HUMANO}}
		& Investigación del trabajo con Internet & 500.000 Gs. \\ \cline{2-3}
		& Viáticos & 100.000 Gs. \\ \hline

		\multirow{3}{*}{\textbf{HARDWARE}}
		& Computadoras personales & 0 Gs. \\ \cline{2-3}
		& Pendrive USB 16 GB & 0 Gs. \\ \cline{2-3}
		& Impresoras / Papelería & 150.000 Gs. \\ \hline

		\multirow{7}{*}{\textbf{SOFTWARE}}
		& Sistemas Operativos GNU/Linux y Windows & 0 Gs. \\ \cline{2-3}
		& Base de Datos con MariaDB & 0 Gs. \\ \cline{2-3}
		& Eclipse IDE Enterprise Edition & 0 Gs. \\ \cline{2-3}
		& Spring (framework) con Java JDK 21 & 0 Gs. \\ \cline{2-3}
		& PlantUML & 0 Gs. \\ \cline{2-3}
		& Podman y Podman Desktop & 0 Gs. \\ \cline{2-3}
		& TeXStudio y LaTeX & 0 Gs. \\ \hline
	\end{tabularx}
	\caption{Presupuesto parcial del proyecto}
\end{table}

\clearpage

\subsection{Diagrama de Actividades}

Se expondrá el siguiente diagrama de Gantt que demuestra las actividades realizadas durante el año.

\begin{figure}[H]
	\centering
	\rotatebox{90}{\includegraphics[width=0.75\textheight]{images/Gantt.png}}
	\caption{Diagrama de actividades hecho con Mermaid y MarkDown.}
	\label{fig:estructura-vertical}
\end{figure}

\clearpage
