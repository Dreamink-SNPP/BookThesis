% ══════════════════════════════════════════════════════════════════
% CAPÍTULO I - MARCO INTRODUCTORIO
% ══════════════════════════════════════════════════════════════════

\clearpage
\thispagestyle{empty}
\vspace*{\fill}
\begin{center}
    {\Large\bfseries CAPÍTULO I - MARCO INTRODUCTORIO}
\end{center}
\vspace*{\fill}
\clearpage

\section*{CAPÍTULO I - MARCO INTRODUCTORIO}
\addcontentsline{toc}{section}{CAPÍTULO I - MARCO INTRODUCTORIO}
\setcounter{section}{1}

\subsection{Planteamiento del Problema}

El proceso de creación de guiones audiovisuales presenta actualmente importantes dificultades en la organización y estructuración de los elementos narrativos esenciales, como personajes y locaciones. Muchos guionistas, especialmente aquellos que trabajan de forma independiente o en equipos pequeños, recurren a métodos manuales -utilizando papel, pizarras y notas sueltas- para conceptualizar y desarrollar sus fichas e historias antes de trasladarlas a un software de guion. Esta práctica, además de ser ineficiente, interrumpe el flujo creativo y genera una duplicación de esfuerzos, ya que la información debe ser transcrita posteriormente a programas especializados que, en la mayoría de los casos, no ofrecen herramientas específicas para la gestión previa de fichas narrativas.

A pesar de la existencia de diversos softwares de escritura de guiones, como Final Draft, Celtx o Fade In, estos suelen ser costosos, poco flexibles o no permiten la integración directa de fichas de personajes y locaciones en formatos abiertos como Fountain. Esta situación genera una brecha significativa entre la etapa de conceptualización y la redacción técnica del guion, limitando la eficiencia y la creatividad del proceso. Además, la falta de soluciones Open Source y adaptadas a distintas necesidades narrativas restringe el acceso de nuevos guionistas y equipos independientes a herramientas profesionales y actualizadas.

El problema se agrava por la ausencia de aplicaciones web que permitan diseñar, estructurar y vincular fichas de personajes y locaciones de manera sencilla, personalizada y compatible con estándares de la industria. La carencia de estas funcionalidades obliga a los creadores a buscar soluciones alternativas o a realizar tareas repetitivas, lo que puede derivar en errores, pérdida de información y una menor calidad en los productos finales. Este vacío tecnológico afecta no solo la productividad, sino también la capacidad de innovación y experimentación en el ámbito audiovisual.

Ante este panorama, surge la necesidad de desarrollar una herramienta accesible, flexible y de código abierto que permita a los guionistas crear, organizar y exportar fichas narrativas al formato Fountain de manera eficiente. Un sistema de estas características facilitaría la integración de las etapas creativas y técnicas del proceso de guion, democratizando el acceso a recursos profesionales y optimizando la producción de obras audiovisuales de calidad.

\newpage

\subsection{Preguntas de Investigación}

Aquí se presentarán las preguntas de investigación, como la pregunta general y las específicas.

\subsubsection{Pregunta General}
¿Cómo se manifiesta la dispersión organizativa de estructuras dramáticas en los guionistas del Departamento Central durante el año 2025?

\subsubsection{Preguntas Específicas}
\begin{enumerate}
	\item ¿Cuáles son las funcionalidades clave que debe tener una herramienta para la creación de fichas de personajes y locaciones en el contexto de la escritura de guiones?
	\item ¿Cómo se puede asegurar la compatibilidad y exportación de estos elementos al formato Fountain de manera eficiente?
	\item ¿Qué características de interfaz y usabilidad son necesarias para facilitar el uso de la herramienta por parte de los guionistas?
\end{enumerate}

\newpage

\subsection{Objetivos}

Aquí se demostrarán los objetivos tanto general como específicos.

\subsubsection{Objetivo General}
Describir la dispersión organizativa de estructuras dramáticas en los guionistas del Departamento Central durante el año 2025.

\subsubsection{Objetivos Específicos}
\begin{enumerate}
	\item Desarrollar una plataforma que permita a los usuarios crear fichas detalladas para personajes (nombre, motivaciones, conflictos internos/externos) y locaciones (descripción física, contexto narrativo).
	\item Asegurar que la plataforma pueda exportar estos elementos directamente al formato Fountain para facilitar su integración en software de escritura de guiones.
	\item Diseñar una interfaz intuitiva y fácil de usar para que los guionistas puedan aprovechar al máximo la herramienta sin necesidad de conocimientos técnicos avanzados.
\end{enumerate}

\newpage

\subsection{Justificación}

El proceso de escritura de un guion literario audiovisual es inherentemente complejo, pues exige gestionar de forma detallada personajes, locaciones y estructuras dramáticas. Sin embargo, una gran parte de los guionistas, que trabajan de forma independiente o en equipos pequeños, no cuentan con herramientas digitales que se ajusten realmente a sus necesidades. El software que existe suele ser costoso, rígido y rara vez permite conectar la fase de conceptualización (el diseño de personajes o mundos) con la escritura literaria en formatos estándar como Fountain.

En respuesta a este problema, surge el desarrollo de DREAMINK, una aplicación web de código abierto. Esta herramienta está diseñada para que los guionistas puedan crear y gestionar fichas detalladas de personajes y locaciones, permitiendo luego exportarlas directamente al formato Fountain. Esta función clave no solo agiliza la transición entre el diseño narrativo y la redacción literaria del guion, sino que también fomenta el uso de estándares abiertos, liberando a los creadores de la dependencia de plataformas propietarias y costosas.

El carácter Open Source de DREAMINK es un pilar fundamental del proyecto. Al ser de código abierto, democratiza el acceso a tecnología profesional, eliminando las barreras del costo. Esto es vital en contextos creativos donde los recursos son limitados. Además, fomenta que la propia comunidad de usuarios pueda personalizar y mejorar la herramienta continuamente, permitiendo que se adapte con flexibilidad a distintos géneros, formatos y necesidades de cada proyecto.

DREAMINK también busca optimizar el proceso creativo. Al reducir la fragmentación entre la conceptualización y la escritura, se minimizan los errores y la pérdida de información que ocurren al saltar entre distintos programas. La herramienta centraliza los datos clave de personajes y locaciones, ofreciendo al guionista una visión integral de su proyecto. Esto facilita la toma de decisiones narrativas y contribuye a desarrollar guiones con mayor coherencia y calidad final.

El desarrollo de DREAMINK puede ser una necesidad concreta de los equipos audiovisuales independientes de hoy, para quienes la agilidad y el acceso a la tecnología son claves. Este proyecto no se limita a resolver un problema técnico; su objetivo es impulsar la profesionalización y el crecimiento de guionistas y creadores en diversos entornos, ofreciendo una solución específica, flexible y alineada con los estándares de la industria.

\newpage

\subsection{Limitaciones}

Este sistema presentará las siguientes limitaciones, de acuerdo con su alcance técnico y funcional:

\begin{enumerate}[label=\arabic*., leftmargin=1.5cm, itemsep=0.2em]
	\item Funcionará únicamente en entornos web.
	\item Será de uso exclusivamente local.
	\item No permitirá trabajo colaborativo en modalidad multiusuario.
	\item Cada usuario gestiona su propio proyecto.
	\item Ningún usuario podrá gestionar el proyecto de otros.
	\item Permitirá únicamente la exportación al formato \textit{Fountain}.
	\item No admitirá la importación de formatos privativos (como \textit{FadeIn}, \textit{Final Draft}, \textit{WriterDuet}, etc.).
	\item Se limitará a la estructuración previa del guion, sin abarcar el desarrollo del guion literario.
	\item No cubrirá la elaboración del guion técnico.
	\item La interfaz estará optimizada para pantallas de computadoras, no para dispositivos móviles.
	\item No se integrará con modelos de lenguaje de gran escala (LLMs) ni con herramientas similares.
\end{enumerate}

\newpage

\subsection{Alcance o Delimitación}

Se expresarán estos siguientes puntos para demostrar el alcance técnico y funcional del estudio:

\begin{enumerate}[label=\arabic*., leftmargin=1.5cm, itemsep=0.2em]
	\item Permitirá el uso de cuentas de usuarios mediante correo y contraseña.
	\item Posee el ingreso de cuentas y registro de nuevos usuarios.
	\item Posee recuperación de contraseñas para cuentas de usuario.
	\item Se podrán dar de alta, baja, modificación y consulta de personajes.
	\item Se podrán dar de alta, baja, modificación y consulta de locaciones.
	\item Se podrán dar de alta, baja, modificación y consulta de ideas.
	\item Permite la organización de la estructura del guion en actos, secuencias y escenas.
	\item Cada proyecto tendrá su propio tratamiento, estructura, personajes, locaciones e ideas.
	\item Posibilidad de realizar reportes en PDF para tratamientos, personajes, locaciones e ideas.
	\item Reportes en PDF individuales para cara personaje, locación o idea.
\end{enumerate}

\newpage

\subsection{Presupuesto}

Aquí se expondrán los presupuestos humanos, hardware y software para la realización del proyecto.

\begin{table}[ht]
	\centering
	\renewcommand{\arraystretch}{1.3}
	\begin{tabularx}{\textwidth}{|>{\centering\arraybackslash}p{2.5cm}|>{\raggedright\arraybackslash}X|>{\raggedleft\arraybackslash}p{3cm}|}
		\hline
		\multicolumn{3}{|c|}{\textbf{PRESUPUESTO PARCIAL}} \\ \hline
		\textbf{RECURSOS} & \textbf{DESCRIPCIÓN} & \textbf{IMPORTE} \\ \hline

		\multirow{2}{*}{\textbf{HUMANO}}
		& Investigación del trabajo con Internet & 500.000 Gs. \\ \cline{2-3}
		& Viáticos & 100.000 Gs. \\ \hline

		\multirow{3}{*}{\textbf{HARDWARE}}
		& Computadoras personales & 0 Gs. \\ \cline{2-3}
		& Pendrive USB 16 GB & 0 Gs. \\ \cline{2-3}
		& Impresoras / Papelería & 150.000 Gs. \\ \hline

		\multirow{7}{*}{\textbf{SOFTWARE}}
		& Sistemas Operativos GNU/Linux y Windows & 0 Gs. \\ \cline{2-3}
		& Base de Datos con MariaDB & 0 Gs. \\ \cline{2-3}
		& Eclipse IDE Enterprise Edition & 0 Gs. \\ \cline{2-3}
		& Spring (framework) con Java JDK 21 & 0 Gs. \\ \cline{2-3}
		& PlantUML & 0 Gs. \\ \cline{2-3}
		& Podman y Podman Desktop & 0 Gs. \\ \cline{2-3}
		& TeXStudio y LaTeX & 0 Gs. \\ \hline
	\end{tabularx}
	\caption{Presupuesto parcial del proyecto}
\end{table}

\clearpage

\subsection{Diagrama de Actividades}

Se expondrá el siguiente diagrama de Gantt que demuestra las actividades realizadas durante el año.

\begin{figure}[H]
	\centering
	\rotatebox{90}{\includegraphics[width=0.75\textheight]{images/Gantt.png}}
	\caption{Diagrama de actividades hecho con Mermaid y MarkDown.}
	\label{fig:estructura-vertical}
\end{figure}

\clearpage
