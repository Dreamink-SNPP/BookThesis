% ══════════════════════════════════════════════════════════════════
% CAPÍTULO I - MARCO INTRODUCTORIO
% ══════════════════════════════════════════════════════════════════

\clearpage
\thispagestyle{empty}
\vspace*{\fill}
\begin{center}
    {\Large\bfseries CAPÍTULO I - MARCO INTRODUCTORIO}
\end{center}
\vspace*{\fill}
\clearpage

\phantomsection
\section*{CAPÍTULO I - MARCO INTRODUCTORIO}
\addcontentsline{toc}{section}{CAPÍTULO I - MARCO INTRODUCTORIO}
\setcounter{section}{1}

\subsection{Planteamiento del Problema}

En el mundo entero, los guionistas enfrentaron desafíos durante la pre-escritura, especialmente por la naturaleza no lineal de su trabajo creativo que transitó sin orden específico entre conceptualización, desarrollo de personajes y estructuración narrativa \parencite{chi2025_screenwriters}. La investigación internacional documentó que los principales desafíos incluyeron falta de inspiración, coherencia insuficiente y profundidad de contenido inadecuada \parencite{screenwriting_research_network}.

En América Latina, los guionistas se formaron mayormente de manera autodidacta: en Chile, 32.8\% aprendieron por sí mismos \parencite{chile_guion_cine}, mientras que en Argentina los estudios revelaron la necesidad de analizar cómo organizaron su trabajo creativo \parencite{argentina_telenovelas_guionistas}. Los centros y programas de escritura en la región fueron recientes y las prácticas organizativas de los guionistas constituyeron un área escasamente documentada \parencite{centros_escritura_latam}.

En la República del Paraguay, el sector audiovisual experimentó una institucionalización reciente con iniciativas como el Premio Nacional de Guion del INAP \parencite{inap_premio_guion} y la Ley N° 6106 de Fomento al Audiovisual que reconoció al guionista como autor \parencite{ley_6106_paraguay}. Existieron espacios como la Asociación de Guionistas del Paraguay (KUATIA) y la Residencia del Lago \parencite{residencia_lago_paraguay}, pero los métodos que usaron los guionistas locales durante la pre-escritura no fueron estudiados académicamente.

La sistematización del proceso de pre-escritura de guiones que emplearon los guionistas del Departamento Central no fue estudiada en la literatura académica.

La causa principal fue que la investigación académica priorizó analizar guiones terminados en lugar de estudiar los procesos creativos que los originaron, y la naturaleza personalizada del trabajo de pre-escritura dificultó su observación sistemática.

La consecuencia para los guionistas, especialmente quienes se iniciaron en la profesión, fue que debieron aprender de manera autodidacta sin referencias sobre métodos efectivos de organización, lo que limitó su desarrollo profesional y prolongó innecesariamente su curva de aprendizaje.

El presente estudio describió los métodos de sistematización del proceso de pre-escritura de guiones utilizados por los guionistas del Departamento Central durante 2025, mediante entrevistas semiestructuradas que permitieron conocer sus prácticas reales de organización, herramientas empleadas y dificultades enfrentadas.

\newpage

\subsection{Preguntas de Investigación}

Aquí se presentaron las preguntas de investigación, como la pregunta general y las específicas.

\subsubsection{Pregunta General}
¿Cuáles fueron los métodos de sistematización del proceso de pre-escritura de guiones utilizados por los guionistas del Departamento Central durante el año 2025?

\subsubsection{Preguntas Específicas}
\begin{enumerate}
	\item ¿Cómo organizaron los guionistas los elementos narrativos (personajes, locaciones e ideas) durante la fase de pre-escritura?
	\item ¿Qué herramientas y métodos utilizaron los guionistas para estructurar dramáticamente sus obras antes del guion literario?
	\item ¿Cuáles fueron las principales dificultades que enfrentaron los guionistas en la sistematización del proceso de pre-escritura?
\end{enumerate}

\newpage

\subsection{Objetivos}

Aquí se demostraron los objetivos tanto general como específicos.

\subsubsection{Objetivo General}
Describir los métodos de sistematización del proceso de pre-escritura de guiones utilizados por los guionistas del Departamento Central durante el año 2025.

\subsubsection{Objetivos Específicos}
\begin{enumerate}
	\item Identificar las formas de organización de elementos narrativos (personajes, locaciones e ideas) empleadas por los guionistas durante la fase de pre-escritura.
	\item Caracterizar las herramientas y métodos utilizados por los guionistas para estructurar dramáticamente sus obras antes del guion literario.
	\item Describir las principales dificultades que enfrentaron los guionistas en la sistematización del proceso de pre-escritura.
\end{enumerate}

\newpage

\subsection{Justificación}

Desde el punto de vista teórico, este estudio aportó conocimiento sobre una variable escasamente investigada en un contexto geográfico específico. La sistematización del proceso de pre-escritura de guiones no fue caracterizada en el Departamento Central de Paraguay, constituyendo un vacío de conocimiento sobre cómo los guionistas locales organizaron efectivamente su trabajo creativo. Al describir estos métodos, se generó evidencia empírica que contribuyó al desarrollo de marcos teóricos sobre procesos creativos en guionismo y prácticas profesionales en contextos latinoamericanos.

Desde el punto de vista práctico, el estudio permitió prevenir que los guionistas, especialmente quienes se iniciaron en la profesión, continuaran aprendiendo exclusivamente de forma autodidacta sin referencias metodológicas. Al documentar las prácticas reales de sistematización empleadas por guionistas activos, se generaron referentes concretos que pudieron orientar la formación de nuevos profesionales y reducir la curva de aprendizaje innecesariamente prolongada que caracterizó al sector.

Desde el punto de vista social, los beneficiarios directos fueron los guionistas del Departamento Central, cuyas prácticas profesionales fueron visibilizadas y legitimadas académicamente. Asimismo, se beneficiaron las instituciones educativas que ofrecieron formación en guionismo, las organizaciones gremiales como KUATIA, los programas de desarrollo audiovisual como la Residencia del Lago, y los futuros estudiantes de narrativa audiovisual que accedieron a conocimiento sistematizado sobre métodos de trabajo profesional.

Desde el punto de vista económico, el estudio pudo contribuir a optimizar la inversión pública y privada en políticas de fomento al sector audiovisual. Al identificar las prácticas efectivas de sistematización y las principales dificultades enfrentadas por los guionistas, las instituciones como el INAP pudieron diseñar programas de capacitación, talleres y recursos de apoyo más pertinentes a las necesidades reales de los profesionales, maximizando el impacto de los recursos destinados al desarrollo del sector.

Desde el punto de vista metodológico, el estudio aportó una operacionalización validada de la variable «sistematización del proceso de pre-escritura de guiones» con dimensiones, indicadores e instrumento de recolección de datos específicamente diseñados para este fenómeno. Esta contribución metodológica pudo ser empleada por otros investigadores interesados en estudiar procesos creativos de guionistas en diferentes contextos geográficos, facilitando la réplica y comparación de hallazgos en futuras investigaciones sobre prácticas profesionales en el sector audiovisual.

\newpage

\subsection{Limitaciones}

El presente estudio reconoció las siguientes limitaciones inherentes a su diseño metodológico y alcance.

El tamaño de la muestra fue reducido, comprendiendo únicamente tres guionistas del Departamento Central. Esta característica, propia de los estudios cualitativos descriptivos que priorizaron la profundidad sobre la amplitud, limitó la posibilidad de generalización estadística de los hallazgos a poblaciones más extensas de guionistas.

La delimitación geográfica al Departamento Central excluyó las experiencias y prácticas de guionistas que trabajaron en otros departamentos de Paraguay, potencialmente omitiendo variaciones regionales en los métodos de sistematización del proceso de pre-escritura.

La temporalidad específica del estudio, circunscrita al año 2025, capturó las prácticas actuales pero no permitió observar evoluciones, cambios o tendencias en los métodos de sistematización a lo largo del tiempo.

La dependencia de entrevistas como único instrumento de recolección de datos implicó que los hallazgos se basaron en las percepciones y relatos autorreportados de los participantes, sin observación directa de sus prácticas de trabajo, lo que pudo introducir sesgos de deseabilidad social o limitaciones de memoria.

La voluntariedad de la participación pudo generar un sesgo de autoselección, donde los guionistas que aceptaron participar pudieron tener mayor interés en la sistematización de su trabajo o mayor apertura para compartir sus métodos, en comparación con aquellos que declinaron participar.

El enfoque cualitativo descriptivo, si bien permitió caracterizar en profundidad los métodos empleados, no estableció relaciones causales ni evaluó la efectividad de diferentes métodos de sistematización, limitándose a describir las prácticas existentes sin valoraciones de su eficacia.

\newpage

\subsection{Alcance o Delimitación}

El presente estudio delimitó su alcance en tres dimensiones fundamentales que definieron el contexto específico de la investigación.

Desde el punto de vista geográfico, el estudio se circunscribió exclusivamente al Departamento Central de Paraguay. Esta delimitación territorial respondió a la concentración de actividad audiovisual profesional en esta zona, que incluyó las principales instituciones, organizaciones gremiales y programas de desarrollo como KUATIA y la Residencia del Lago. Los hallazgos del estudio reflejaron las prácticas de sistematización de guionistas que trabajaron en este contexto geográfico específico, sin extenderse a otros departamentos del país.

Desde el punto de vista temporal, la investigación se desarrolló durante el año 2025. Este alcance temporal determinó que las prácticas, métodos y herramientas de sistematización descritas correspondieron al estado actual del sector en este período específico. La delimitación temporal permitió capturar las condiciones contemporáneas del trabajo de pre-escritura de guionistas, incluyendo las tecnologías digitales disponibles, las metodologías en uso y los desafíos actuales del sector audiovisual paraguayo.

Desde el punto de vista social, el estudio se enfocó en guionistas activos del Departamento Central que tuvieron experiencia comprobada en escritura de guiones audiovisuales y que voluntariamente aceptaron participar en el estudio. Esta delimitación poblacional especificó que los participantes fueron profesionales o practicantes con trayectoria en guionismo, excluyendo estudiantes sin experiencia práctica o aficionados sin proyectos desarrollados. La muestra cualitativa de tres guionistas permitió profundizar en sus experiencias individuales, caracterizando la diversidad de métodos empleados en distintos tipos de proyectos audiovisuales (ficción, documental, series) dentro de este grupo social específico.

\clearpage
