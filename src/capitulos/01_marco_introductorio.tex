% ══════════════════════════════════════════════════════════════════
% CAPÍTULO I - MARCO INTRODUCTORIO
% ══════════════════════════════════════════════════════════════════

\clearpage
\thispagestyle{empty}
\vspace*{\fill}
\begin{center}
    {\Large\bfseries CAPÍTULO I - MARCO INTRODUCTORIO}
\end{center}
\vspace*{\fill}
\clearpage

\section*{CAPÍTULO I - MARCO INTRODUCTORIO}
\addcontentsline{toc}{section}{CAPÍTULO I - MARCO INTRODUCTORIO}
\setcounter{section}{1}

\subsection{Planteamiento del Problema}

La investigación sobre procesos creativos en guionismo ha identificado desafíos sistemáticos que enfrentan los escritores audiovisuales durante la fase de desarrollo narrativo. Un estudio presentado en la Conferencia CHI 2025 en Yokohama que entrevistó a 23 guionistas profesionales documentó que los principales desafíos incluyen la falta de inspiración, coherencia insuficiente y profundidad de contenido inadecuada \parencite{chi2025_screenwriters}, evidenciando que el workflow creativo es fundamentalmente no lineal, transitando sin orden específico entre conceptualización, desarrollo de personajes, estructuración narrativa y escritura de diálogos. En Latinoamérica, la realidad revela ausencia de formación sistemática y predominancia del aprendizaje autodidacta: un estudio sobre cine chileno evidenció que la escritura creativa constituye la dimensión menos profesionalizada de la pre-producción, con 32.8\% de guionistas formándose de manera autodidacta \parencite{chile_guion_cine}. En Argentina, la investigación sobre guionistas en telenovelas ha documentado la necesidad de analizar rutinas de trabajo y procesos creativos asociados a la escritura \parencite{argentina_telenovelas_guionistas}, mientras que el estado de los centros y programas de escritura en la región revela que las prácticas organizativas de los guionistas constituyen un área de conocimiento escasamente documentada \parencite{centros_escritura_latam}.

En Paraguay, el desarrollo del sector audiovisual experimenta un proceso de institucionalización reciente. El Instituto Nacional del Audiovisual Paraguayo (INAP) anunció la primera edición del Premio Nacional de Guion para promover la profesionalización de los guionistas \parencite{inap_premio_guion}, mientras que la Ley N° 6106 de Fomento al Audiovisual define al guionista como la persona natural que es autora del guion que sirve de base creativa para la obra audiovisual \parencite{ley_6106_paraguay}. La existencia de la Asociación de Guionistas del Paraguay (KUATIA) y programas como la Residencia del Lago \parencite{residencia_lago_paraguay} manifiestan la emergencia de espacios de desarrollo profesional, aunque la sistematización de los métodos de pre-escritura utilizados por los guionistas locales permanece como un área de conocimiento no documentada académicamente.

La sistematización del proceso de pre-escritura de guiones —entendida como el conjunto de actividades previas a la redacción formal del guion literario donde los guionistas organizan elementos narrativos complejos tales como personajes, locaciones, ideas y estructuras dramáticas— constituye un aspecto fundamental pero escasamente estudiado. Los métodos, herramientas y prácticas organizativas empleados para sistematizar este proceso creativo no han sido suficientemente caracterizados en la literatura académica, particularmente en el Departamento Central de Paraguay. Este vacío de conocimiento se explica por varios factores: la investigación académica ha privilegiado el análisis textual de guiones terminados sobre el estudio de los procesos creativos que los originan; la naturaleza no lineal y personalizada del proceso creativo dificulta su observación mediante métodos tradicionales; la emergencia reciente de espacios académicos dedicados al guionismo en Paraguay ha limitado la investigación empírica local; y la ausencia de marcos teóricos específicamente desarrollados para analizar la sistematización de la pre-escritura ha dificultado su conceptualización y medición. Las consecuencias son múltiples: a nivel académico, se limita el desarrollo de teorías fundamentadas sobre el proceso de pre-escritura; a nivel profesional, se dificulta la formación de nuevos guionistas que recurren predominantemente al aprendizaje autodidacta sin referentes metodológicos claros; a nivel institucional, se limita el diseño de políticas de fomento, programas de capacitación y herramientas adecuadas; y se perpetúa la percepción de que el proceso creativo es exclusivamente intuitivo, desconociendo los aspectos sistemáticos que efectivamente utilizan los guionistas profesionales.

Ante esta problemática, el presente estudio se propone describir los métodos de sistematización del proceso de pre-escritura de guiones utilizados por los guionistas del Departamento Central, caracterizando las formas de organización de elementos narrativos, las herramientas y métodos empleados para estructurar dramáticamente las obras, y las dificultades enfrentadas durante este proceso. A través de un enfoque cualitativo descriptivo mediante entrevistas semiestructuradas aplicadas a guionistas activos, se busca generar conocimiento empírico sobre las prácticas reales de sistematización, contribuyendo a visibilizar los aspectos metodológicos del trabajo creativo y proveer información fundamentada para la formación, el desarrollo profesional y el diseño de políticas de fomento al sector audiovisual paraguayo.

\newpage

\subsection{Preguntas de Investigación}

Aquí se presentarán las preguntas de investigación, como la pregunta general y las específicas.

\subsubsection{Pregunta General}
¿Cuáles son los métodos de sistematización del proceso de pre-escritura de guiones utilizados por los guionistas del Departamento Central durante el año 2025?

\subsubsection{Preguntas Específicas}
\begin{enumerate}
	\item ¿Cómo organizan los guionistas los elementos narrativos (personajes, locaciones e ideas) durante la fase de pre-escritura?
	\item ¿Qué herramientas y métodos utilizan los guionistas para estructurar dramáticamente sus obras antes del guion literario?
	\item ¿Cuáles son las principales dificultades que enfrentan los guionistas en la sistematización del proceso de pre-escritura?
\end{enumerate}

\newpage

\subsection{Objetivos}

Aquí se demostrarán los objetivos tanto general como específicos.

\subsubsection{Objetivo General}
Describir los métodos de sistematización del proceso de pre-escritura de guiones utilizados por los guionistas del Departamento Central durante el año 2025.

\subsubsection{Objetivos Específicos}
\begin{enumerate}
	\item Identificar las formas de organización de elementos narrativos (personajes, locaciones e ideas) empleadas por los guionistas durante la fase de pre-escritura.
	\item Caracterizar las herramientas y métodos utilizados por los guionistas para estructurar dramáticamente sus obras antes del guion literario.
	\item Describir las principales dificultades que enfrentan los guionistas en la sistematización del proceso de pre-escritura.
\end{enumerate}

\newpage

\subsection{Justificación}

El proceso de escritura de un guion literario audiovisual es inherentemente complejo, pues exige gestionar de forma detallada personajes, locaciones y estructuras dramáticas. Sin embargo, una gran parte de los guionistas, que trabajan de forma independiente o en equipos pequeños, no cuentan con herramientas digitales que se ajusten realmente a sus necesidades. El software que existe suele ser costoso, rígido y rara vez permite conectar la fase de conceptualización (el diseño de personajes o mundos) con la escritura literaria en formatos estándar como Fountain.

En respuesta a este problema, surge el desarrollo de DREAMINK, una aplicación web de código abierto. Esta herramienta está diseñada para que los guionistas puedan crear y gestionar fichas detalladas de personajes y locaciones, permitiendo luego exportarlas directamente al formato Fountain. Esta función clave no solo agiliza la transición entre el diseño narrativo y la redacción literaria del guion, sino que también fomenta el uso de estándares abiertos, liberando a los creadores de la dependencia de plataformas propietarias y costosas.

El carácter Open Source de DREAMINK es un pilar fundamental del proyecto. Al ser de código abierto, democratiza el acceso a tecnología profesional, eliminando las barreras del costo. Esto es vital en contextos creativos donde los recursos son limitados. Además, fomenta que la propia comunidad de usuarios pueda personalizar y mejorar la herramienta continuamente, permitiendo que se adapte con flexibilidad a distintos géneros, formatos y necesidades de cada proyecto.

DREAMINK también busca optimizar el proceso creativo. Al reducir la fragmentación entre la conceptualización y la escritura, se minimizan los errores y la pérdida de información que ocurren al saltar entre distintos programas. La herramienta centraliza los datos clave de personajes y locaciones, ofreciendo al guionista una visión integral de su proyecto. Esto facilita la toma de decisiones narrativas y contribuye a desarrollar guiones con mayor coherencia y calidad final.

El desarrollo de DREAMINK puede ser una necesidad concreta de los equipos audiovisuales independientes de hoy, para quienes la agilidad y el acceso a la tecnología son claves. Este proyecto no se limita a resolver un problema técnico; su objetivo es impulsar la profesionalización y el crecimiento de guionistas y creadores en diversos entornos, ofreciendo una solución específica, flexible y alineada con los estándares de la industria.

\newpage

\subsection{Limitaciones}

Este sistema presentará las siguientes limitaciones, de acuerdo con su alcance técnico y funcional:

\begin{enumerate}[label=\arabic*., leftmargin=1.5cm, itemsep=0.2em]
	\item Funcionará únicamente en entornos web.
	\item Será de uso exclusivamente local.
	\item No permitirá trabajo colaborativo en modalidad multiusuario.
	\item Cada usuario gestiona su propio proyecto.
	\item Ningún usuario podrá gestionar el proyecto de otros.
	\item Permitirá únicamente la exportación al formato \textit{Fountain}.
	\item No admitirá la importación de formatos privativos (como \textit{FadeIn}, \textit{Final Draft}, \textit{WriterDuet}, etc.).
	\item Se limitará a la estructuración previa del guion, sin abarcar el desarrollo del guion literario.
	\item No cubrirá la elaboración del guion técnico.
	\item La interfaz estará optimizada para pantallas de computadoras, no para dispositivos móviles.
	\item No se integrará con modelos de lenguaje de gran escala (LLMs) ni con herramientas similares.
\end{enumerate}

\newpage

\subsection{Alcance o Delimitación}

Se expresarán estos siguientes puntos para demostrar el alcance técnico y funcional del estudio:

\begin{enumerate}[label=\arabic*., leftmargin=1.5cm, itemsep=0.2em]
	\item Permitirá el uso de cuentas de usuarios mediante correo y contraseña.
	\item Posee el ingreso de cuentas y registro de nuevos usuarios.
	\item Posee recuperación de contraseñas para cuentas de usuario.
	\item Se podrán dar de alta, baja, modificación y consulta de personajes.
	\item Se podrán dar de alta, baja, modificación y consulta de locaciones.
	\item Se podrán dar de alta, baja, modificación y consulta de ideas.
	\item Permite la organización de la estructura del guion en actos, secuencias y escenas.
	\item Cada proyecto tendrá su propio tratamiento, estructura, personajes, locaciones e ideas.
	\item Posibilidad de realizar reportes en PDF para tratamientos, personajes, locaciones e ideas.
	\item Reportes en PDF individuales para cara personaje, locación o idea.
\end{enumerate}

\newpage

\subsection{Presupuesto}

Aquí se expondrán los presupuestos humanos, hardware y software para la realización del proyecto.

\begin{table}[ht]
	\centering
	\renewcommand{\arraystretch}{1.3}
	\begin{tabularx}{\textwidth}{|>{\centering\arraybackslash}p{2.5cm}|>{\raggedright\arraybackslash}X|>{\raggedleft\arraybackslash}p{3cm}|}
		\hline
		\multicolumn{3}{|c|}{\textbf{PRESUPUESTO PARCIAL}} \\ \hline
		\textbf{RECURSOS} & \textbf{DESCRIPCIÓN} & \textbf{IMPORTE} \\ \hline

		\multirow{2}{*}{\textbf{HUMANO}}
		& Investigación del trabajo con Internet & 500.000 Gs. \\ \cline{2-3}
		& Viáticos & 100.000 Gs. \\ \hline

		\multirow{3}{*}{\textbf{HARDWARE}}
		& Computadoras personales & 0 Gs. \\ \cline{2-3}
		& Pendrive USB 16 GB & 0 Gs. \\ \cline{2-3}
		& Impresoras / Papelería & 150.000 Gs. \\ \hline

		\multirow{7}{*}{\textbf{SOFTWARE}}
		& Sistemas Operativos GNU/Linux y Windows & 0 Gs. \\ \cline{2-3}
		& Base de Datos con MariaDB & 0 Gs. \\ \cline{2-3}
		& Eclipse IDE Enterprise Edition & 0 Gs. \\ \cline{2-3}
		& Spring (framework) con Java JDK 21 & 0 Gs. \\ \cline{2-3}
		& PlantUML & 0 Gs. \\ \cline{2-3}
		& Podman y Podman Desktop & 0 Gs. \\ \cline{2-3}
		& TeXStudio y LaTeX & 0 Gs. \\ \hline
	\end{tabularx}
	\caption{Presupuesto parcial del proyecto}
\end{table}

\clearpage

\subsection{Diagrama de Actividades}

Se expondrá el siguiente diagrama de Gantt que demuestra las actividades realizadas durante el año.

\begin{figure}[H]
	\centering
	\rotatebox{90}{\includegraphics[width=0.75\textheight]{images/Gantt.png}}
	\caption{Diagrama de actividades hecho con Mermaid y MarkDown.}
	\label{fig:estructura-vertical}
\end{figure}

\clearpage
