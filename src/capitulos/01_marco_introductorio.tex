% ══════════════════════════════════════════════════════════════════
% CAPÍTULO I - MARCO INTRODUCTORIO
% ══════════════════════════════════════════════════════════════════

\clearpage
\thispagestyle{empty}
\vspace*{\fill}
\begin{center}
    {\Large\bfseries CAPÍTULO I - MARCO INTRODUCTORIO}
\end{center}
\vspace*{\fill}
\clearpage

\section*{CAPÍTULO I - MARCO INTRODUCTORIO}
\addcontentsline{toc}{section}{CAPÍTULO I - MARCO INTRODUCTORIO}
\setcounter{section}{1}

\subsection{Planteamiento del Problema}

En el mundo entero, los guionistas enfrentan desafíos durante la pre-escritura, especialmente por la naturaleza no lineal de su trabajo creativo que transita sin orden específico entre conceptualización, desarrollo de personajes y estructuración narrativa \parencite{chi2025_screenwriters}. La investigación internacional ha documentado que los principales desafíos incluyen falta de inspiración, coherencia insuficiente y profundidad de contenido inadecuada \parencite{screenwriting_research_network}.

En América Latina, los guionistas se forman mayormente de manera autodidacta: en Chile, 32.8\% aprenden por sí mismos \parencite{chile_guion_cine}, mientras que en Argentina los estudios revelan la necesidad de analizar cómo organizan su trabajo creativo \parencite{argentina_telenovelas_guionistas}. Los centros y programas de escritura en la región son recientes y las prácticas organizativas de los guionistas constituyen un área escasamente documentada \parencite{centros_escritura_latam}.

En la República del Paraguay, el sector audiovisual experimenta una institucionalización reciente con iniciativas como el Premio Nacional de Guion del INAP \parencite{inap_premio_guion} y la Ley N° 6106 de Fomento al Audiovisual que reconoce al guionista como autor \parencite{ley_6106_paraguay}. Existen espacios como la Asociación de Guionistas del Paraguay (KUATIA) y la Residencia del Lago \parencite{residencia_lago_paraguay}, pero los métodos que usan los guionistas locales durante la pre-escritura no han sido estudiados académicamente.

La sistematización del proceso de pre-escritura de guiones que emplean los guionistas del Departamento Central no ha sido estudiada en la literatura académica.

La causa principal es que la investigación académica ha priorizado analizar guiones terminados en lugar de estudiar los procesos creativos que los originan, y la naturaleza personalizada del trabajo de pre-escritura dificulta su observación sistemática.

La consecuencia para los guionistas, especialmente quienes se inician en la profesión, es que deben aprender de manera autodidacta sin referencias sobre métodos efectivos de organización, lo que limita su desarrollo profesional y prolonga innecesariamente su curva de aprendizaje.

El presente estudio describe los métodos de sistematización del proceso de pre-escritura de guiones utilizados por los guionistas del Departamento Central durante 2025, mediante entrevistas semiestructuradas que permitan conocer sus prácticas reales de organización, herramientas empleadas y dificultades enfrentadas.

\newpage

\subsection{Preguntas de Investigación}

Aquí se presentarán las preguntas de investigación, como la pregunta general y las específicas.

\subsubsection{Pregunta General}
¿Cuáles son los métodos de sistematización del proceso de pre-escritura de guiones utilizados por los guionistas del Departamento Central durante el año 2025?

\subsubsection{Preguntas Específicas}
\begin{enumerate}
	\item ¿Cómo organizan los guionistas los elementos narrativos (personajes, locaciones e ideas) durante la fase de pre-escritura?
	\item ¿Qué herramientas y métodos utilizan los guionistas para estructurar dramáticamente sus obras antes del guion literario?
	\item ¿Cuáles son las principales dificultades que enfrentan los guionistas en la sistematización del proceso de pre-escritura?
\end{enumerate}

\newpage

\subsection{Objetivos}

Aquí se demostrarán los objetivos tanto general como específicos.

\subsubsection{Objetivo General}
Describir los métodos de sistematización del proceso de pre-escritura de guiones utilizados por los guionistas del Departamento Central durante el año 2025.

\subsubsection{Objetivos Específicos}
\begin{enumerate}
	\item Identificar las formas de organización de elementos narrativos (personajes, locaciones e ideas) empleadas por los guionistas durante la fase de pre-escritura.
	\item Caracterizar las herramientas y métodos utilizados por los guionistas para estructurar dramáticamente sus obras antes del guion literario.
	\item Describir las principales dificultades que enfrentan los guionistas en la sistematización del proceso de pre-escritura.
\end{enumerate}

\newpage

\subsection{Justificación}

Desde el punto de vista teórico, este estudio aporta conocimiento sobre una variable escasamente investigada en un contexto geográfico específico. La sistematización del proceso de pre-escritura de guiones no ha sido caracterizada en el Departamento Central de Paraguay, constituyendo un vacío de conocimiento sobre cómo los guionistas locales organizan efectivamente su trabajo creativo. Al describir estos métodos, se genera evidencia empírica que contribuye al desarrollo de marcos teóricos sobre procesos creativos en guionismo y prácticas profesionales en contextos latinoamericanos.

Desde el punto de vista práctico, el estudio permite prevenir que los guionistas, especialmente quienes se inician en la profesión, continúen aprendiendo exclusivamente de forma autodidacta sin referencias metodológicas. Al documentar las prácticas reales de sistematización empleadas por guionistas activos, se generan referentes concretos que pueden orientar la formación de nuevos profesionales y reducir la curva de aprendizaje innecesariamente prolongada que caracteriza al sector.

Desde el punto de vista social, los beneficiarios directos son los guionistas del Departamento Central, cuyas prácticas profesionales serán visibilizadas y legitimadas académicamente. Asimismo, se benefician las instituciones educativas que ofrecen formación en guionismo, las organizaciones gremiales como KUATIA, los programas de desarrollo audiovisual como la Residencia del Lago, y los futuros estudiantes de narrativa audiovisual que accederán a conocimiento sistematizado sobre métodos de trabajo profesional.

Desde el punto de vista económico, el estudio puede contribuir a optimizar la inversión pública y privada en políticas de fomento al sector audiovisual. Al identificar las prácticas efectivas de sistematización y las principales dificultades enfrentadas por los guionistas, las instituciones como el INAP podrán diseñar programas de capacitación, talleres y recursos de apoyo más pertinentes a las necesidades reales de los profesionales, maximizando el impacto de los recursos destinados al desarrollo del sector.

Desde el punto de vista metodológico, el estudio aporta una operacionalización validada de la variable «sistematización del proceso de pre-escritura de guiones» con dimensiones, indicadores e instrumento de recolección de datos específicamente diseñados para este fenómeno. Esta contribución metodológica puede ser empleada por otros investigadores interesados en estudiar procesos creativos de guionistas en diferentes contextos geográficos, facilitando la réplica y comparación de hallazgos en futuras investigaciones sobre prácticas profesionales en el sector audiovisual.

\newpage

\subsection{Limitaciones}

El presente estudio reconoce las siguientes limitaciones inherentes a su diseño metodológico y alcance.

El tamaño de la muestra es reducido, comprendiendo únicamente tres guionistas del Departamento Central. Esta característica, propia de los estudios cualitativos descriptivos que priorizan la profundidad sobre la amplitud, limita la posibilidad de generalización estadística de los hallazgos a poblaciones más extensas de guionistas.

La delimitación geográfica al Departamento Central excluye las experiencias y prácticas de guionistas que trabajan en otros departamentos de Paraguay, potencialmente omitiendo variaciones regionales en los métodos de sistematización del proceso de pre-escritura.

La temporalidad específica del estudio, circunscrita al año 2025, captura las prácticas actuales pero no permite observar evoluciones, cambios o tendencias en los métodos de sistematización a lo largo del tiempo.

La dependencia de entrevistas como único instrumento de recolección de datos implica que los hallazgos se basan en las percepciones y relatos autorreportados de los participantes, sin observación directa de sus prácticas de trabajo, lo que puede introducir sesgos de deseabilidad social o limitaciones de memoria.

La voluntariedad de la participación puede generar un sesgo de autoselección, donde los guionistas que aceptan participar podrían tener mayor interés en la sistematización de su trabajo o mayor apertura para compartir sus métodos, en comparación con aquellos que declinan participar.

El enfoque cualitativo descriptivo, si bien permite caracterizar en profundidad los métodos empleados, no establece relaciones causales ni evalúa la efectividad de diferentes métodos de sistematización, limitándose a describir las prácticas existentes sin valoraciones de su eficacia.

\newpage

\subsection{Alcance o Delimitación}

El presente estudio delimita su alcance en tres dimensiones fundamentales que definen el contexto específico de la investigación.

Desde el punto de vista geográfico, el estudio se circunscribe exclusivamente al Departamento Central de Paraguay. Esta delimitación territorial responde a la concentración de actividad audiovisual profesional en esta zona, que incluye las principales instituciones, organizaciones gremiales y programas de desarrollo como KUATIA y la Residencia del Lago. Los hallazgos del estudio reflejarán las prácticas de sistematización de guionistas que trabajan en este contexto geográfico específico, sin extenderse a otros departamentos del país.

Desde el punto de vista temporal, la investigación se desarrolla durante el año 2025. Este alcance temporal determina que las prácticas, métodos y herramientas de sistematización descritas corresponden al estado actual del sector en este período específico. La delimitación temporal permite capturar las condiciones contemporáneas del trabajo de pre-escritura de guionistas, incluyendo las tecnologías digitales disponibles, las metodologías en uso y los desafíos actuales del sector audiovisual paraguayo.

Desde el punto de vista social, el estudio se enfoca en guionistas activos del Departamento Central que tengan experiencia comprobada en escritura de guiones audiovisuales y que voluntariamente acepten participar en el estudio. Esta delimitación poblacional especifica que los participantes son profesionales o practicantes con trayectoria en guionismo, excluyendo estudiantes sin experiencia práctica o aficionados sin proyectos desarrollados. La muestra cualitativa de tres guionistas permite profundizar en sus experiencias individuales, caracterizando la diversidad de métodos empleados en distintos tipos de proyectos audiovisuales (ficción, documental, series) dentro de este grupo social específico.

\newpage

\subsection{Presupuesto}

Aquí se expondrán los presupuestos humanos, hardware y software para la realización del proyecto.

\begin{table}[ht]
	\centering
	\renewcommand{\arraystretch}{1.3}
	\begin{tabularx}{\textwidth}{|>{\centering\arraybackslash}p{2.5cm}|>{\raggedright\arraybackslash}X|>{\raggedleft\arraybackslash}p{3cm}|}
		\hline
		\multicolumn{3}{|c|}{\textbf{PRESUPUESTO PARCIAL}} \\ \hline
		\textbf{RECURSOS} & \textbf{DESCRIPCIÓN} & \textbf{IMPORTE} \\ \hline

		\multirow{2}{*}{\textbf{HUMANO}}
		& Investigación del trabajo con Internet & 500.000 Gs. \\ \cline{2-3}
		& Viáticos & 100.000 Gs. \\ \hline

		\multirow{3}{*}{\textbf{HARDWARE}}
		& Computadoras personales & 0 Gs. \\ \cline{2-3}
		& Pendrive USB 16 GB & 0 Gs. \\ \cline{2-3}
		& Impresoras / Papelería & 150.000 Gs. \\ \hline

		\multirow{7}{*}{\textbf{SOFTWARE}}
		& Sistemas Operativos GNU/Linux y Windows & 0 Gs. \\ \cline{2-3}
		& Base de Datos con MariaDB & 0 Gs. \\ \cline{2-3}
		& Eclipse IDE Enterprise Edition & 0 Gs. \\ \cline{2-3}
		& Spring (framework) con Java JDK 21 & 0 Gs. \\ \cline{2-3}
		& PlantUML & 0 Gs. \\ \cline{2-3}
		& Podman y Podman Desktop & 0 Gs. \\ \cline{2-3}
		& TeXStudio y LaTeX & 0 Gs. \\ \hline
	\end{tabularx}
	\caption{Presupuesto parcial del proyecto}
\end{table}

\clearpage

\subsection{Diagrama de Actividades}

Se expondrá el siguiente diagrama de Gantt que demuestra las actividades realizadas durante el año.

\begin{figure}[H]
	\centering
	\rotatebox{90}{\includegraphics[width=0.75\textheight]{images/Gantt.png}}
	\caption{Diagrama de actividades hecho con Mermaid y MarkDown.}
	\label{fig:estructura-vertical}
\end{figure}

\clearpage
