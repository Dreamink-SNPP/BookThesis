% ══════════════════════════════════════════════════════════════════
% CAPÍTULO III - MARCO METODOLÓGICO
% ══════════════════════════════════════════════════════════════════

\clearpage
\thispagestyle{empty}
\vspace*{\fill}
\begin{center}
    {\Large\bfseries CAPÍTULO III - MARCO METODOLÓGICO}
\end{center}
\vspace*{\fill}
\clearpage

\section*{CAPÍTULO III - MARCO METODOLÓGICO}
\addcontentsline{toc}{section}{CAPÍTULO III - MARCO METODOLÓGICO}
\setcounter{section}{3}

\subsection{Tipo de Investigación}

El presente estudio emplea un método de investigación \textbf{básico}, orientado a la comprensión y descripción de fenómenos relacionados con la dispersión organizativa de estructuras dramáticas en el contexto de la escritura de guiones audiovisuales.

Según la profundidad, la investigación es de tipo \textbf{descriptiva}, dirigida a caracterizar y describir cómo se manifiesta la dispersión organizativa de estructuras dramáticas en los guionistas del Departamento Central.

El método utilizado es la observación directa y el análisis cualitativo de contenidos, permitiendo comprender en profundidad las prácticas y desafíos que enfrentan los guionistas en su proceso creativo.

El alcance temporal es de diseño tipo \textbf{transversal}, dado que el estudio se realiza en un único periodo de tiempo, específicamente durante el año 2025.

\subsection{Diseño de la Investigación}

El diseño de la investigación es \textbf{descriptivo simple}, centrado en la observación y descripción del fenómeno sin manipulación de variables. Este diseño permite caracterizar la dispersión organizativa tal como se presenta naturalmente en el contexto estudiado.

\subsection{Enfoque}

El estudio se enmarca dentro del enfoque \textbf{cualitativo}, privilegiando la comprensión profunda de las experiencias, percepciones y prácticas de los guionistas en relación con la organización de estructuras dramáticas, mediante técnicas de recolección de datos que permiten capturar la riqueza y complejidad del fenómeno.

\subsection{Nivel de la Investigación}

El nivel de la investigación es de tipo \textbf{observacional descriptivo}, enfocado en la observación sistemática y descripción detallada del fenómeno estudiado sin intervención experimental. Este nivel permite caracterizar y documentar las manifestaciones de la dispersión organizativa en su contexto natural.

\subsection{Universo de la Investigación}

El universo de la investigación está conformado por el total de guionistas que desarrollan su actividad profesional en el Departamento Central de la República del Paraguay, específicamente aquellos dedicados a la creación de guiones audiovisuales cinematográficos.

\subsection{Población}

La población está constituida por los guionistas cinematográficos activos del Departamento Central durante el año 2025, quienes presentan características relacionadas con la problemática de dispersión organizativa en sus estructuras dramáticas.

\subsection{Muestra}

La muestra está conformada por \textbf{tres (3) guionistas} del Departamento Central, seleccionados mediante muestreo aleatorio simple. Este tamaño muestral es apropiado para el enfoque cualitativo descriptivo del estudio, permitiendo una exploración profunda de las experiencias y prácticas individuales de cada participante.

\textbf{Criterios de inclusión:}
\begin{itemize}
    \item Guionistas activos en el Departamento Central
    \item Experiencia demostrable en escritura de guiones cinematográficos
    \item Disposición voluntaria para participar en el estudio
\end{itemize}

\subsection{Técnica e Instrumento de Recolección de Datos}

La técnica de recolección de datos empleada es la \textbf{entrevista}, que permite obtener información cualitativa profunda sobre las experiencias, percepciones y prácticas de los guionistas participantes.

El instrumento utilizado es una \textbf{guía de entrevista semiestructurada}, diseñada específicamente para medir teóricamente la variable principal del estudio: la dispersión organizativa de estructuras dramáticas. Este instrumento contiene preguntas abiertas que permiten a los entrevistados expresar libremente sus experiencias y perspectivas, facilitando la comprensión profunda del fenómeno estudiado.

\subsection{Técnica de Análisis de los Datos}

Dado el enfoque cualitativo del estudio, se empleará la técnica de \textbf{análisis de contenido cualitativo}, mediante la cual se identificarán categorías, patrones y temas emergentes de las transcripciones de las entrevistas. Este análisis permitirá caracterizar y describir las manifestaciones de la dispersión organizativa en las estructuras dramáticas utilizadas por los guionistas participantes.

El procesamiento de los datos incluirá:
\begin{itemize}
    \item Transcripción de entrevistas
    \item Codificación y categorización de contenidos
    \item Identificación de patrones y temas recurrentes
    \item Interpretación y descripción del fenómeno estudiado
\end{itemize}

\newpage
