% ══════════════════════════════════════════════════════════════════
% CAPÍTULO III - MARCO METODOLÓGICO
% ══════════════════════════════════════════════════════════════════

\section{CAPÍTULO III - MARCO METODOLÓGICO}

\subsection{Tipo de Investigación}

El presente estudio emplea un método de investigación \textbf{básico}, dado que tiene como objetivo el desarrollo de una herramienta tecnológica orientada a resolver un problema concreto identificado en el proceso de pre--escritura de guiones audiovisuales, particularmente en la etapa de estructuración narrativa.

Según la profundidad, puede ser \textbf{descriptiva} al ser dirigidos a determinar cómo es y cómo está la forma actual de realizar una etapa de estructuración narrativa previa al guion literario.

El método que fue utilizado era la observación participativa y el apoyo de análisis del problema y de los contenidos, así como también la observación.

El alcance temporal de diseño tipo transversal (porque el estudio se realiza en un solo periodo), entre los meses de febrero a noviembre del año 2025.

\subsection{Diseño de la Investigación}

El diseño de la investigación es \textbf{experimental}, porque se realiza previo a los datos a ser recolectados mediante encuestas y las pruebas de uso de la aplicación.

\subsection{Enfoque}

El estudio se encuentra dentro del enfoque mixto, porque se pueden utilizar los tipos de datos cuantitativos y cualitativos.

\subsection{Nivel de la Investigación}

El nivel de la investigación es de tipo \textbf{explicativo}, porque nos enfocaremos en conocer los problemas que rodean a la investigación y de allí partir a desarrollar una solución.

\subsection{Universo de la Investigación}

El universo de la investigación es para diferentes guionistas paraguayos elegidos de manera aleatoria, que se dediquen explícitamente al cine dentro de la República del Paraguay.

\subsection{Población}

La población va a ser para los guionistas paraguayos que serán elegidos de manera aleatoria dentro de la República del Paraguay.

\subsection{Muestra}

La muestra será la población de esos guionistas paraguayos elegidos de manera aleatoria.

\subsection{Instrumento de Recolección de Datos}

El instrumento de recolección de datos será utilizado las encuestas y las observaciones, sobre las encuestas se puede usar preguntas cerradas para obtener la información de la población.

\subsection{Técnica de Análisis de los Datos}

La técnica de análisis de los datos que se podrán usar es la cuantitativa y la cualitativa, y todos los datos serán procesados de manera gráfica y el uso de formularios electrónicos.

\newpage
