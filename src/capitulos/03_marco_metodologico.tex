% ══════════════════════════════════════════════════════════════════
% CAPÍTULO III - MARCO METODOLÓGICO
% ══════════════════════════════════════════════════════════════════

\section{CAPÍTULO III - MARCO METODOLÓGICO}

\subsection{Tipo de Investigación}

El presente trabajo es un proyecto de \textbf{desarrollo tecnológico aplicado} orientado a resolver un problema concreto identificado en el proceso de pre--escritura de guiones audiovisuales, particularmente en la etapa de estructuración narrativa mediante fichas de personajes y locaciones.

Este proyecto combina el desarrollo de software con validación empírica a través de usuarios reales, siguiendo un enfoque de ingeniería de software centrado en el usuario. El componente investigativo es de tipo \textbf{descriptivo}, ya que se busca comprender cómo los guionistas paraguayos estructuran actualmente sus elementos narrativos previos al guion literario, para luego diseñar una solución tecnológica que mejore este proceso.

\subsection{Diseño de la Investigación}

El diseño metodológico se divide en dos etapas complementarias:

\begin{enumerate}[label=\arabic*., leftmargin=1.5cm, itemsep=0.2em]
	\item \textbf{Etapa de levantamiento de información:} Recolección de datos sobre las prácticas actuales de estructuración narrativa mediante encuestas y observaciones a guionistas paraguayos.
	\item \textbf{Etapa de desarrollo y validación:} Diseño, implementación y pruebas de usabilidad de la aplicación DREAMINK, utilizando metodologías ágiles de desarrollo de software.
\end{enumerate}

Este diseño permite iterar entre la comprensión del problema y el desarrollo de la solución, asegurando que el producto final responda a necesidades reales.

\subsection{Enfoque}

El proyecto adopta un \textbf{enfoque mixto}, combinando:

\begin{itemize}[label=---, leftmargin=1.5cm, itemsep=0.2em]
	\item \textbf{Datos cuantitativos:} Obtenidos mediante encuestas con preguntas cerradas para identificar patrones de uso, frecuencias y preferencias sobre herramientas y flujos de trabajo.
	\item \textbf{Datos cualitativos:} Obtenidos mediante observación directa durante las pruebas de usabilidad, donde se registran comportamientos, dificultades y comentarios de los usuarios al interactuar con la aplicación.
\end{itemize}

Este enfoque mixto es característico de proyectos de desarrollo de software centrado en el usuario (UX), donde se necesita tanto información medible como comprensión profunda del contexto de uso.

\subsection{Alcance Temporal}

El proyecto se desarrolla bajo un diseño \textbf{transversal}, realizándose durante un periodo único comprendido entre los meses de febrero a noviembre del año 2025. Durante este tiempo se llevarán a cabo las etapas de investigación, desarrollo, implementación y validación de la herramienta.

\subsection{Universo de la Investigación}

El universo del estudio está conformado por guionistas audiovisuales que trabajan en la República del Paraguay, tanto de forma independiente como en productoras o colectivos creativos, y que se encuentren activos en la creación de contenido para cine, televisión o plataformas digitales.

\subsection{Población}

La población objetivo corresponde a guionistas paraguayos dedicados a la escritura de guiones audiovisuales, con experiencia en el proceso de estructuración narrativa previo a la redacción del guion literario. Se incluyen tanto profesionales establecidos como guionistas emergentes o estudiantes de áreas afines que practiquen activamente la escritura de guiones.

\subsection{Muestra}

La muestra será seleccionada de forma no probabilística por conveniencia, considerando la disponibilidad y accesibilidad de los participantes. Se buscará incluir un grupo diverso de guionistas que representen diferentes niveles de experiencia, géneros narrativos y contextos de producción (independiente, productoras pequeñas, educativo).

El tamaño de la muestra se determinará en función de la saturación de información obtenida, buscando un mínimo de participantes que permita identificar patrones claros en las prácticas actuales y validar la usabilidad de la herramienta desarrollada.

\subsection{Instrumentos de Recolección de Datos}

Para el levantamiento de información y validación de la herramienta, se utilizarán los siguientes instrumentos:

\subsubsection{Encuestas}

Se diseñará un cuestionario estructurado con preguntas cerradas y de opción múltiple, dirigido a caracterizar las prácticas actuales de los guionistas en la etapa de estructuración narrativa. Los temas a abordar incluyen:

\begin{itemize}[label=---, leftmargin=1.5cm, itemsep=0.2em]
	\item Herramientas actualmente utilizadas para organizar personajes y locaciones
	\item Dificultades enfrentadas en el proceso de pre--escritura
	\item Nivel de familiaridad con el formato Fountain
	\item Expectativas sobre funcionalidades de una herramienta digital
	\item Preferencias de interfaz y flujo de trabajo
\end{itemize}

Las encuestas se administrarán mediante formularios electrónicos (Google Forms, Microsoft Forms u otra plataforma similar), facilitando la tabulación y análisis posterior de los datos.

\subsubsection{Observación Directa}

Durante las pruebas de usabilidad de la aplicación DREAMINK, se realizará observación participativa estructurada, registrando:

\begin{itemize}[label=---, leftmargin=1.5cm, itemsep=0.2em]
	\item Tiempo requerido para completar tareas específicas (crear un personaje, exportar a Fountain, etc.)
	\item Errores cometidos durante la interacción con la interfaz
	\item Puntos de confusión o frustración expresados por el usuario
	\item Comentarios espontáneos sobre la experiencia de uso
	\item Sugerencias de mejora manifestadas durante la prueba
\end{itemize}

Estas observaciones se documentarán mediante notas de campo y, de ser posible, grabaciones de pantalla con consentimiento del participante.

\subsubsection{Entrevistas (Opcional)}

De considerarse necesario, se podrán realizar entrevistas semi--estructuradas breves con algunos participantes para profundizar en aspectos específicos de su experiencia, tanto con las prácticas actuales como con el uso de la aplicación desarrollada.

\subsection{Técnica de Análisis de los Datos}

Los datos recolectados serán procesados utilizando técnicas de análisis mixto:

\subsubsection{Análisis Cuantitativo}

Los datos de las encuestas serán tabulados y procesados mediante:

\begin{itemize}[label=---, leftmargin=1.5cm, itemsep=0.2em]
	\item Estadística descriptiva (frecuencias, porcentajes, promedios)
	\item Representación gráfica mediante tablas, gráficos de barras y diagramas circulares
	\item Uso de hojas de cálculo (Excel, Google Sheets) o herramientas de análisis de formularios
\end{itemize}

Este análisis permitirá identificar tendencias, patrones de uso y necesidades comunes entre los guionistas encuestados.

\subsubsection{Análisis Cualitativo}

Los datos de las observaciones y eventuales entrevistas serán analizados mediante:

\begin{itemize}[label=---, leftmargin=1.5cm, itemsep=0.2em]
	\item Categorización de problemas de usabilidad identificados
	\item Análisis temático de comentarios y sugerencias
	\item Identificación de patrones de comportamiento durante el uso de la aplicación
	\item Priorización de mejoras según frecuencia e impacto
\end{itemize}

Los resultados de ambos análisis se integrarán para fundamentar tanto el diseño inicial de DREAMINK como las iteraciones de mejora posteriores a las pruebas de usabilidad.

\newpage
