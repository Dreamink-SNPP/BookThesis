% ══════════════════════════════════════════════════════════════════
% CAPÍTULO III - MARCO METODOLÓGICO
% ══════════════════════════════════════════════════════════════════

\clearpage
\thispagestyle{empty}
\vspace*{\fill}
\begin{center}
    {\Large\bfseries CAPÍTULO III - MARCO METODOLÓGICO}
\end{center}
\vspace*{\fill}
\clearpage

\section*{CAPÍTULO III - MARCO METODOLÓGICO}
\addcontentsline{toc}{section}{CAPÍTULO II - MARCO TEÓRICO}
\setcounter{section}{2}

\subsection{Tipo de investigación}

El presente estudio empleó un método de investigación \textbf{básico}, orientado a la comprensión y descripción de fenómenos relacionados con la sistematización del proceso de pre-escritura de guiones en el contexto de la creación audiovisual.

Según la profundidad, la investigación fue de tipo \textbf{descriptiva}, dirigida a caracterizar y describir los métodos de sistematización del proceso de pre-escritura de guiones utilizados por los guionistas del Departamento Central.

El método utilizado fue la observación directa y el análisis cualitativo de contenidos, permitiendo comprender en profundidad las prácticas y desafíos que enfrentaron los guionistas en su proceso creativo.

El alcance temporal fue de diseño tipo \textbf{transversal}, dado que el estudio se realizó en un único periodo de tiempo, específicamente durante el año 2025.

\subsubsection{Enfoque}

El estudio se enmarcó dentro del enfoque \textbf{cualitativo}, privilegiando la comprensión profunda de las experiencias, percepciones y prácticas de los guionistas en relación con la sistematización del proceso de pre-escritura de guiones, mediante técnicas de recolección de datos que permitieron capturar la riqueza y complejidad del fenómeno.

\subsection{Diseño de la investigación}

El diseño de la investigación fue \textbf{descriptivo simple}, centrado en la observación y descripción del fenómeno sin manipulación de variables. Este diseño permitió caracterizar los métodos de sistematización del proceso de pre-escritura tal como se presentaron naturalmente en el contexto estudiado.

\subsection{Nivel de investigación}

El nivel de la investigación fue de tipo \textbf{observacional descriptivo}, enfocado en la observación sistemática y descripción detallada del fenómeno estudiado sin intervención experimental. Este nivel permitió caracterizar y documentar los métodos de sistematización del proceso de pre-escritura utilizados por los guionistas en su contexto natural.

\clearpage

\subsection{Universo del discurso}

El universo de la investigación estuvo conformado por el total de guionistas que desarrollaron su actividad profesional en el Departamento Central de la República del Paraguay, específicamente aquellos dedicados a la creación de guiones audiovisuales cinematográficos.

\subsection{Población muestra}

\subsubsection{Población}

La población estuvo constituida por los guionistas cinematográficos activos del Departamento Central durante el año 2025, quienes desarrollaron procesos de sistematización en la fase de pre-escritura de sus guiones.

\subsubsection{Muestra}

La muestra estuvo conformada por \textbf{tres (3) guionistas} del Departamento Central, seleccionados mediante muestreo aleatorio simple. Este tamaño muestral fue apropiado para el enfoque cualitativo descriptivo del estudio, permitiendo una exploración profunda de las experiencias y prácticas individuales de cada participante.

\textbf{Criterios de inclusión:}
\begin{itemize}
    \item Guionistas activos en el Departamento Central
    \item Experiencia demostrable en escritura de guiones cinematográficos
    \item Disposición voluntaria para participar en el estudio
\end{itemize}

\subsection{Instrumentos de recolección de datos}

La técnica de recolección de datos empleada fue la \textbf{entrevista}, que permitió obtener información cualitativa profunda sobre las experiencias, percepciones y prácticas de los guionistas participantes.

El instrumento utilizado fue una \textbf{guía de entrevista semiestructurada}, diseñada específicamente para explorar la variable principal del estudio: la sistematización del proceso de pre-escritura de guiones. Este instrumento contuvo preguntas abiertas que permitieron a los entrevistados expresar libremente sus experiencias, métodos, herramientas y perspectivas sobre sus procesos de organización narrativa, facilitando la comprensión profunda del fenómeno estudiado.

\clearpage

\subsection{Técnica de análisis de los datos, pasos o procedimientos a seguir}

Dado el enfoque cualitativo del estudio, se empleó la técnica de \textbf{análisis de contenido cualitativo}, mediante la cual se identificaron categorías, patrones y temas emergentes de las transcripciones de las entrevistas. Este análisis permitió caracterizar y describir los métodos de sistematización del proceso de pre-escritura de guiones utilizados por los guionistas participantes.

El procesamiento de los datos incluyó:
\begin{itemize}
    \item Transcripción de entrevistas
    \item Codificación y categorización de contenidos
    \item Identificación de patrones y temas recurrentes
    \item Interpretación y descripción del fenómeno estudiado
\end{itemize}

\newpage
