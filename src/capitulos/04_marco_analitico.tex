% ══════════════════════════════════════════════════════════════════
% CAPÍTULO IV - MARCO ANALÍTICO
% ══════════════════════════════════════════════════════════════════
% TEMPLATE: Este capítulo debe completarse con datos reales de entrevistas
% Siga las instrucciones en cada sección marcadas con [COMPLETAR]
% ══════════════════════════════════════════════════════════════════

\clearpage
\thispagestyle{empty}
\vspace*{\fill}
\begin{center}
    {\Large\bfseries CAPÍTULO IV - MARCO ANALÍTICO}
\end{center}
\vspace*{\fill}
\clearpage

\section*{CAPÍTULO IV - MARCO ANALÍTICO}
\addcontentsline{toc}{section}{CAPÍTULO IV - MARCO ANALÍTICO}
\setcounter{section}{4}

% ══════════════════════════════════════════════════════════════════
% 4.1. CARACTERIZACIÓN DE LOS PARTICIPANTES
% ══════════════════════════════════════════════════════════════════

\subsection{Caracterización de los Participantes}

% [INSTRUCCIÓN]: Completar con información real de los 3 guionistas entrevistados
% Use códigos E1, E2, E3 para mantener anonimato según consentimiento informado

El presente estudio se realizó con la participación de tres guionistas del Departamento Central de Paraguay, quienes aceptaron voluntariamente colaborar en la investigación mediante la firma del consentimiento informado. A continuación se presentó la caracterización de los participantes, utilizando códigos para preservar su anonimato conforme a los principios éticos de la investigación.

\begin{table}[ht]
	\centering
	\renewcommand{\arraystretch}{1.3}
	\begin{tabular}{|c|c|p{4cm}|p{3.5cm}|}
		\hline
		\textbf{Código} & \textbf{Experiencia} & \textbf{Tipo de proyectos} & \textbf{Formación} \\ \hline

		E1 & [COMPLETAR: ej. 5-10 años] & [COMPLETAR: ej. Ficción, series web] & [COMPLETAR: ej. Talleres + autodidacta] \\ \hline

		E2 & [COMPLETAR: ej. 3-5 años] & [COMPLETAR: ej. Documental, cortometrajes] & [COMPLETAR: ej. Autodidacta] \\ \hline

		E3 & [COMPLETAR: ej. 10+ años] & [COMPLETAR: ej. Largometrajes, publicidad] & [COMPLETAR: ej. Cursos universitarios] \\ \hline
	\end{tabular}
	\caption{Caracterización de los participantes del estudio}
	\label{tab:participantes}
\end{table}

% [INSTRUCCIÓN]: Agregar descripción narrativa de cada participante
% Incluir: trayectoria, proyectos relevantes, contexto profesional actual
% Extensión sugerida: 1-2 párrafos por participante

\textbf{Participante E1:} [COMPLETAR: Descripción narrativa del perfil profesional. Ejemplo: Se trata de un guionista con experiencia en ficción televisiva, quien inició su carrera de manera autodidacta y posteriormente consolidó su formación mediante talleres especializados...]

\textbf{Participante E2:} [COMPLETAR: Descripción narrativa del perfil profesional]

\textbf{Participante E3:} [COMPLETAR: Descripción narrativa del perfil profesional]

\newpage

% ══════════════════════════════════════════════════════════════════
% 4.2. PROCESAMIENTO DE LOS DATOS
% ══════════════════════════════════════════════════════════════════

\subsection{Procesamiento de los Datos}

\subsubsection{Procedimiento de Recolección de Datos}

% [INSTRUCCIÓN]: Completar con fechas y condiciones reales de aplicación

Las entrevistas se realizaron durante el período comprendido entre [COMPLETAR: mes] y [COMPLETAR: mes] de 2025. Cada entrevista tuvo una duración aproximada de [COMPLETAR: 50-60] minutos y se llevó a cabo en modalidad [COMPLETAR: presencial/virtual], respetando las preferencias y disponibilidad de cada participante.

Previo a cada entrevista, se procedió a la lectura y firma del consentimiento informado, garantizando la comprensión plena de los objetivos del estudio, la voluntariedad de la participación, el anonimato de las respuestas y el uso exclusivamente académico de la información recopilada. Todas las entrevistas fueron grabadas en audio con la autorización expresa de los participantes, utilizando [COMPLETAR: dispositivo de grabación utilizado] y respaldo en [COMPLETAR: dispositivo de respaldo].

El instrumento utilizado fue una guía de entrevista semiestructurada, diseñada a partir de la matriz de operacionalización de variables, la cual permitió explorar las cuatro dimensiones establecidas: organización de elementos narrativos, estructuración dramática, métodos y herramientas de sistematización, y desafíos y dificultades en el proceso de pre-escritura.

\subsubsection{Proceso de Transcripción y Codificación}

% [INSTRUCCIÓN]: Completar con método real de transcripción utilizado

Las grabaciones de audio fueron transcritas de manera literal (verbatim) dentro de las 72 horas posteriores a cada entrevista, siguiendo el protocolo de aplicación del instrumento. La transcripción incluyó las respuestas textuales de los participantes, pausas significativas y énfasis relevantes para el análisis posterior.

Cada participante fue codificado con un identificador alfanumérico (E1, E2, E3) para preservar su anonimato, eliminando toda información identificable de las transcripciones. Las transcripciones fueron almacenadas en formato [COMPLETAR: ej. Markdown/Word] y respaldadas en [COMPLETAR: sistema de almacenamiento seguro].

\subsubsection{Técnica de Análisis de Contenido Cualitativo}

El análisis de los datos se realizó mediante la técnica de análisis de contenido cualitativo, siguiendo un proceso sistemático de tres fases:

\textbf{Primera fase - Lectura exhaustiva:} Se realizó una lectura completa y repetida de las tres transcripciones para familiarizarse con el contenido global de las entrevistas y obtener una comprensión holística de las experiencias relatadas por los participantes.

\textbf{Segunda fase - Codificación:} Se aplicó un proceso de codificación en dos etapas. La codificación abierta permitió identificar categorías emergentes directamente desde los datos, mientras que la codificación axial estableció relaciones entre las categorías identificadas. Este proceso se organizó según las cuatro dimensiones de la matriz de operacionalización de variables.

\textbf{Tercera fase - Identificación de patrones:} Se identificaron patrones comunes y divergentes entre los tres participantes, triangulando los hallazgos con el marco teórico establecido en el Capítulo II. Esta fase permitió caracterizar las prácticas de sistematización empleadas por los guionistas y detectar convergencias y particularidades en sus métodos de trabajo.

\newpage

% ══════════════════════════════════════════════════════════════════
% 4.3. PRESENTACIÓN DE RESULTADOS POR DIMENSIONES
% ══════════════════════════════════════════════════════════════════

\subsection{Presentación de Resultados por Dimensiones}

% ══════════════════════════════════════════════════════════════════
% 4.3.1. DIMENSIÓN 1: ORGANIZACIÓN DE ELEMENTOS NARRATIVOS
% ══════════════════════════════════════════════════════════════════

\subsubsection{Dimensión 1: Organización de Elementos Narrativos}

Esta dimensión respondió al Objetivo Específico 1: \textit{Identificar las formas de organización de elementos narrativos (personajes, locaciones e ideas) empleadas por los guionistas durante la fase de pre-escritura.}

Los indicadores analizados fueron: gestión de personajes, gestión de locaciones, gestión de ideas y conceptos, y métodos de registro y documentación.

% ──────────────────────────────────────────────────────────────────
% A. GESTIÓN DE PERSONAJES
% ──────────────────────────────────────────────────────────────────

\textbf{A. Gestión de personajes}

% [INSTRUCCIÓN]: Completar con hallazgos reales sobre cómo gestionan personajes
% Incluir: métodos comunes, diferencias, herramientas específicas

[COMPLETAR: Descripción narrativa de los hallazgos. Ejemplo: Los tres guionistas entrevistados manifestaron utilizar fichas de personajes durante la fase de pre-escritura, aunque con niveles de detalle variables. E1 y E3 desarrollaron biografías extensas incluyendo aspectos psicológicos...]

\textit{Citas textuales representativas:}

\begin{quote}
\textbf{E1:} [COMPLETAR: Cita textual de la entrevista sobre gestión de personajes]
\end{quote}

\begin{quote}
\textbf{E2:} [COMPLETAR: Cita textual de la entrevista sobre gestión de personajes]
\end{quote}

\begin{quote}
\textbf{E3:} [COMPLETAR: Cita textual de la entrevista sobre gestión de personajes]
\end{quote}

\textit{Patrones identificados:}

% [INSTRUCCIÓN]: Listar patrones comunes y diferencias observadas

\begin{itemize}
	\item [COMPLETAR: Patrón común 1. Ejemplo: Uso de fichas de personajes con información básica (nombre, edad, ocupación)]
	\item [COMPLETAR: Patrón común 2]
	\item [COMPLETAR: Diferencia individual observada]
\end{itemize}

% ──────────────────────────────────────────────────────────────────
% B. GESTIÓN DE LOCACIONES
% ──────────────────────────────────────────────────────────────────

\textbf{B. Gestión de locaciones}

% [INSTRUCCIÓN]: Completar con hallazgos sobre gestión de locaciones

[COMPLETAR: Descripción narrativa de los hallazgos sobre cómo organizan las locaciones]

\textit{Citas textuales representativas:}

\begin{quote}
\textbf{E1:} [COMPLETAR: Cita textual sobre gestión de locaciones]
\end{quote}

\begin{quote}
\textbf{E2:} [COMPLETAR: Cita textual sobre gestión de locaciones]
\end{quote}

\begin{quote}
\textbf{E3:} [COMPLETAR: Cita textual sobre gestión de locaciones]
\end{quote}

\textit{Patrones identificados:}

\begin{itemize}
	\item [COMPLETAR: Patrón común 1]
	\item [COMPLETAR: Patrón común 2]
	\item [COMPLETAR: Diferencia individual observada]
\end{itemize}

% ──────────────────────────────────────────────────────────────────
% C. GESTIÓN DE IDEAS Y CONCEPTOS
% ──────────────────────────────────────────────────────────────────

\textbf{C. Gestión de ideas y conceptos}

% [INSTRUCCIÓN]: Completar con hallazgos sobre captura y organización de ideas

[COMPLETAR: Descripción narrativa de los hallazgos sobre gestión de ideas]

\textit{Citas textuales representativas:}

\begin{quote}
\textbf{E1:} [COMPLETAR: Cita textual sobre gestión de ideas]
\end{quote}

\begin{quote}
\textbf{E2:} [COMPLETAR: Cita textual sobre gestión de ideas]
\end{quote}

\begin{quote}
\textbf{E3:} [COMPLETAR: Cita textual sobre gestión de ideas]
\end{quote}

\textit{Patrones identificados:}

\begin{itemize}
	\item [COMPLETAR: Patrón común 1]
	\item [COMPLETAR: Patrón común 2]
	\item [COMPLETAR: Diferencia individual observada]
\end{itemize}

% ──────────────────────────────────────────────────────────────────
% D. MÉTODOS DE REGISTRO Y DOCUMENTACIÓN
% ──────────────────────────────────────────────────────────────────

\textbf{D. Métodos de registro y documentación}

% [INSTRUCCIÓN]: Completar con hallazgos sobre formatos y plantillas utilizadas

[COMPLETAR: Descripción narrativa de los hallazgos sobre métodos de registro]

\textit{Citas textuales representativas:}

\begin{quote}
\textbf{E1:} [COMPLETAR: Cita textual sobre métodos de documentación]
\end{quote}

\begin{quote}
\textbf{E2:} [COMPLETAR: Cita textual sobre métodos de documentación]
\end{quote}

\begin{quote}
\textbf{E3:} [COMPLETAR: Cita textual sobre métodos de documentación]
\end{quote}

\textit{Patrones identificados:}

\begin{itemize}
	\item [COMPLETAR: Patrón común 1]
	\item [COMPLETAR: Patrón común 2]
	\item [COMPLETAR: Diferencia individual observada]
\end{itemize}

% ──────────────────────────────────────────────────────────────────
% SÍNTESIS DE LA DIMENSIÓN 1
% ──────────────────────────────────────────────────────────────────

\textbf{Síntesis de la Dimensión 1:}

% [INSTRUCCIÓN]: Escribir párrafo integrador que responda directamente al OE1
% Debe sintetizar los hallazgos principales de los 4 indicadores

[COMPLETAR: Párrafo integrador de 3-5 oraciones que responda: ¿Cómo organizan los guionistas los elementos narrativos durante la pre-escritura? Integrar hallazgos de personajes, locaciones, ideas y métodos de registro.]

\newpage

% ══════════════════════════════════════════════════════════════════
% 4.3.2. DIMENSIÓN 2: ESTRUCTURACIÓN DRAMÁTICA
% ══════════════════════════════════════════════════════════════════

\subsubsection{Dimensión 2: Estructuración Dramática}

Esta dimensión respondió parcialmente al Objetivo Específico 2: \textit{Caracterizar las herramientas y métodos utilizados por los guionistas para estructurar dramáticamente sus obras antes del guion literario.}

Los indicadores analizados fueron: organización en actos, organización en secuencias, organización en escenas y paradigmas estructurales utilizados.

% ──────────────────────────────────────────────────────────────────
% A. PARADIGMAS ESTRUCTURALES UTILIZADOS
% ──────────────────────────────────────────────────────────────────

\textbf{A. Paradigmas estructurales utilizados}

% [INSTRUCCIÓN]: Completar con hallazgos sobre qué paradigmas usan (3 actos, secuencias, etc.)

[COMPLETAR: Descripción narrativa de los paradigmas estructurales empleados]

\textit{Citas textuales representativas:}

\begin{quote}
\textbf{E1:} [COMPLETAR: Cita sobre paradigma estructural utilizado]
\end{quote}

\begin{quote}
\textbf{E2:} [COMPLETAR: Cita sobre paradigma estructural utilizado]
\end{quote}

\begin{quote}
\textbf{E3:} [COMPLETAR: Cita sobre paradigma estructural utilizado]
\end{quote}

\textit{Patrones identificados:}

\begin{itemize}
	\item [COMPLETAR: Ejemplo: Uso predominante del paradigma de tres actos]
	\item [COMPLETAR: Patrón común 2]
	\item [COMPLETAR: Diferencia individual observada]
\end{itemize}

% ──────────────────────────────────────────────────────────────────
% B. ORGANIZACIÓN EN ACTOS, SECUENCIAS Y ESCENAS
% ──────────────────────────────────────────────────────────────────

\textbf{B. Organización en actos, secuencias y escenas}

% [INSTRUCCIÓN]: Completar con hallazgos sobre niveles de organización dramática

[COMPLETAR: Descripción narrativa sobre cómo organizan en actos/secuencias/escenas]

\textit{Citas textuales representativas:}

\begin{quote}
\textbf{E1:} [COMPLETAR: Cita sobre organización estructural]
\end{quote}

\begin{quote}
\textbf{E2:} [COMPLETAR: Cita sobre organización estructural]
\end{quote}

\begin{quote}
\textbf{E3:} [COMPLETAR: Cita sobre organización estructural]
\end{quote}

% ──────────────────────────────────────────────────────────────────
% C. VISUALIZACIÓN DE LA ESTRUCTURA
% ──────────────────────────────────────────────────────────────────

\textbf{C. Visualización de la estructura dramática}

% [INSTRUCCIÓN]: Completar con hallazgos sobre cómo visualizan la estructura

[COMPLETAR: Descripción narrativa sobre métodos de visualización]

\textit{Citas textuales representativas:}

\begin{quote}
\textbf{E1:} [COMPLETAR: Cita sobre visualización de estructura]
\end{quote}

\begin{quote}
\textbf{E2:} [COMPLETAR: Cita sobre visualización de estructura]
\end{quote}

\begin{quote}
\textbf{E3:} [COMPLETAR: Cita sobre visualización de estructura]
\end{quote}

\textit{Patrones identificados:}

\begin{itemize}
	\item [COMPLETAR: Patrón común 1]
	\item [COMPLETAR: Patrón común 2]
	\item [COMPLETAR: Diferencia individual observada]
\end{itemize}

\textbf{Síntesis de la Dimensión 2:}

% [INSTRUCCIÓN]: Escribir párrafo integrador sobre estructuración dramática

[COMPLETAR: Párrafo integrador que responda: ¿Cómo estructuran dramáticamente sus obras antes del guion literario? Integrar paradigmas, niveles de organización y métodos de visualización.]

\newpage

% ══════════════════════════════════════════════════════════════════
% 4.3.3. DIMENSIÓN 3: MÉTODOS Y HERRAMIENTAS DE SISTEMATIZACIÓN
% ══════════════════════════════════════════════════════════════════

\subsubsection{Dimensión 3: Métodos y Herramientas de Sistematización}

Esta dimensión completó la respuesta al Objetivo Específico 2, caracterizando las herramientas y métodos concretos utilizados por los guionistas.

Los indicadores analizados fueron: herramientas digitales utilizadas, métodos análogos, documentos de pre-escritura y combinaciones de métodos.

% ──────────────────────────────────────────────────────────────────
% A. HERRAMIENTAS DIGITALES Y ANÁLOGAS
% ──────────────────────────────────────────────────────────────────

\textbf{A. Herramientas digitales y análogas utilizadas}

% [INSTRUCCIÓN]: Completar con hallazgos sobre herramientas específicas

[COMPLETAR: Descripción narrativa de las herramientas utilizadas]

\textit{Citas textuales representativas:}

\begin{quote}
\textbf{E1:} [COMPLETAR: Cita sobre herramientas utilizadas]
\end{quote}

\begin{quote}
\textbf{E2:} [COMPLETAR: Cita sobre herramientas utilizadas]
\end{quote}

\begin{quote}
\textbf{E3:} [COMPLETAR: Cita sobre herramientas utilizadas]
\end{quote}

% [INSTRUCCIÓN]: Completar tabla comparativa con datos reales
% Marcar con ✓ o ✗ según uso reportado en entrevistas

\begin{table}[ht]
	\centering
	\renewcommand{\arraystretch}{1.3}
	\begin{tabular}{|l|c|c|c|p{4cm}|}
		\hline
		\textbf{Herramienta} & \textbf{E1} & \textbf{E2} & \textbf{E3} & \textbf{Propósito} \\ \hline

		Final Draft & [✓/✗] & [✓/✗] & [✓/✗] & [COMPLETAR] \\ \hline

		Celtx & [✓/✗] & [✓/✗] & [✓/✗] & [COMPLETAR] \\ \hline

		Google Docs & [✓/✗] & [✓/✗] & [✓/✗] & [COMPLETAR] \\ \hline

		Notion/Trello & [✓/✗] & [✓/✗] & [✓/✗] & [COMPLETAR] \\ \hline

		Papel/Tarjetas & [✓/✗] & [✓/✗] & [✓/✗] & [COMPLETAR] \\ \hline

		Cuadernos & [✓/✗] & [✓/✗] & [✓/✗] & [COMPLETAR] \\ \hline

		[Agregar otras] & [✓/✗] & [✓/✗] & [✓/✗] & [COMPLETAR] \\ \hline
	\end{tabular}
	\caption{Herramientas utilizadas por los participantes}
	\label{tab:herramientas}
\end{table}

% ──────────────────────────────────────────────────────────────────
% B. DOCUMENTOS DE PRE-ESCRITURA
% ──────────────────────────────────────────────────────────────────

\textbf{B. Documentos de pre-escritura elaborados}

% [INSTRUCCIÓN]: Completar con hallazgos sobre qué documentos crean (logline, sinopsis, escaleta, tratamiento)

[COMPLETAR: Descripción narrativa sobre documentos de pre-escritura]

\textit{Citas textuales representativas:}

\begin{quote}
\textbf{E1:} [COMPLETAR: Cita sobre documentos de pre-escritura]
\end{quote}

\begin{quote}
\textbf{E2:} [COMPLETAR: Cita sobre documentos de pre-escritura]
\end{quote}

\begin{quote}
\textbf{E3:} [COMPLETAR: Cita sobre documentos de pre-escritura]
\end{quote}

% ──────────────────────────────────────────────────────────────────
% C. PREFERENCIAS METODOLÓGICAS
% ──────────────────────────────────────────────────────────────────

\textbf{C. Preferencias metodológicas: digital vs. análogo}

% [INSTRUCCIÓN]: Completar con hallazgos sobre preferencias y razones

[COMPLETAR: Descripción narrativa sobre preferencias metodológicas]

\textit{Citas textuales representativas:}

\begin{quote}
\textbf{E1:} [COMPLETAR: Cita sobre preferencias digital/análogo]
\end{quote}

\begin{quote}
\textbf{E2:} [COMPLETAR: Cita sobre preferencias digital/análogo]
\end{quote}

\begin{quote}
\textbf{E3:} [COMPLETAR: Cita sobre preferencias digital/análogo]
\end{quote}

\textit{Patrones identificados:}

\begin{itemize}
	\item [COMPLETAR: Patrón común sobre preferencias]
	\item [COMPLETAR: Razones mencionadas]
	\item [COMPLETAR: Diferencias individuales]
\end{itemize}

\textbf{Síntesis de la Dimensión 3:}

% [INSTRUCCIÓN]: Escribir párrafo integrador sobre herramientas y métodos

[COMPLETAR: Párrafo integrador que responda: ¿Qué herramientas y métodos utilizan? ¿Prefieren digital o análogo? ¿Qué documentos elaboran?]

\newpage

% ══════════════════════════════════════════════════════════════════
% 4.3.4. DIMENSIÓN 4: DESAFÍOS Y DIFICULTADES
% ══════════════════════════════════════════════════════════════════

\subsubsection{Dimensión 4: Desafíos y Dificultades en la Sistematización}

Esta dimensión respondió al Objetivo Específico 3: \textit{Describir las principales dificultades que enfrentaron los guionistas en la sistematización del proceso de pre-escritura.}

Los indicadores analizados fueron: dificultades técnicas, dificultades metodológicas, limitaciones de herramientas y necesidades no cubiertas.

% ──────────────────────────────────────────────────────────────────
% A. DIFICULTADES TÉCNICAS Y METODOLÓGICAS
% ──────────────────────────────────────────────────────────────────

\textbf{A. Dificultades técnicas y metodológicas identificadas}

% [INSTRUCCIÓN]: Completar con hallazgos sobre dificultades reportadas

[COMPLETAR: Descripción narrativa de las dificultades principales]

\textit{Citas textuales representativas:}

\begin{quote}
\textbf{E1:} [COMPLETAR: Cita sobre dificultades enfrentadas]
\end{quote}

\begin{quote}
\textbf{E2:} [COMPLETAR: Cita sobre dificultades enfrentadas]
\end{quote}

\begin{quote}
\textbf{E3:} [COMPLETAR: Cita sobre dificultades enfrentadas]
\end{quote}

% ──────────────────────────────────────────────────────────────────
% B. LIMITACIONES DE HERRAMIENTAS ACTUALES
% ──────────────────────────────────────────────────────────────────

\textbf{B. Limitaciones de herramientas actuales}

% [INSTRUCCIÓN]: Completar con hallazgos sobre limitaciones de software/herramientas

[COMPLETAR: Descripción narrativa sobre limitaciones identificadas]

\textit{Citas textuales representativas:}

\begin{quote}
\textbf{E1:} [COMPLETAR: Cita sobre limitaciones de herramientas]
\end{quote}

\begin{quote}
\textbf{E2:} [COMPLETAR: Cita sobre limitaciones de herramientas]
\end{quote}

\begin{quote}
\textbf{E3:} [COMPLETAR: Cita sobre limitaciones de herramientas]
\end{quote}

% ──────────────────────────────────────────────────────────────────
% C. NECESIDADES NO CUBIERTAS
% ──────────────────────────────────────────────────────────────────

\textbf{C. Necesidades no cubiertas por herramientas existentes}

% [INSTRUCCIÓN]: Completar con hallazgos sobre qué funcionalidades desearían tener
% IMPORTANTE: Estos hallazgos deben justificar las características de DREAMINK

[COMPLETAR: Descripción narrativa sobre necesidades expresadas]

\textit{Citas textuales representativas:}

\begin{quote}
\textbf{E1:} [COMPLETAR: Cita sobre necesidades no satisfechas]
\end{quote}

\begin{quote}
\textbf{E2:} [COMPLETAR: Cita sobre necesidades no satisfechas]
\end{quote}

\begin{quote}
\textbf{E3:} [COMPLETAR: Cita sobre necesidades no satisfechas]
\end{quote}

% [INSTRUCCIÓN]: Completar tabla de frecuencia con datos reales
% Contar cuántos participantes mencionaron cada dificultad

\begin{table}[ht]
	\centering
	\renewcommand{\arraystretch}{1.3}
	\begin{tabular}{|p{7cm}|c|c|c|c|}
		\hline
		\textbf{Dificultad / Necesidad} & \textbf{E1} & \textbf{E2} & \textbf{E3} & \textbf{Total} \\ \hline

		[COMPLETAR: ej. Falta de herramientas integradas] & [✓/✗] & [✓/✗] & [✓/✗] & [0-3] \\ \hline

		[COMPLETAR: ej. Dificultad para visualizar estructura] & [✓/✗] & [✓/✗] & [✓/✗] & [0-3] \\ \hline

		[COMPLETAR: ej. Dispersión de información] & [✓/✗] & [✓/✗] & [✓/✗] & [0-3] \\ \hline

		[COMPLETAR: ej. Software costoso/propietario] & [✓/✗] & [✓/✗] & [✓/✗] & [0-3] \\ \hline

		[COMPLETAR: ej. Falta de soporte para estructura pre-guion] & [✓/✗] & [✓/✗] & [✓/✗] & [0-3] \\ \hline

		[Agregar otras dificultades] & [✓/✗] & [✓/✗] & [✓/✗] & [0-3] \\ \hline
	\end{tabular}
	\caption{Frecuencia de dificultades mencionadas por los participantes}
	\label{tab:dificultades}
\end{table}

\textbf{Síntesis de la Dimensión 4:}

% [INSTRUCCIÓN]: Escribir párrafo integrador sobre dificultades
% CLAVE: Debe conectar con la justificación de DREAMINK

[COMPLETAR: Párrafo integrador que responda: ¿Cuáles son las principales dificultades? ¿Qué necesidades expresaron? ¿Qué limitaciones encontraron en herramientas actuales?]

\newpage

% ══════════════════════════════════════════════════════════════════
% 4.4. DISCUSIÓN DE RESULTADOS
% ══════════════════════════════════════════════════════════════════

\subsection{Discusión de Resultados}

\subsubsection{Hallazgos Principales por Objetivo Específico}

% ──────────────────────────────────────────────────────────────────
% DISCUSIÓN OE1
% ──────────────────────────────────────────────────────────────────

\textbf{Respecto al Objetivo Específico 1: Organización de elementos narrativos}

% [INSTRUCCIÓN]: Integrar hallazgos de Dimensión 1
% Responder: ¿Cómo organizan? ¿Qué patrones? ¿Qué diferencias?

[COMPLETAR: Párrafos de discusión integradora. Ejemplo: Los hallazgos revelaron que los tres guionistas entrevistados emplearon métodos diversos para la organización de elementos narrativos, identificándose tanto patrones comunes como particularidades individuales...

En relación a la gestión de personajes, se observó que...

Respecto a las locaciones, los participantes...

En cuanto a la captura de ideas...]

% ──────────────────────────────────────────────────────────────────
% DISCUSIÓN OE2
% ──────────────────────────────────────────────────────────────────

\textbf{Respecto al Objetivo Específico 2: Herramientas y métodos de estructuración}

% [INSTRUCCIÓN]: Integrar hallazgos de Dimensiones 2 y 3
% Responder: ¿Qué herramientas? ¿Cómo estructuran? ¿Digital vs análogo?

[COMPLETAR: Párrafos de discusión integradora sobre herramientas, métodos de estructuración dramática y preferencias metodológicas]

% ──────────────────────────────────────────────────────────────────
% DISCUSIÓN OE3
% ──────────────────────────────────────────────────────────────────

\textbf{Respecto al Objetivo Específico 3: Dificultades en la sistematización}

% [INSTRUCCIÓN]: Integrar hallazgos de Dimensión 4
% Responder: ¿Cuáles son las dificultades? ¿Hay patrones? ¿Qué necesidades?

[COMPLETAR: Párrafos de discusión integradora sobre dificultades, limitaciones y necesidades no cubiertas]

\newpage

\subsubsection{Contraste con el Marco Teórico}

% [INSTRUCCIÓN]: Relacionar hallazgos con autores citados en Capítulo II
% Referencias clave: Field (2005), McKee (1997), Koestler (1964), Lovato (2020)

% ──────────────────────────────────────────────────────────────────
% CONTRASTE CON FIELD
% ──────────────────────────────────────────────────────────────────

\textbf{Contraste con el paradigma de Field (2005)}

% [INSTRUCCIÓN]: Discutir si los guionistas aplican el paradigma de 3 actos
% Relacionar hallazgos de Dimensión 2 con teoría de Field

[COMPLETAR: Ejemplo: Los hallazgos del estudio mostraron coherencia con el paradigma estructural propuesto por Field (2005), quien estableció que la organización sistemática de elementos narrativos mediante el paradigma de tres actos fue fundamental antes de iniciar la escritura del guion. Los participantes E1 y E3 manifestaron utilizar explícitamente...]

% ──────────────────────────────────────────────────────────────────
% CONTRASTE CON MCKEE
% ──────────────────────────────────────────────────────────────────

\textbf{Contraste con McKee (1997) sobre organización narrativa}

% [INSTRUCCIÓN]: Discutir coherencia con principios de organización de McKee

[COMPLETAR: Relacionar hallazgos con énfasis de McKee en organización consciente de elementos narrativos]

% ──────────────────────────────────────────────────────────────────
% CONTRASTE CON KOESTLER
% ──────────────────────────────────────────────────────────────────

\textbf{Contraste con el modelo de proceso creativo de Koestler (1964)}

% [INSTRUCCIÓN]: Discutir si se identifican las 3 fases: lógica, intuitiva, crítica

[COMPLETAR: Analizar si los métodos reportados reflejan las fases del proceso creativo de Koestler]

% ──────────────────────────────────────────────────────────────────
% CONTRASTE CON LOVATO
% ──────────────────────────────────────────────────────────────────

\textbf{Contraste con la sistematización del proceso transmedia (Lovato, 2020)}

% [INSTRUCCIÓN]: Discutir coherencia con sistematización del proceso creativo

[COMPLETAR: Relacionar hallazgos con concepto de sistematización del proceso creativo de Lovato]

% ──────────────────────────────────────────────────────────────────
% HALLAZGOS EMERGENTES
% ──────────────────────────────────────────────────────────────────

\textbf{Hallazgos no anticipados por el marco teórico}

% [INSTRUCCIÓN]: Discutir aspectos que emergieron y no estaban en la teoría

[COMPLETAR: Identificar categorías emergentes no previstas en el marco teórico]

\newpage

\subsubsection{Implicaciones para la Práctica Profesional y Justificación de DREAMINK}

% [INSTRUCCIÓN]: SECCIÓN CRÍTICA - Conectar hallazgos con justificación de DREAMINK
% Debe explicar por qué DREAMINK es una respuesta a las necesidades identificadas

% ──────────────────────────────────────────────────────────────────
% NECESIDADES IDENTIFICADAS
% ──────────────────────────────────────────────────────────────────

\textbf{Necesidades identificadas en el estudio}

% [INSTRUCCIÓN]: Sintetizar necesidades de Dimensión 4 que justifican DREAMINK

[COMPLETAR: Ejemplo: El análisis de las entrevistas reveló necesidades concretas expresadas por los guionistas participantes, las cuales no fueron satisfechas por las herramientas existentes en el mercado...

Primera necesidad identificada: [COMPLETAR basado en Dimensión 4]

Segunda necesidad identificada: [COMPLETAR basado en Dimensión 4]

Tercera necesidad identificada: [COMPLETAR basado en Dimensión 4]]

% ──────────────────────────────────────────────────────────────────
% JUSTIFICACIÓN DE DREAMINK
% ──────────────────────────────────────────────────────────────────

\textbf{DREAMINK como respuesta a las necesidades identificadas}

% [INSTRUCCIÓN]: Explicar cómo DREAMINK responde a cada necesidad identificada
% Conectar con características del software descritas en Capítulo V

[COMPLETAR: Ejemplo: Los hallazgos del presente estudio proporcionaron evidencia empírica que justificó el desarrollo del sistema DREAMINK como herramienta de sistematización del proceso de pre-escritura de guiones.

En respuesta a la necesidad de [NECESIDAD 1 identificada], DREAMINK incorporó [CARACTERÍSTICA del Cap. V que la resuelve]...

En relación a [NECESIDAD 2 identificada], el sistema propuesto incluyó [CARACTERÍSTICA del Cap. V que la resuelve]...

Respecto a [NECESIDAD 3 identificada], DREAMINK ofreció [CARACTERÍSTICA del Cap. V que la resuelve]...]

% ──────────────────────────────────────────────────────────────────
% CONTRIBUCIÓN AL SECTOR
% ──────────────────────────────────────────────────────────────────

\textbf{Contribución al sector audiovisual paraguayo}

% [INSTRUCCIÓN]: Discutir impacto potencial para guionistas locales

[COMPLETAR: Explicar cómo DREAMINK puede beneficiar a guionistas paraguayos basándose en hallazgos del estudio]

% ──────────────────────────────────────────────────────────────────
% ALINEACIÓN CON PRÁCTICAS PROFESIONALES
% ──────────────────────────────────────────────────────────────────

\textbf{Alineación con prácticas profesionales identificadas}

% [INSTRUCCIÓN]: Explicar cómo DREAMINK se alinea con métodos que ya usan

[COMPLETAR: Explicar que DREAMINK no impone métodos nuevos, sino que sistematiza prácticas que los guionistas ya emplean de manera dispersa]

\newpage

% ══════════════════════════════════════════════════════════════════
% FIN DEL CAPÍTULO IV - TEMPLATE
% ══════════════════════════════════════════════════════════════════
% RECORDATORIO FINAL:
%
% Este template debe ser completado con datos reales de las 3 entrevistas.
% Siga el Protocolo de Aplicación (docs/instrumento_entrevista/ProtocoloAplicacion.md)
% Use la Guía de Entrevista (docs/instrumento_entrevista/GuiaEntrevista.md)
% Aplique análisis de contenido cualitativo a las transcripciones reales
%
% Extensión estimada del capítulo completo: 25-34 páginas
% ══════════════════════════════════════════════════════════════════
