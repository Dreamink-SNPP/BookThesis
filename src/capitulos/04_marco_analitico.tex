% ══════════════════════════════════════════════════════════════════
% CAPÍTULO IV - MARCO ANALÍTICO
% ══════════════════════════════════════════════════════════════════
% TEMPLATE: Este capítulo debe completarse con datos reales de entrevistas
% Siga las instrucciones en cada sección marcadas con [COMPLETAR]
% ══════════════════════════════════════════════════════════════════

\clearpage
\thispagestyle{empty}
\vspace*{\fill}
\begin{center}
    {\Large\bfseries CAPÍTULO IV - MARCO ANALÍTICO}
\end{center}
\vspace*{\fill}
\clearpage

\section*{CAPÍTULO IV - MARCO ANALÍTICO}
\addcontentsline{toc}{section}{CAPÍTULO IV - MARCO ANALÍTICO}
\setcounter{section}{4}

% ══════════════════════════════════════════════════════════════════
% 4.1. CARACTERIZACIÓN DE LOS PARTICIPANTES
% ══════════════════════════════════════════════════════════════════

\subsection{Caracterización de los Participantes}

% [INSTRUCCIÓN]: Completar con información real de los 3 guionistas entrevistados
% Use códigos E1, E2, E3 para mantener anonimato según consentimiento informado

El presente estudio se realizó con la participación de tres guionistas del Departamento Central de Paraguay, quienes aceptaron voluntariamente colaborar en la investigación mediante la firma del consentimiento informado. A continuación se presentó la caracterización de los participantes, utilizando códigos para preservar su anonimato conforme a los principios éticos de la investigación.

\begin{table}[ht]
	\centering
	\renewcommand{\arraystretch}{1.3}
	\begin{tabular}{|c|c|p{4cm}|p{3.5cm}|}
		\hline
		\textbf{Código} & \textbf{Experiencia} & \textbf{Tipo de proyectos} & \textbf{Formación} \\ \hline

		E1 & 10 años & Ficción, series, largometrajes, cortometrajes, publicidad & Autodidacta \\ \hline

		E2 & 8 años & Ficción, thriller psicológico, drama, cortometrajes, largometrajes, series & Autodidacta (McKee, Field, Snyder) \\ \hline

		E3 & 2 años & Ficción, thriller, ciencia ficción, cortometrajes, series web & Autodidacta (MasterClass, YouTube) \\ \hline
	\end{tabular}
	\caption{Caracterización de los participantes del estudio}
	\label{tab:participantes}
\end{table}

% [INSTRUCCIÓN]: Agregar descripción narrativa de cada participante
% Incluir: trayectoria, proyectos relevantes, contexto profesional actual
% Extensión sugerida: 1-2 párrafos por participante

\textbf{Participante E1:} Se trata de un guionista con 10 años de experiencia profesional en el sector audiovisual paraguayo. Inició su trayectoria de manera autodidacta desde la adolescencia, buscando software y tutoriales en Internet para desarrollar sus habilidades de escritura de guiones. Su trabajo se ha centrado principalmente en cortometrajes y largometrajes de ficción, habiendo producido aproximadamente 10 cortometrajes y 3 publicidades. Actualmente trabaja en dos proyectos simultáneos: un largometraje de aventura y acción en primera versión, y un cortometraje de comedia en fase de versión final lista para rodaje.

\textbf{Participante E2:} Guionista con 8 años de experiencia profesional, especializado en ficción con énfasis en thriller psicológico y drama. Su formación autodidacta se fundamentó en el estudio sistemático de guiones publicados, análisis cronometrado de películas y lectura de teóricos clásicos como McKee, Field y Snyder. Su trayectoria comenzó como editor de video, identificando problemas estructurales en guiones antes de llegar a rodaje, lo que motivó su transición hacia la escritura. Ha producido 6 cortometrajes y un mediometraje para televisión. Ha desarrollado tres largometrajes completos, uno de los cuales se encuentra actualmente en preproducción. Trabaja en un piloto de serie neo-noir ambientado en Asunción.

\textbf{Participante E3:} Guionista emergente con 2 años de experiencia profesional en escritura de guiones con formato. Su formación autodidacta se desarrolló durante la pandemia mediante suscripción a MasterClass, canales de YouTube de análisis cinematográfico y lectura de guiones descargados en PDF. Se especializa en thriller y ciencia ficción de bajo presupuesto. Ha escrito 6 cortometrajes completos, de los cuales 2 han sido producidos (uno grabado con celulares y otro con cámara profesional). Actualmente trabaja en un piloto de serie web sobre hackers en Asunción, encontrándose en la etapa de reescritura.

\newpage

% ══════════════════════════════════════════════════════════════════
% 4.2. PROCESAMIENTO DE LOS DATOS
% ══════════════════════════════════════════════════════════════════

\subsection{Procesamiento de los Datos}

\subsubsection{Procedimiento de Recolección de Datos}

Las entrevistas se realizaron durante el período comprendido entre noviembre y diciembre de 2025. La entrevista E1 tuvo una duración de 30 minutos aproximadamente y se realizó de forma presencial en San Lorenzo, Departamento Central. La entrevista E2 se llevó a cabo de manera virtual con una duración de 58 minutos. La entrevista E3 se realizó mediante videoconferencia (Jitsi Meet) con una duración de 48 minutos, conectando desde Itauguá, Paraguay.

Previo a cada entrevista, se procedió a la lectura y firma del consentimiento informado, garantizando la comprensión plena de los objetivos del estudio, la voluntariedad de la participación, el anonimato de las respuestas y el uso exclusivamente académico de la información recopilada. La entrevista E1 fue grabada en audio con la autorización expresa del participante. Las entrevistas E2 y E3 no permitieron grabación de audio, procediéndose a tomar notas extensivas durante la realización de las mismas.

El instrumento utilizado fue una guía de entrevista semiestructurada, diseñada a partir de la matriz de operacionalización de variables, la cual permitió explorar las cuatro dimensiones establecidas: organización de elementos narrativos, estructuración dramática, métodos y herramientas de sistematización, y desafíos y dificultades en el proceso de pre-escritura.

\subsubsection{Proceso de Transcripción y Codificación}

Las entrevistas fueron transcritas de manera literal (verbatim) dentro de las 72 horas posteriores a cada encuentro, siguiendo el protocolo de aplicación del instrumento. La transcripción de E1 se basó en la grabación de audio autorizada, mientras que las transcripciones de E2 y E3 se realizaron a partir de las notas extensivas tomadas durante las entrevistas. Las transcripciones incluyeron las respuestas textuales de los participantes, pausas significativas, énfasis relevantes y observaciones del entrevistador sobre lenguaje corporal y tono de voz para el análisis posterior.

Cada participante fue codificado con un identificador alfanumérico (E1, E2, E3) para preservar su anonimato, eliminando toda información identificable de las transcripciones. Las transcripciones fueron almacenadas en formato Markdown y respaldadas en repositorio Git con control de versiones.

\subsubsection{Técnica de Análisis de Contenido Cualitativo}

El análisis de los datos se realizó mediante la técnica de análisis de contenido cualitativo, siguiendo un proceso sistemático de tres fases:

\textbf{Primera fase - Lectura exhaustiva:} Se realizó una lectura completa y repetida de las tres transcripciones para familiarizarse con el contenido global de las entrevistas y obtener una comprensión holística de las experiencias relatadas por los participantes.

\textbf{Segunda fase - Codificación:} Se aplicó un proceso de codificación en dos etapas. La codificación abierta permitió identificar categorías emergentes directamente desde los datos, mientras que la codificación axial estableció relaciones entre las categorías identificadas. Este proceso se organizó según las cuatro dimensiones de la matriz de operacionalización de variables.

\textbf{Tercera fase - Identificación de patrones:} Se identificaron patrones comunes y divergentes entre los tres participantes, triangulando los hallazgos con el marco teórico establecido en el Capítulo II. Esta fase permitió caracterizar las prácticas de sistematización empleadas por los guionistas y detectar convergencias y particularidades en sus métodos de trabajo.

\newpage

% ══════════════════════════════════════════════════════════════════
% 4.3. PRESENTACIÓN DE RESULTADOS POR DIMENSIONES
% ══════════════════════════════════════════════════════════════════

\subsection{Presentación de Resultados por Dimensiones}

% ══════════════════════════════════════════════════════════════════
% 4.3.1. DIMENSIÓN 1: ORGANIZACIÓN DE ELEMENTOS NARRATIVOS
% ══════════════════════════════════════════════════════════════════

\subsubsection{Dimensión 1: Organización de Elementos Narrativos}

Esta dimensión respondió al Objetivo Específico 1: \textit{Identificar las formas de organización de elementos narrativos (personajes, locaciones e ideas) empleadas por los guionistas durante la fase de pre-escritura.}

Los indicadores analizados fueron: gestión de personajes, gestión de locaciones, gestión de ideas y conceptos, y métodos de registro y documentación.

% ──────────────────────────────────────────────────────────────────
% A. GESTIÓN DE PERSONAJES
% ──────────────────────────────────────────────────────────────────

\textbf{A. Gestión de personajes}

Los tres guionistas entrevistados manifestaron utilizar métodos visuales para la gestión de personajes durante la fase de pre-escritura, rechazando las fichas técnicas tradicionales. Se identificó un patrón común de priorización de aspectos psicológicos sobre características físicas superficiales. E1 desarrolló su método mediante escritura manual de datos sueltos del personaje en cuadernos, que posteriormente transforma en mini-biografías y perfiles digitalizados en archivos de texto. E2 manifestó trabajar con referencias visuales de actores o personas reales, complementando con un método particular de entrevistas ficticias con los personajes, enfocándose en "la herida del pasado" como elemento central. E3 desarrolló un sistema basado en collages visuales con fotografías de Pinterest, reduciendo la información registrada al binomio fundamental de Deseo y Miedo. Los tres participantes mostraron rechazo explícito hacia plantillas estándar que solicitan información como color de ojos, altura o comida favorita, privilegiando en su lugar la construcción de perfiles psicológicos orientados a la funcionalidad narrativa.

\textit{Citas textuales representativas:}

\begin{quote}
\textbf{E1:} ``generalmente, eh, hago a mano suelo escribir en un cuaderno a vece en hoja en blanco. Eh... Como... Eh... Son como datos sueltos del personaje que é como traduzco en, en una, o sea, transformo en una mini biografía. Y después hago perfiles, en... en ya en la computadora en un archivo de texto.''
\end{quote}

\begin{quote}
\textbf{E2:} ``Hago perfiles psicológicos basados en sus heridas. 'La herida del pasado'. Eso sí lo escribo y lo subrayo en rojo. Es mi \textit{brújula}.''
\end{quote}

\begin{quote}
\textbf{E3:} ``Mmm, soy muy visual. No me gusta llenar esas fichas técnicas que te piden la 'comida favorita' del personaje si no sirve para la trama. Lo que hago es... busco fotos. Actores, gente random en Pinterest. Armo como un collage. Y al lado de la foto pongo: \textbf{Deseo} y \textbf{Miedo}. Solo eso.''
\end{quote}

\textit{Patrones identificados:}

\begin{itemize}
	\item Uso de métodos visuales para representación de personajes (fotografías de actores o personas reales) - los 3 participantes
	\item Rechazo a fichas técnicas tradicionales estandarizadas - los 3 participantes
	\item Priorización de aspectos psicológicos y motivacionales sobre características físicas - los 3 participantes
	\item Diferencia en nivel de formalización: E1 desarrolla mini-biografías estructuradas, E2 utiliza entrevistas ficticias, E3 emplea binomio Deseo/Miedo
\end{itemize}

% ──────────────────────────────────────────────────────────────────
% B. GESTIÓN DE LOCACIONES
% ──────────────────────────────────────────────────────────────────

\textbf{B. Gestión de locaciones}

Los tres guionistas entrevistados manifestaron utilizar referencias visuales para la gestión de locaciones, considerando factores productivos en su selección. E1 señaló trabajar mayormente con bajo presupuesto, limitando intencionalmente la cantidad de locaciones y registrando descripciones breves del tipo de locación requerida. E2 manifestó concebir la locación como un personaje adicional, visitando físicamente los lugares cuando existen o utilizando Pinterest para referencias visuales, desarrollando descripciones sensoriales detalladas (olores, sonidos) previo a la escritura de escenas. E3 desarrolló un método pragmático basado en Google Maps y Street View, capturando pantallas de locaciones reales del entorno cercano y pegándolas en documentos de Word. Se identificó como patrón común la consideración de factores de producción (viabilidad, presupuesto) en la selección de locaciones, diferenciándose E2 por el nivel de detalle en las descripciones sensoriales.

\textit{Citas textuales representativas:}

\begin{quote}
\textbf{E1:} ``Como yo trabajo mucho [ríe] en el bajo presupuesto [...] intento de que tenga eh, la menor cantidad [ríe] de locaciones posibles [...] escribo el tipo de locación que quiero y una breve descripción.''
\end{quote}

\begin{quote}
\textbf{E2:} ``La locación es un personaje más, ¿verdad? Trato de visitarlas si existen. Si no, Pinterest. Tengo tableros ocultos llenos de pasillos oscuros, habitaciones desordenadas... Lo que hago es describir sensorialmente el lugar en un cuaderno antes de escribir la escena. A qué huele, qué se escucha.''
\end{quote}

\begin{quote}
\textbf{E3:} ``Uso Google Maps, te juro. Hago capturas de pantalla de Street View y las pego en un documento de Word. Pongo 'Casa del Protagonista' y la foto de una casa que vi en mi barrio.''
\end{quote}

\textit{Patrones identificados:}

\begin{itemize}
	\item Uso de referencias visuales (Pinterest, Google Maps, visitas físicas) - los 3 participantes
	\item Consideración de factores productivos (viabilidad, presupuesto) en la selección - los 3 participantes
	\item Diferencia en nivel de detalle: E2 desarrolla descripciones sensoriales elaboradas; E1 y E3 mantienen descripciones básicas
\end{itemize}

% ──────────────────────────────────────────────────────────────────
% C. GESTIÓN DE IDEAS Y CONCEPTOS
% ──────────────────────────────────────────────────────────────────

\textbf{C. Gestión de ideas y conceptos}

Los tres guionistas entrevistados manifestaron utilizar intensivamente el celular para la captura espontánea de ideas mediante aplicaciones de notas y grabaciones de voz. E1 desarrolló un sistema organizado mediante la aplicación Standard Notes, creando carpetas y subcarpetas categorizadas (ideas de personaje, locación, eventos históricos), aunque reconoció la dificultad de transcribir consistentemente desde cuadernos y celular. E2 manifestó acumular miles de notas en el celular, incluyendo grabaciones de audio realizadas mientras conduce, identificando como problema principal que las ideas quedan ``enterradas meses'' sin organizar. E3 se identificó como ``adicto a las notas de voz'', capturando ideas espontáneas en la ducha o conduciendo, reconociendo la dificultad de transferir esos audios a la computadora. Se identificó como patrón universal el uso del celular para captura inmediata, con una dificultad recurrente en la organización y transcripción posterior de esas ideas. E1 se diferenció por haber desarrollado un sistema de organización digital estructurado, mientras E2 y E3 acumulan ideas sin sistematizar.

\textit{Citas textuales representativas:}

\begin{quote}
\textbf{E1:} ``Anoto en donde sea que pueda anotar, que podría ser mi celular, a veces los cuadernos que llevo conmigo. Y... después... esta es la parte más difícil que es justamente organizar [...] Yo uso específicamente un, un, un software de notas [...] Se llama Standard Notes.''
\end{quote}

\begin{quote}
\textbf{E2:} ``[Saca su celular del bolsillo y lo muestra] Aquí. Notas del celular. Tengo miles. A veces voy manejando y se me ocurre una línea de diálogo, entonces grabo un audio [...] El problema es que luego... ehh, esas ideas se quedan ahí enterradas meses.''
\end{quote}

\begin{quote}
\textbf{E3:} ``Notas de voz. Soy adicto a las notas de voz. Voy manejando o estoy en la ducha y ¡bum!, se me ocurre un diálogo. Grabo rápido [...] el problema es pasar eso a la compu. Tengo el celular lleno de audios que a veces olvido escuchar.''
\end{quote}

\textit{Patrones identificados:}

\begin{itemize}
	\item Uso intensivo del celular (notas/audio) para captura espontánea de ideas - los 3 participantes
	\item Dificultad recurrente para organizar y transcribir ideas capturadas - los 3 participantes
	\item Diferencia: E1 desarrolló sistema organizado (Standard Notes con carpetas categorizadas); E2 y E3 acumulan sin sistematizar
\end{itemize}

% ──────────────────────────────────────────────────────────────────
% D. MÉTODOS DE REGISTRO Y DOCUMENTACIÓN
% ──────────────────────────────────────────────────────────────────

\textbf{D. Métodos de registro y documentación}

Los tres guionistas entrevistados manifestaron rechazo hacia plantillas estándar o rígidas, desarrollando sistemas propios de organización personalizados. E1 desarrolló un sistema de carpetas y subcarpetas en Standard Notes con categorías inventadas por él mismo (ideas de personaje, locación, eventos históricos). E2 manifestó sentirse ``encorsetado'' con plantillas descargadas de internet tipo ``La Biblia de la Serie'', optando por abrir documentos en blanco de Google Docs y utilizar códigos de colores propios (verde para trama A, azul para trama B). E3 rechazó plantillas de Excel por considerar que ``matan la creatividad'' al sentirse como ``llenar un formulario de impuestos'', desarrollando una estructura propia flexible en Notion que modifica según cada proyecto. Se identificó como patrón universal el rechazo a formatos estandarizados externos y el desarrollo de sistemas personalizados que priorizan la flexibilidad como valor central. Los tres participantes manifestaron explícitamente que las plantillas rígidas inhiben su proceso creativo.

\textit{Citas textuales representativas:}

\begin{quote}
\textbf{E1:} ``tengo una carpeta que se llama ideas. Porque [muy baja pronunciación de la palabra porque] la aplicación me deja crear carpetas y subcarpetas con notas. Tengo una carpeta que llama ideas y tengo subcarpetas que son, por ejemplo, ideas de personaje o locación [...] son categorías que yo inventé, prácticamente.''
\end{quote}

\begin{quote}
\textbf{E2:} ``No realmente. O sea, he intentado usar plantillas que bajas de internet, tipo 'La Biblia de la Serie', pero me siento... encorsetado. Termino abriendo un Google Doc en blanco y usando mis propios códigos de colores. Verde para trama A, azul para trama B.''
\end{quote}

\begin{quote}
\textbf{E3:} ``He intentado usar plantillas de Excel que bajé de internet, pero me matan la creatividad. Me siento como llenando un formulario de la SAT (impuestos) [...] Ahora estoy usando Notion. Me armé una página media básica [...] es una estructura propia, muy flexible. Cambio el formato con cada proyecto.''
\end{quote}

\textit{Patrones identificados:}

\begin{itemize}
	\item Rechazo universal a plantillas estándar o rígidas - los 3 participantes
	\item Desarrollo de sistemas propios personalizados de organización - los 3 participantes
	\item Flexibilidad como valor central en la sistematización - los 3 participantes
	\item Manifestación explícita de que plantillas rígidas inhiben el proceso creativo - los 3 participantes
\end{itemize}

% ──────────────────────────────────────────────────────────────────
% SÍNTESIS DE LA DIMENSIÓN 1
% ──────────────────────────────────────────────────────────────────

\textbf{Síntesis de la Dimensión 1:}

Los guionistas entrevistados emplearon métodos diversos pero con patrones consistentes en la organización de elementos narrativos durante la fase de pre-escritura. Se identificó una convergencia universal en tres aspectos fundamentales: la priorización de aspectos psicológicos sobre características físicas en personajes, la utilización de referencias visuales para la representación de locaciones, y el uso intensivo del celular para la captura espontánea de ideas. El rechazo a plantillas estándar rígidas emergió como patrón transversal, manifestándose los tres participantes desarrollar sistemas propios personalizados que priorizan la flexibilidad como valor central. La fragmentación de información entre múltiples herramientas y soportes (cuadernos, celular, aplicaciones digitales) se identificó como desafío recurrente, particularmente en la dificultad para organizar y transcribir ideas capturadas espontáneamente. Los métodos empleados reflejan una sistematización adaptativa donde cada guionista construyó soluciones personalizadas que responden a su proceso creativo particular, evidenciándose diferencias en el nivel de formalización (E1 más estructurado mediante carpetas categorizadas, E2 y E3 más flexibles) sin que esto implique ausencia de sistematización.

\newpage

% ══════════════════════════════════════════════════════════════════
% 4.3.2. DIMENSIÓN 2: ESTRUCTURACIÓN DRAMÁTICA
% ══════════════════════════════════════════════════════════════════

\subsubsection{Dimensión 2: Estructuración Dramática}

Esta dimensión respondió parcialmente al Objetivo Específico 2: \textit{Caracterizar las herramientas y métodos utilizados por los guionistas para estructurar dramáticamente sus obras antes del guion literario.}

Los indicadores analizados fueron: organización en actos, organización en secuencias, organización en escenas y paradigmas estructurales utilizados.

% ──────────────────────────────────────────────────────────────────
% A. PARADIGMAS ESTRUCTURALES UTILIZADOS
% ──────────────────────────────────────────────────────────────────

\textbf{A. Paradigmas estructurales utilizados}

Los tres guionistas entrevistados manifestaron utilizar predominantemente el paradigma de tres actos como base estructural, aunque con conocimiento y aplicación de múltiples modelos según el tipo de proyecto. E1 señaló emplear generalmente el método de tres actos, aunque también utiliza el método secuencial y ocasionalmente cinco actos dependiendo del proyecto. E2 manifestó conocer de memoria el modelo ``Save the Cat'' de Blake Snyder y el Viaje del Héroe, aunque su preferencia se centra en la estructura clásica de tres actos pensada desde los puntos de giro clave (Midpoint y Clímax), utilizando ocasionalmente el modelo de 8 secuencias para historias extensas. E3 demostró familiaridad con la estructura de 15 beats de Blake Snyder (``Save the Cat''), aunque para sus proyectos de thriller utiliza principalmente la estructura clásica de tres actos con puntos de giro bien marcados. Se identificó como patrón universal el uso del paradigma de tres actos como fundamento estructural, con conocimiento teórico de modelos alternativos (Save the Cat, Viaje del Héroe, método secuencial) que se aplican adaptativamente según las características de cada proyecto.

\textit{Citas textuales representativas:}

\begin{quote}
\textbf{E1:} ``Eh... Varía, como eh... Generalmente, estoy entre los 3 actos y los 5, pero, como no pienso dirigirme los auto, yo uso más el método secuencial, entonces pienso, según la historia, eh... En qué estructura le le conviene, pero sí, sería entre los 3 actos generalmente.''
\end{quote}

\begin{quote}
\textbf{E2:} ``Conozco el 'Save the Cat' de Blake Snyder de memoria, y el Viaje del Héroe. Generalmente uso la estructura de tres actos clásica, pero me gusta pensar más en 'Midpoint' y 'Clímax'. Esos son mis pilares. Sé dónde empieza, sé qué pasa a la mitad para joderle la vida al protagonista, y sé cómo termina. El resto es rellenar el mapa. A veces uso el de 8 secuencias si la historia es muy larga.''
\end{quote}

\begin{quote}
\textbf{E3:} ``¡Save the Cat! [levanta los pulgares]. Blake Snyder es mi pastor. Nah, mentira, pero sí uso mucho su estructura de los 15 beats [...] para mis thrillers, la estructura de tres actos clásica con los puntos de giro bien marcados es lo que mejor me funciona.''
\end{quote}

\textit{Patrones identificados:}

\begin{itemize}
	\item Uso predominante del paradigma de tres actos como base estructural - los 3 participantes
	\item Conocimiento teórico de múltiples paradigmas (Save the Cat, Viaje del Héroe, método secuencial) - los 3 participantes
	\item Adaptación del paradigma según características del proyecto - los 3 participantes
	\item Énfasis en puntos de giro clave (Midpoint, Clímax) como pilares estructurales - E2 y E3 explícitamente
\end{itemize}

% ──────────────────────────────────────────────────────────────────
% B. ORGANIZACIÓN EN ACTOS, SECUENCIAS Y ESCENAS
% ──────────────────────────────────────────────────────────────────

\textbf{B. Organización en actos, secuencias y escenas}

Los tres guionistas entrevistados manifestaron seguir un proceso bifásico consistente durante la estructuración dramática, caracterizado por una fase inicial de exploración creativa desestructurada seguida de una fase de organización estructural rigurosa. E1 describió este proceso mediante el concepto de ``proceso vómito'', donde escribe instintivamente sin pensar, para posteriormente estructurar esa historia en actos y secuencias, descubriendo las escenas específicas durante la escritura del guion. E2 manifestó que la escaleta es ``sagrada'', no escribiendo ninguna línea de diálogo en formato de guion hasta que la escaleta no funciona estructuralmente. E3 relató que anteriormente se trababa en la página 10 al lanzarse directamente a escribir, por lo que ahora es ``religioso'' con la escaleta, no escribiendo diálogo hasta tener la historia completa en puntos. Se identificó como patrón universal la centralidad de la escaleta como documento estructural fundamental, con un proceso consistente donde la estructura se define completamente antes de iniciar la escritura del guion literario. Los tres participantes enfatizaron que la definición estructural precede la redacción de diálogos.

\textit{Citas textuales representativas:}

\begin{quote}
\textbf{E1:} ``Intento que la historia surja primero, pero más como un cuento, es decir, de una forma más breve. Eh, Escribo la, eh, escribo muchas veces la historia sacando directa de mí. Eh, este proceso le llamamos justamente el proceso vomito, porque vos escribí instantes sin pensar, y una vez que terminá esa primera versión, [...] re en contra en sucio, ahí yo veo cómo estructurar esa historia.''
\end{quote}

\begin{quote}
\textbf{E2:} ``Escaleta. La escaleta es sagrada. No escribo ni una línea de diálogo en formato de guion hasta que la escaleta no funciona. Hago una lista numerada: Escena 1, Escena 2... y describo qué pasa en una frase. Si la estructura no se sostiene ahí, no se va a sostener después.''
\end{quote}

\begin{quote}
\textbf{E3:} ``Antes me lanzaba a escribir y me trababa en la página 10. Ahora soy religioso con la escaleta. No escribo ni una línea de diálogo en el programa de guion hasta que no tengo la historia completa en puntos. Hago una lista: Escena 1, pasa esto. Escena 2, pasa esto.''
\end{quote}

% ──────────────────────────────────────────────────────────────────
% C. VISUALIZACIÓN DE LA ESTRUCTURA
% ──────────────────────────────────────────────────────────────────

\textbf{C. Visualización de la estructura dramática}

Los tres guionistas entrevistados manifestaron utilizar post-its o tarjetas físicas en pared como método principal de visualización de la estructura dramática, representando una convergencia total en esta práctica. E1 señaló utilizar una pizarra de corcho donde clava tarjetas que representan escenas sueltas, marcando también la secuencia y estructura, aunque reconoció la limitación de que esa pizarra no puede transportarse. E2 manifestó utilizar post-its de colores en la pared de su estudio, enfatizando la importancia de la manipulación física: ``puedo mover una escena del acto 1 al acto 2 físicamente. Esa sensación táctil de mover la historia... ninguna pantalla me la da igual''. E3 describió llenar literalmente la pared de su cuarto con papelitos de colores (amarillo para Acto 1, rosa para Acto 2, verde para Acto 3), manifestando haber probado Trello digitalmente pero necesitar ``ver todo el panorama de un golpe de vista sin hacer scroll''. Se identificó como patrón universal la utilización de códigos de color, la necesidad de manipulación táctil física, y el rechazo a soluciones digitales fundamentado en la imposibilidad de obtener visión panorámica completa sin desplazamiento (scroll).

\textit{Citas textuales representativas:}

\begin{quote}
\textbf{E1:} ``Uso todo [ríe] básicamente, porque suelo escribir a mano, después paso, ehm..., en la computadora. Eh, También, por ejemplo, tengo un una pizarra corcho, donde voy donde tengo, las tarjetas que voy clavando, que serían la, básicamente, la escena suelta y marco también la la secuencia estructura.''
\end{quote}

\begin{quote}
\textbf{E2:} ``Post-its. En la pared de mi estudio. [Sonríe ampliamente]. Es el cliché del guionista, lo sé, pero funciona. Post-its de colores. Puedo mover una escena del acto 1 al acto 2 físicamente. Esa sensación táctil de mover la historia... ninguna pantalla me la da igual. Tomo fotos de la pared por si se caen, claro.''
\end{quote}

\begin{quote}
\textbf{E3:} ``Post-its. Literalmente lleno la pared de mi cuarto con papelitos de colores. Amarillo para Acto 1, Rosa para Acto 2, Verde para Acto 3 [...] En digital he probado Trello, pero... no sé, necesito ver todo el panorama de un golpe de vista sin hacer scroll.''
\end{quote}

\textit{Patrones identificados:}

\begin{itemize}
	\item Convergencia total en el uso de post-its/tarjetas físicas en pared - los 3 participantes
	\item Utilización de códigos de color para diferenciar actos o tramas - los 3 participantes
	\item Necesidad de manipulación táctil física de la estructura - los 3 participantes
	\item Rechazo a soluciones digitales por falta de visión panorámica sin scroll - los 3 participantes
\end{itemize}

\textbf{Síntesis de la Dimensión 2:}

Los guionistas entrevistados estructuran dramáticamente sus obras mediante un proceso sistemático fundamentado en el paradigma de tres actos, con conocimiento y aplicación adaptativa de modelos alternativos. Se identificó un proceso bifásico universal caracterizado por una fase inicial de exploración creativa desestructurada (``proceso vómito'') seguida de una fase de organización estructural rigurosa centrada en la escaleta como documento fundamental. La convergencia más significativa se manifestó en la visualización física de la estructura mediante post-its o tarjetas en pared, práctica universal entre los tres participantes fundamentada en la necesidad de manipulación táctil y visión panorámica completa. El rechazo a soluciones digitales para visualización estructural emergió como patrón consistente, justificado por la imposibilidad de obtener visión panorámica sin desplazamiento (scroll) y la ausencia de manipulación táctil que consideran esencial para el proceso creativo. Los hallazgos evidencian que la estructuración dramática no constituye una fase aislada sino un proceso iterativo donde la definición estructural completa precede obligatoriamente la escritura del guion literario.

\newpage

% ══════════════════════════════════════════════════════════════════
% 4.3.3. DIMENSIÓN 3: MÉTODOS Y HERRAMIENTAS DE SISTEMATIZACIÓN
% ══════════════════════════════════════════════════════════════════

\subsubsection{Dimensión 3: Métodos y Herramientas de Sistematización}

Esta dimensión completó la respuesta al Objetivo Específico 2, caracterizando las herramientas y métodos concretos utilizados por los guionistas.

Los indicadores analizados fueron: herramientas digitales utilizadas, métodos análogos, documentos de pre-escritura y combinaciones de métodos.

% ──────────────────────────────────────────────────────────────────
% A. HERRAMIENTAS DIGITALES Y ANÁLOGAS
% ──────────────────────────────────────────────────────────────────

\textbf{A. Herramientas digitales y análogas utilizadas}

Los tres guionistas entrevistados manifestaron utilizar una combinación de herramientas análogas y digitales durante la fase de pre-escritura. E1 desarrolló su flujo de trabajo mediante cuadernos y hojas en blanco para la fase exploratoria, utilizando Standard Notes y Notepad++ para organización digital, y pizarra de corcho con tarjetas para visualización estructural. E2 manifestó utilizar cuadernos y post-its físicos para creatividad, Google Docs para colaboración con códigos de colores, Notion para organización de la biblia del proyecto, Pinterest para referencias visuales, y Final Draft para escritura del guion. E3 señaló utilizar libreta y bolígrafo para ideas iniciales, post-its de colores para estructura en pared, celular para notas de voz, Notion para la biblia del proyecto, Pinterest para referencias, Google Maps para locaciones, y WriterDuet o Kit Scenarist (versiones gratuitas/open source) para escritura del guion. Se identificó como patrón universal el uso de cuadernos físicos, post-its para visualización, celular para captura espontánea, y software específico para escritura del guion. E2 y E3 coincidieron en el rechazo a Scrivener por complejidad excesiva.

\textit{Citas textuales representativas:}

\begin{quote}
\textbf{E1:} ``Sí, y, justamente, eh, como había dicho, tengo, suelo empezar en cuadernos. Cuando son ideas sueltas, uso un cuaderno, en general, de varias ideas [...] suelo usar, como dije, esta aplicación que se llama Standard Notes, o el el notepad++.''
\end{quote}

\begin{quote}
\textbf{E2:} ``Análogo: Pared, post-its, libreta y bolígrafo. Eh... Digital: Google Docs para compartir con colaboradores si los hay, y Final Draft cuando ya voy a escribir el guion. Pero para la pre-escritura, para el 'barro'... uso mucho Notion últimamente, tratando de organizarme.''
\end{quote}

\begin{quote}
\textbf{E3:} ``Es una mezcla rara. Uso libreta y birome para las ideas locas, Post-its para la estructura, y después paso a la compu. En la compu uso Notion para organizar la 'biblia' y para escribir el guion uso WriterDuet, la versión gratuita, o Kit Scenarist que es open source. No tengo plata para Final Draft todavía [ríe].''
\end{quote}

% [INSTRUCCIÓN]: Completar tabla comparativa con datos reales
% Marcar con ✓ o ✗ según uso reportado en entrevistas

\begin{table}[ht]
	\centering
	\renewcommand{\arraystretch}{1.3}
	\begin{tabular}{|l|c|c|c|p{4.5cm}|}
		\hline
		\textbf{Herramienta} & \textbf{E1} & \textbf{E2} & \textbf{E3} & \textbf{Propósito} \\ \hline

		Cuaderno/Libreta & ✓ & ✓ & ✓ & Captura inicial, exploración creativa \\ \hline

		Post-its/Tarjetas & ✓ & ✓ & ✓ & Visualización de estructura en pared \\ \hline

		Celular (notas/audio) & ✓ & ✓ & ✓ & Captura espontánea de ideas \\ \hline

		Standard Notes & ✓ & ✗ & ✗ & Organización con carpetas categorizadas \\ \hline

		Notepad++ & ✓ & ✗ & ✗ & Edición de texto plano \\ \hline

		Google Docs & ✗ & ✓ & ✗ & Colaboración, códigos de colores \\ \hline

		Notion & ✗ & ✓ & ✓ & Organización de biblia del proyecto \\ \hline

		Final Draft & ✗ & ✓ & ✗ & Escritura de guion (software propietario) \\ \hline

		WriterDuet & ✗ & ✗ & ✓ & Escritura de guion (versión gratuita) \\ \hline

		Kit Scenarist & ✗ & ✗ & ✓ & Escritura de guion (open source) \\ \hline

		Pinterest & ✗ & ✓ & ✓ & Referencias visuales \\ \hline

		Google Maps & ✗ & ✗ & ✓ & Locaciones reales (Street View) \\ \hline

		Scrivener & ✗ & ✗ (rechazado) & ✗ (rechazado) & Probado y rechazado por complejidad \\ \hline
	\end{tabular}
	\caption{Herramientas utilizadas por los participantes durante la pre-escritura}
	\label{tab:herramientas}
\end{table}

% ──────────────────────────────────────────────────────────────────
% B. DOCUMENTOS DE PRE-ESCRITURA
% ──────────────────────────────────────────────────────────────────

\textbf{B. Documentos de pre-escritura elaborados}

Los tres guionistas entrevistados manifestaron elaborar una secuencia metodológica consistente de documentos durante la pre-escritura: Logline, Sinopsis y Escaleta, en ese orden específico. E1 señaló que independientemente del punto de partida (personaje o idea), cuando se organiza busca siempre iniciar con el logline para tener bien definido de qué trata la historia, siguiendo con sinopsis corta y larga. E2 manifestó trabajar ``religiosamente'' en ese orden, enfatizando que si no puede contar la historia en dos frases (logline), no sabe qué está escribiendo, luego desarrolla sinopsis de una página y posteriormente la escaleta, elaborando el tratamiento en prosa solo cuando le es requerido externamente por fondos o productores ``porque me da pereza escribir en prosa''. E3 coincidió en que el logline es lo primero e imprescindible, seguido de sinopsis de una página y salto directo a escaleta de escenas, elaborando tratamiento literario extenso solo por exigencia externa. Se identificó como patrón universal la secuencia Logline → Sinopsis → Escaleta, siendo el logline considerado documento fundacional por los tres participantes. El tratamiento en prosa emergió como documento elaborado solo por requerimiento externo, no por iniciativa propia del proceso creativo.

\textit{Citas textuales representativas:}

\begin{quote}
\textbf{E1:} ``donde sea que empiece, siempre, cuando me organizo, quiero que el inicio sea un logline, tener bien definido de qué va la historia que estoy haciendo. Así que sí, sigo ese orden de logline, storyline, sinopsis larga y corta. Intento, intento seguir ese orden.''
\end{quote}

\begin{quote}
\textbf{E2:} ``Sí, en ese orden, religiosamente. Logline primero. Si no puedo contártelo en dos frases, no sé qué estoy escribiendo. Luego una sinopsis de una página. Luego la escaleta. El tratamiento... ehh, el tratamiento lo hago solo si me lo piden para un fondo o un productor, porque me da pereza escribir en prosa [se ríe].''
\end{quote}

\begin{quote}
\textbf{E3:} ``Sí. El Logline es lo primero, si no puedo resumirlo en dos líneas, no sé qué estoy escribiendo. Después hago una sinopsis de una página. Y de ahí salto directo a la escaleta de escenas. El tratamiento... mmm, casi nunca hago tratamiento literario extenso a menos que alguien me lo pida.''
\end{quote}

% ──────────────────────────────────────────────────────────────────
% C. PREFERENCIAS METODOLÓGICAS
% ──────────────────────────────────────────────────────────────────

\textbf{C. Preferencias metodológicas: digital vs. análogo}

Los tres guionistas entrevistados manifestaron una convergencia total en la preferencia por un enfoque híbrido que combina métodos análogos y digitales, con una fundamentación cognitiva explícita. E1 señaló que todo proyecto inicia en formato análogo (papel y bolígrafo) porque ese momento análogo es de exploración, pasando al método digital una vez que la idea está sólida para agilizar el proceso de escritura. E2 manifestó trabajar de forma ``híbrida, totalmente'', utilizando lo análogo para la creatividad pura, argumentando que ``el cerebro funciona distinto con papel y lápiz, es más libre'' mientras que ``la computadora te pone en modo 'editor' muy rápido, empiezas a corregir ortografía antes de tener la idea clara. Eso mata la creatividad'', reservando lo digital para ordenar y pulir. E3 coincidió definitivamente en la combinación, utilizando lo análogo para crear y lo digital para ordenar, enfatizando que el papel permite tachar y dibujar flechas de forma orgánica, aunque no permite compartir con productores. Se identificó como patrón universal la fundamentación cognitiva del enfoque híbrido: el método análogo favorece la creatividad al inhibir la edición prematura, mientras el método digital agiliza la organización y formalización.

\textit{Citas textuales representativas:}

\begin{quote}
\textbf{E1:} ``Y es una combinación de ambos [...] proyecta para mí siempre inicia análogo, siempre para mí con papel y bolígrafo [...] Una vez que yo tengo más sólida la idea, paso al método digital, porque se escribe más rápido. En el momento análogo es un momento para mi exploración.''
\end{quote}

\begin{quote}
\textbf{E2:} ``Híbrido, totalmente. Lo análogo para la creatividad pura, para soltar la mano. Lo digital para ordenar y pulir. El cerebro funciona distinto con papel y lápiz, es más libre. La computadora te pone en modo 'editor' muy rápido, empiezas a corregir ortografía antes de tener la idea clara. Eso mata la creatividad.''
\end{quote}

\begin{quote}
\textbf{E3:} ``Combinación, definitivamente. Lo análogo para crear, lo digital para ordenar. El papel te permite tachar, dibujar flechas, es más... orgánico. Pero no podés compartir un cuaderno con un productor (o con mis amigos). Lo digital es necesario para formalizar y editar rápido.''
\end{quote}

\textit{Patrones identificados:}

\begin{itemize}
	\item Convergencia total en preferencia híbrida análogo-digital - los 3 participantes
	\item Método análogo para fase creativa/exploratoria - los 3 participantes
	\item Método digital para organización/formalización - los 3 participantes
	\item Fundamentación cognitiva: papel inhibe edición prematura que ``mata la creatividad'' - E2 y E3 explícitamente
\end{itemize}

\textbf{Síntesis de la Dimensión 3:}

Los guionistas entrevistados utilizaron una combinación de herramientas análogas (cuadernos, post-its) y digitales (aplicaciones de notas, Notion, software de guion) durante la fase de pre-escritura. Se identificó como patrón universal el uso de cuadernos físicos para exploración inicial, post-its para visualización estructural en pared, celular para captura espontánea, y software específico para escritura del guion literario. La preferencia metodológica híbrida fue unánime entre los tres participantes, fundamentada cognitivamente en que el método análogo favorece la creatividad al inhibir la edición prematura que ``mata la creatividad'', mientras el método digital agiliza la organización y formalización. La secuencia metodológica de documentos fue consistente: Logline → Sinopsis → Escaleta, siendo el logline considerado documento fundacional imprescindible por los tres participantes. El tratamiento en prosa emergió como documento elaborado solo por requerimiento externo (fondos, productores), no por iniciativa propia del proceso creativo. Los hallazgos evidencian que la sistematización no implica necesariamente digitalización completa, sino la construcción de flujos de trabajo híbridos que respetan las características cognitivas de cada fase del proceso creativo.

\newpage

% ══════════════════════════════════════════════════════════════════
% 4.3.4. DIMENSIÓN 4: DESAFÍOS Y DIFICULTADES
% ══════════════════════════════════════════════════════════════════

\subsubsection{Dimensión 4: Desafíos y Dificultades en la Sistematización}

Esta dimensión respondió al Objetivo Específico 3: \textit{Describir las principales dificultades que enfrentaron los guionistas en la sistematización del proceso de pre-escritura.}

Los indicadores analizados fueron: dificultades técnicas, dificultades metodológicas, limitaciones de herramientas y necesidades no cubiertas.

% ──────────────────────────────────────────────────────────────────
% A. DIFICULTADES TÉCNICAS Y METODOLÓGICAS
% ──────────────────────────────────────────────────────────────────

\textbf{A. Dificultades técnicas y metodológicas identificadas}

Los tres guionistas entrevistados identificaron la dispersión de información entre múltiples herramientas como dificultad principal en la organización de su pre-escritura. E1 señaló que la mayor dificultad es la visualización, manifestando que nunca pudo encontrar un software que replique satisfactoriamente el tablero de corcho en versión digital, limitándose la pizarra física a su espacio de trabajo sin posibilidad de transporte. E2 describió la fragmentación como problema central: ``la biografía del protagonista está en un cuaderno de hace tres meses, la escaleta está en un Google Doc, y las notas de corrección están en un PDF en el correo'', manifestando pasar la vida haciendo Alt+Tab entre ventanas. E3 coincidió en la dispersión como dificultad principal, teniendo ``las fotos en una carpeta, la escaleta en Notion, los diálogos sueltos en las notas del celular y el guion en otro programa'', señalando que cambiar constantemente de ventana le saca del flujo creativo. Se identificó como dificultad universal la falta de visualización integrada de elementos y la pérdida de flujo creativo causada por la necesidad de cambiar entre múltiples aplicaciones y soportes.

\textit{Citas textuales representativas:}

\begin{quote}
\textbf{E1:} ``Y, justamente, la mayor dificultad es la la visualización. Como son notas sueltas, escritos, de repente visualizar [...] Y nunca pude encontrar un software específico que me haga el tablero de corchos en una versión digital de forma satisfactoria. Y eso, para mí, sería lo ideal, no no pude trabajar.''
\end{quote}

\begin{quote}
\textbf{E2:} ``La dispersión. Tengo la información fragmentada. La biografía del protagonista está en un cuaderno de hace tres meses, la escaleta está en un Google Doc, y las notas de corrección están en un PDF en el correo. Unificar todo eso a la hora de escribir la versión final es un dolor de cabeza. Me paso la vida haciendo Alt+Tab entre ventanas.''
\end{quote}

\begin{quote}
\textbf{E3:} ``La dispersión. Tengo las fotos en una carpeta, la escaleta en Notion, los diálogos sueltos en las notas del celular y el guion en otro programa. Tener que estar cambiando de ventana... [hace gesto de cambiar ventanas rápido con las manos] me saca del flujo.''
\end{quote}

% ──────────────────────────────────────────────────────────────────
% B. LIMITACIONES DE HERRAMIENTAS ACTUALES
% ──────────────────────────────────────────────────────────────────

\textbf{B. Limitaciones de herramientas actuales}

Los tres guionistas entrevistados identificaron limitaciones específicas en las herramientas actuales disponibles para la pre-escritura. E1 señaló que las aplicaciones de notas satisfacen parcialmente porque permiten escribir texto plano sin formato, pero no logran visualizar adecuadamente las relaciones entre elementos. E2 manifestó que Final Draft es excelente para formato del guion (márgenes, presentación) pero pésimo para pensar, mientras que las herramientas de notas son buenas para pensar pero malas para vincular con el guion, sintiendo que ``falta un puente'' entre ambos tipos de herramientas. E3 señaló que las herramientas de escritura son muy rígidas (enfocadas solo en texto y formato) mientras que las herramientas de organización tipo Trello ``no están pensadas para guionistas'', manifestando que no conversan entre ellas. Se identificó como limitación universal la falta de integración entre herramientas especializadas: software de guion optimizado para formato pero no para pensamiento creativo, herramientas de organización no diseñadas específicamente para narrativa audiovisual, y ausencia de vinculación dinámica entre ambos tipos de aplicaciones.

\textit{Citas textuales representativas:}

\begin{quote}
\textbf{E1:} ``Sí, como te dije, hasta cierto punto, eh; a mí me encantan las aplicaciones notas porque escribo, no tiene, no es necesario que tengan formato, solamente es el texto plano, puntual, que es lo que quiero anotar. Eh, Lo que no me satisfaga, lo que me gustaría que sea mejor es [...] que no puedo visualizar demasiado bien en general.''
\end{quote}

\begin{quote}
\textbf{E2:} ``A medias. Final Draft es genial para formato, te pone los márgenes bonitos, pero es pésimo para \textit{pensar}. Y las herramientas de notas son buenas para pensar pero malas para vincular con el guion. Siento que falta un puente.''
\end{quote}

\begin{quote}
\textbf{E3:} ``A medias. O sea, funcionan, se puede escribir. Pero siento que las herramientas de escritura son muy rígidas (solo texto, formato, margen) y las herramientas de organización (tipo Trello) no están pensadas para guionistas. No conversan entre ellas.''
\end{quote}

% ──────────────────────────────────────────────────────────────────
% C. NECESIDADES NO CUBIERTAS
% ──────────────────────────────────────────────────────────────────

\textbf{C. Necesidades no cubiertas por herramientas existentes}

Los tres guionistas entrevistados expresaron necesidades específicas no satisfechas por las herramientas actuales, convergiendo en tres requerimientos fundamentales. E1 manifestó necesidad de ``ver dentro de un software todas las ideas entre sí, cómo se relacionan'', enfatizando la visualización de relaciones como funcionalidad crítica ausente. E2 expresó el deseo de ``un software donde yo pueda tener mi guion en el centro, y si hago clic en el nombre de 'JUAN', se despliegue al lado su ficha, sus fotos, sus notas... sin tapar el guion'', agregando la necesidad de que ``si cambio algo en la ficha, me avise en el guion'', conceptualizando esto como ``una especie de... ecosistema conectado'' donde ``la pre-escritura viva dentro del software de escritura''. E3 manifestó el ``sueño dorado'' de ``una pantalla dividida inteligente'' donde ``si hago click en la Escena 5 del guion, a la derecha automáticamente me aparezcan las fotos, las notas y los objetivos de ESA escena. Que no tenga que buscarlo'', definiendo esto como ``integración visual inmediata''. Se identificaron tres necesidades universales: visualización integrada de relaciones entre elementos, unificación de información dispersa en ecosistema único, y vinculación dinámica contextual entre elementos narrativos y guion.

\textit{Citas textuales representativas:}

\begin{quote}
\textbf{E1:} ``Y lo que dije, la visualización, que yo pueda ver dentro de un software todas las ideas entre sí, cómo se relacionan.''
\end{quote}

\begin{quote}
\textbf{E2:} ``¡Uf! Sueño con esto. Me gustaría un software donde yo pueda tener mi guion en el centro, y si hago clic en el nombre de 'JUAN', se despliegue al lado su ficha, sus fotos, sus notas... sin tapar el guion. Y que si cambio algo en la ficha, me avise en el guion. Una especie de... ecosistema conectado. Que no sean archivos muertos, sino que la pre-escritura viva dentro del software de escritura.''
\end{quote}

\begin{quote}
\textbf{E3:} ``¡Uff, tengo el sueño dorado! Me gustaría una pantalla dividida inteligente. Que a la izquierda tenga mi guion, y a la derecha tenga mi 'muro creativo' visual. Que si hago click en la Escena 5 del guion, a la derecha automáticamente me aparezcan las fotos, las notas y los objetivos de ESA escena. Que no tenga que buscarlo. \textbf{Integración visual inmediata}. Eso... eso me cambiaría la vida.''
\end{quote}

% [INSTRUCCIÓN]: Completar tabla de frecuencia con datos reales
% Contar cuántos participantes mencionaron cada dificultad

\begin{table}[ht]
	\centering
	\renewcommand{\arraystretch}{1.3}
	\begin{tabular}{|p{7cm}|c|c|c|c|}
		\hline
		\textbf{Dificultad / Necesidad} & \textbf{E1} & \textbf{E2} & \textbf{E3} & \textbf{Total} \\ \hline

		Falta de visualización integrada de elementos & ✓ & ✓ & ✓ & 3/3 \\ \hline

		Dispersión de información en múltiples herramientas & ✓ & ✓ & ✓ & 3/3 \\ \hline

		Falta de vinculación dinámica entre elementos narrativos y guion & ✓ & ✓ & ✓ & 3/3 \\ \hline

		Pérdida de flujo creativo al cambiar entre aplicaciones & ✗ & ✓ & ✓ & 2/3 \\ \hline

		Software especializado demasiado complejo (Scrivener) & ✗ & ✓ & ✓ & 2/3 \\ \hline

		Dificultad para reordenar estructura fácilmente (limitación de pizarra física) & ✓ & ✗ & ✓ & 2/3 \\ \hline

		Dificultad para organizar/transcribir ideas capturadas en celular & ✓ & ✓ & ✗ & 2/3 \\ \hline

		Herramientas de escritura pésimas para pensar creativamente & ✗ & ✓ & ✗ & 1/3 \\ \hline

		Software propietario costoso (Final Draft) & ✗ & ✗ & ✓ & 1/3 \\ \hline
	\end{tabular}
	\caption{Frecuencia de dificultades y necesidades mencionadas por los participantes}
	\label{tab:dificultades}
\end{table}

\textbf{Síntesis de la Dimensión 4:}

Las tres dificultades universales identificadas fueron: falta de visualización integrada de elementos, dispersión de información en múltiples herramientas, y ausencia de vinculación dinámica entre elementos narrativos y guion. Los participantes expresaron necesidad de un ``ecosistema conectado'' (E2) que permita ``integración visual inmediata'' (E3) y capacidad de ``ver todas las ideas entre sí, cómo se relacionan'' (E1). Las limitaciones identificadas en herramientas actuales revelaron una dicotomía fundamental: el software especializado para escritura de guiones (Final Draft) está optimizado para formato pero resulta inadecuado para pensamiento creativo, mientras que las herramientas de organización (Notion, Trello) no están diseñadas específicamente para narrativa audiovisual y carecen de vinculación con el guion. La pérdida de flujo creativo causada por cambios constantes entre aplicaciones (``hacer Alt+Tab entre ventanas'') emergió como consecuencia directa de la fragmentación de información. Los hallazgos evidencian que las necesidades expresadas no constituyen requerimientos superficiales sino carencias fundamentales que afectan directamente la eficacia del proceso creativo, proporcionando base empírica sólida para el desarrollo de soluciones integradas que unifiquen pre-escritura y escritura de guion en ecosistema único con visualización relacional y vinculación dinámica contextual.

\newpage

% ══════════════════════════════════════════════════════════════════
% 4.4. DISCUSIÓN DE RESULTADOS
% ══════════════════════════════════════════════════════════════════

\subsection{Discusión de Resultados}

\subsubsection{Hallazgos Principales por Objetivo Específico}

% ──────────────────────────────────────────────────────────────────
% DISCUSIÓN OE1
% ──────────────────────────────────────────────────────────────────

\textbf{Respecto al Objetivo Específico 1: Organización de elementos narrativos}

Los hallazgos revelaron que los tres guionistas entrevistados emplearon métodos diversos para la organización de elementos narrativos, identificándose tanto patrones comunes como particularidades individuales. La convergencia más significativa se manifestó en el rechazo universal a fichas técnicas tradicionales estandarizadas, desarrollando los tres participantes sistemas propios personalizados que priorizan aspectos psicológicos sobre características físicas descriptivas.

En relación a la gestión de personajes, se observó que E1, E2 y E3 coincidieron en utilizar métodos visuales (fotografías de actores o personas reales) y enfocarse en motivaciones internas del personaje. E2 conceptualizó esto como ``la herida del pasado'' que funciona como ``brújula'' narrativa, mientras E3 lo redujo al binomio esencial Deseo/Miedo. Esta priorización de aspectos psicológicos sobre información física superficial contrasta con plantillas tradicionales que los participantes consideraron irrelevantes para la funcionalidad narrativa.

Respecto a las locaciones, los participantes manifestaron utilizar referencias visuales (Pinterest, Google Maps, visitas físicas) considerando factores productivos (viabilidad, presupuesto) en la selección. E2 se diferenció por desarrollar descripciones sensoriales elaboradas (olores, sonidos), conceptualizando la locación como ``un personaje más'', mientras E1 y E3 mantuvieron descripciones básicas funcionales.

En cuanto a la captura de ideas, se identificó el uso intensivo y universal del celular (notas de voz, aplicaciones) para registro espontáneo, emergiendo como dificultad recurrente la organización y transcripción posterior de esas ideas. E1 se diferenció por haber desarrollado un sistema estructurado de organización (Standard Notes con carpetas categorizadas), mientras E2 y E3 manifestaron acumular ideas sin sistematizar, quedando ``enterradas meses'' (E2) o sin transferir a computadora (E3).

% ──────────────────────────────────────────────────────────────────
% DISCUSIÓN OE2
% ──────────────────────────────────────────────────────────────────

\textbf{Respecto al Objetivo Específico 2: Herramientas y métodos de estructuración}

Los hallazgos revelaron patrones altamente consistentes en las herramientas y métodos utilizados por los guionistas para estructurar dramáticamente sus obras antes del guion literario. La convergencia más significativa se manifestó en tres aspectos: el paradigma estructural de tres actos como base, el proceso bifásico creación-estructuración, y la preferencia híbrida análogo-digital con fundamentación cognitiva explícita.

En relación a los paradigmas estructurales, los tres participantes manifestaron utilizar predominantemente el modelo de tres actos, aunque con conocimiento y aplicación adaptativa de modelos alternativos (Save the Cat, Viaje del Héroe, método secuencial). E2 y E3 enfatizaron la importancia de puntos de giro específicos (Midpoint, Clímax) como ``pilares'' estructurales. La escaleta emergió como documento central universal, siendo considerada ``sagrada'' (E2) y de uso ``religioso'' (E3).

Respecto al proceso de estructuración, se identificó un patrón bifásico consistente: fase inicial de exploración creativa desestructurada (``proceso vómito'' - E1) seguida de fase de organización estructural rigurosa. Los tres participantes enfatizaron que la estructura debe definirse completamente antes de iniciar la escritura del guion literario, aprendizaje que E3 adquirió tras ``trabarse en la página 10'' al lanzarse directamente a escribir.

En cuanto a la visualización estructural, se identificó convergencia total en el uso de post-its o tarjetas físicas en pared, con códigos de color y necesidad de manipulación táctil. E2 explicitó el fundamento: ``esa sensación táctil de mover la historia... ninguna pantalla me la da igual''. El rechazo a soluciones digitales (Trello) se fundamentó en la imposibilidad de obtener visión panorámica sin scroll.

Respecto a las herramientas utilizadas, todos manifestaron preferencia híbrida análogo-digital con fundamentación cognitiva explícita: el método análogo favorece creatividad al inhibir edición prematura que ``mata la creatividad'' (E2), mientras el método digital agiliza organización y formalización. La secuencia metodológica de documentos fue consistente: Logline → Sinopsis → Escaleta, siendo el logline considerado fundamento imprescindible (``si no puedo contártelo en dos frases, no sé qué estoy escribiendo''). El tratamiento en prosa emergió como documento elaborado solo por requerimiento externo, no por iniciativa propia.

% ──────────────────────────────────────────────────────────────────
% DISCUSIÓN OE3
% ──────────────────────────────────────────────────────────────────

\textbf{Respecto al Objetivo Específico 3: Dificultades en la sistematización}

Los hallazgos revelaron tres dificultades universales identificadas por los tres participantes: falta de visualización integrada de elementos, dispersión de información en múltiples herramientas, y ausencia de vinculación dinámica entre elementos narrativos y guion. Estas dificultades no constituyen inconvenientes menores sino carencias fundamentales que afectan directamente la eficacia del proceso creativo.

La dispersión de información emergió como problema central, manifestándose los participantes tener ``la biografía del protagonista en un cuaderno de hace tres meses, la escaleta en un Google Doc, y las notas de corrección en un PDF en el correo'' (E2), o ``las fotos en una carpeta, la escaleta en Notion, los diálogos sueltos en las notas del celular y el guion en otro programa'' (E3). Esta fragmentación genera pérdida de flujo creativo al requerir cambios constantes entre aplicaciones (``hacer Alt+Tab entre ventanas'' - E2).

Las limitaciones identificadas en herramientas actuales revelaron una dicotomía problemática: el software especializado para escritura de guiones (Final Draft) está optimizado para formato pero resulta ``pésimo para pensar'' (E2), mientras que las herramientas de organización (Notion, Trello) ``no están pensadas para guionistas'' y ``no conversan entre ellas'' (E3). E2 conceptualizó esto como falta de ``un puente'' entre herramientas de pensamiento creativo y herramientas de escritura formal.

Los participantes expresaron necesidades específicas convergentes: ``ver dentro de un software todas las ideas entre sí, cómo se relacionan'' (E1), ``un ecosistema conectado'' donde ``la pre-escritura viva dentro del software de escritura'' (E2), e ``integración visual inmediata'' mediante ``pantalla dividida inteligente'' donde la información contextual aparezca automáticamente sin necesidad de búsqueda manual (E3). Estas necesidades evidencian requerimientos no satisfechos por herramientas existentes, proporcionando base empírica para el desarrollo de soluciones integradas.

\newpage

\subsubsection{Contraste con el Marco Teórico}

% [INSTRUCCIÓN]: Relacionar hallazgos con autores citados en Capítulo II
% Referencias clave: Field (2005), McKee (1997), Koestler (1964), Lovato (2020)

% ──────────────────────────────────────────────────────────────────
% CONTRASTE CON FIELD
% ──────────────────────────────────────────────────────────────────

\textbf{Contraste con el paradigma de Field (2005)}

Los hallazgos del estudio mostraron coherencia con el paradigma estructural propuesto por Field (2005), quien estableció que la organización sistemática de elementos narrativos mediante el paradigma de tres actos es fundamental antes de iniciar la escritura del guion. Los tres participantes manifestaron utilizar explícitamente este modelo como base estructural, coincidiendo con el énfasis de Field en que la estructura debe definirse completamente antes de escribir diálogos. E2 y E3 enfatizaron particularmente los puntos de giro (Midpoint, Clímax) que Field identifica como pilares estructurales del paradigma. La escaleta emergió como documento central, coincidiendo con la propuesta de Field de que la estructura debe visualizarse antes del guion literario. Los hallazgos validan empíricamente la vigencia del paradigma de Field en la práctica profesional contemporánea de guionistas paraguayos.

% ──────────────────────────────────────────────────────────────────
% CONTRASTE CON MCKEE
% ──────────────────────────────────────────────────────────────────

\textbf{Contraste con McKee (1997) sobre organización narrativa}

McKee (1997) enfatizó la importancia de la organización narrativa consciente y el desarrollo profundo de personajes basado en dimensiones psicológicas. Los hallazgos del presente estudio revelaron que los tres participantes aplicaron este principio mediante sistemas personalizados que priorizaron aspectos psicológicos sobre descripciones físicas superficiales. E2 manifestó utilizar ``la herida del pasado'' como brújula narrativa, coincidiendo directamente con el concepto de McKee sobre dimensiones caracterológicas profundas. E3 empleó el binomio ``Deseo y Miedo'' como sistema de caracterización, alineándose con el énfasis de McKee en motivaciones internas. Los tres participantes rechazaron fichas técnicas tradicionales centradas en datos físicos, validando la propuesta de McKee de que la caracterización efectiva requiere exploración psicológica antes que inventario de atributos externos. Los sistemas propios desarrollados por los guionistas reflejan organización consciente y metodológica, tal como McKee propone para la construcción narrativa profesional.

% ──────────────────────────────────────────────────────────────────
% CONTRASTE CON KOESTLER
% ──────────────────────────────────────────────────────────────────

\textbf{Contraste con el modelo de proceso creativo de Koestler (1964)}

El modelo de proceso creativo propuesto por Koestler (1964) identificó tres fases fundamentales: fase lógica (planificación), fase intuitiva (desarrollo creativo) y fase crítica (refinamiento). Los hallazgos del estudio evidenciaron que los tres guionistas transitaron estas fases de manera consistente, aunque no necesariamente en orden lineal. E1 manifestó explícitamente iniciar con el ``proceso vómito'', caracterizado por exploración libre sin restricciones estructurales, correspondiente a la fase intuitiva de Koestler. Posteriormente, los participantes reportaron organizar este material mediante paradigmas estructurales (tres actos, secuencias), correspondiente a la fase lógica. E2 enfatizó la importancia de la escaleta como documento que ``debe funcionar'' antes de escribir diálogos, evidenciando fase crítica de refinamiento. El proceso bifásico identificado (exploración creativa → estructuración) refleja la alternancia entre fases intuitiva y lógica que Koestler propone como característica del acto creativo. Los hallazgos validan empíricamente el modelo trifásico aplicado al contexto específico de la escritura de guiones.

% ──────────────────────────────────────────────────────────────────
% CONTRASTE CON LOVATO
% ──────────────────────────────────────────────────────────────────

\textbf{Contraste con la sistematización del proceso transmedia (Lovato, 2020)}

Lovato (2020) definió la sistematización del proceso creativo como la estructuración metodológica de las fases de desarrollo narrativo. Los hallazgos del presente estudio evidenciaron que los tres participantes emplearon secuencias metodológicas consistentes, particularmente el orden Logline → Sinopsis → Escaleta, coincidiendo con el concepto de sistematización propuesto por Lovato. E1, E2 y E3 manifestaron seguir esta secuencia de manera invariable, evidenciando consciencia metodológica en la organización del proceso creativo. Sin embargo, los hallazgos también revelaron una contradicción significativa: aunque los guionistas sistematizaron sus procesos mediante documentos estructurados, la gestión de estos elementos se realizó mediante soluciones parciales no integradas. La dispersión de información identificada en la Dimensión 4 (fragmentación en múltiples herramientas) sugiere que la sistematización actual permanece incompleta. Los participantes sistematizaron \textit{qué} hacer, pero carecen de sistemas integrados para \textit{cómo} gestionar eficientemente los elementos durante el proceso. Esta brecha entre sistematización metodológica y sistematización operativa constituye hallazgo central del estudio.

% ──────────────────────────────────────────────────────────────────
% HALLAZGOS EMERGENTES
% ──────────────────────────────────────────────────────────────────

\textbf{Hallazgos no anticipados por el marco teórico}

El estudio identificó hallazgos emergentes no anticipados explícitamente por el marco teórico consultado. En primer lugar, los tres participantes articularon fundamentación cognitiva para su preferencia por métodos híbridos análogo-digitales: E2 manifestó que ``el cerebro funciona distinto con papel y lápiz, es más libre'', mientras E1 explicó que el método análogo permite ``exploración'' sin activar prematuramente el modo editor. Esta distinción cognitiva entre herramientas no fue abordada por Field, McKee ni Lovato. En segundo lugar, emergió rechazo universal a fichas técnicas tradicionales de personajes, consideradas ``encorsetadoras'' (E2) y que ``matan la creatividad'' (E3). Los guionistas priorizaron flexibilidad sobre estandarización, contradiciendo manuales prescriptivos de escritura de guiones. En tercer lugar, los participantes enfatizaron importancia de manipulación táctil física en la visualización estructural: E2 manifestó que ``esa sensación táctil de mover la historia... ninguna pantalla me la da igual''. Esta dimensión háptica del proceso creativo no aparece en la literatura teórica consultada. Finalmente, el celular emergió como herramienta central de captura espontánea, pero con dificultad recurrente de ``ideas enterradas'' (E2) que nunca se organizan. Estos hallazgos sugieren dimensiones del proceso creativo (cognitivas, hápticas, tecnológicas) que merecen investigación futura.

\newpage

\subsubsection{Implicaciones para la Práctica Profesional y Justificación de DREAMINK}

% [INSTRUCCIÓN]: SECCIÓN CRÍTICA - Conectar hallazgos con justificación de DREAMINK
% Debe explicar por qué DREAMINK es una respuesta a las necesidades identificadas

% ──────────────────────────────────────────────────────────────────
% NECESIDADES IDENTIFICADAS
% ──────────────────────────────────────────────────────────────────

\textbf{Necesidades identificadas en el estudio}

El análisis de contenido de las entrevistas reveló tres necesidades críticas expresadas consistentemente por los tres participantes, las cuales no fueron satisfechas por las herramientas existentes en el mercado ni por los sistemas propios desarrollados por los guionistas.

\textbf{Primera necesidad identificada: Visualización integrada de elementos narrativos}

Los tres participantes manifestaron dificultad para visualizar simultáneamente todos los elementos del proyecto y sus relaciones. E1 expresó explícitamente: ``que yo pueda ver dentro de un software todas las ideas entre sí, cómo se relacionan''. E2 manifestó frustración por tener ``información fragmentada en quince lugares diferentes''. E3 solicitó ``integración visual inmediata'' donde elementos contextuales aparezcan automáticamente al trabajar en una escena. Aunque los guionistas emplearon post-its físicos en pared como solución parcial, esta estrategia presenta limitaciones de portabilidad (E1: ``esa pizarra no puedo llevar a todos lados'') y no permite vinculación dinámica entre elementos. Las herramientas digitales probadas (Trello, pizarras virtuales) fueron rechazadas por requerir scroll y no ofrecer ``panorama de un golpe de vista'' (E3).

\textbf{Segunda necesidad identificada: Unificación de información dispersa}

Los tres participantes reportaron dispersión de información en múltiples herramientas desconectadas como dificultad principal. E2 describió: ``La biografía del protagonista está en un cuaderno de hace tres meses, la escaleta está en un Google Doc, y las notas de corrección están en un PDF en el correo''. E3 manifestó: ``Tengo las fotos en una carpeta, la escaleta en Notion, los diálogos sueltos en las notas del celular''. Esta fragmentación genera pérdida de flujo creativo al cambiar constantemente entre aplicaciones (E2: ``Me paso la vida haciendo Alt+Tab entre ventanas''). Los participantes expresaron necesidad de centralización que mantenga flexibilidad, rechazando soluciones como Scrivener por ``complejidad excesiva'' que ``mata la creatividad''.

\textbf{Tercera necesidad identificada: Vinculación dinámica entre elementos}

Los participantes manifestaron necesidad de relaciones automáticas entre entidades narrativas sin navegación manual. E2 expresó: ``Me gustaría un software donde yo pueda tener mi guion en el centro, y si hago clic en el nombre de 'JUAN', se despliegue al lado su ficha, sus fotos, sus notas... sin tapar el guion. Y que si cambio algo en la ficha, me avise en el guion''. E3 solicitó ``pantalla dividida inteligente'' donde información contextual aparezca automáticamente según escena activa. E1 manifestó necesidad de ``ver cómo se relacionan'' ideas entre sí. Esta necesidad refleja concepto de ``ecosistema conectado'' (E2) donde modificaciones se propaguen automáticamente y navegación entre elementos sea contextual e inmediata.

% ──────────────────────────────────────────────────────────────────
% JUSTIFICACIÓN DE DREAMINK
% ──────────────────────────────────────────────────────────────────

\textbf{DREAMINK como respuesta a las necesidades identificadas}

Los hallazgos del presente estudio proporcionaron evidencia empírica que justificó el desarrollo del sistema DREAMINK como herramienta de sistematización del proceso de pre-escritura de guiones, diseñada específicamente para resolver las tres necesidades críticas identificadas.

\textbf{Respuesta a la necesidad de visualización integrada}

En respuesta a la necesidad de visualización panorámica manifestada por los tres participantes, DREAMINK incorporó interfaz tipo tablero Kanban que replica digitalmente la experiencia de post-its en pared, resolviendo simultáneamente las limitaciones de portabilidad y falta de vinculación. El sistema permite visualizar estructura dramática completa (actos, secuencias, escenas) en vista panorámica sin scroll, respondiendo directamente a la solicitud de E3 de ``ver todo el panorama de un golpe de vista''. La representación visual mediante tarjetas organizadas en columnas configurables permite replicar la práctica universal identificada (organización física en pared) mientras añade capacidades digitales: búsqueda, filtrado, reorganización mediante arrastre, y exportación. El diseño preserva la experiencia táctil de ``mover la historia'' (E2) mediante interacciones de arrastre que simulan manipulación física, reconociendo la importancia de la dimensión háptica identificada como hallazgo emergente.

\textbf{Respuesta a la necesidad de unificación de información}

En relación a la dispersión de información reportada como dificultad principal, DREAMINK centralizó en un único sistema la gestión de personajes, locaciones, ideas, documentos de pre-escritura (logline, sinopsis, escaleta) y estructura dramática. El sistema elimina necesidad de cambiar entre múltiples aplicaciones (Standard Notes, Google Docs, Notion, carpetas de archivos) que caracterizó el flujo de trabajo fragmentado identificado. A diferencia de soluciones rechazadas como Scrivener, DREAMINK mantiene flexibilidad mediante arquitectura modular donde cada tipo de elemento (personaje, locación, idea) posee campos personalizables, respetando el rechazo universal a plantillas rígidas identificado en la Dimensión 1. La centralización no impone estructura, sino que proporciona contenedor unificado para sistemas propios que los guionistas ya desarrollaron.

\textbf{Respuesta a la necesidad de vinculación dinámica}

Respecto a la necesidad de ``ecosistema conectado'' expresada por E2, DREAMINK implementó sistema de vinculación relacional donde elementos narrativos se conectan automáticamente. Cuando un guionista referencia un personaje en una escena, el sistema establece relación bidireccional que permite navegación contextual inmediata: desde la escena hacia la ficha del personaje, y desde el personaje hacia todas las escenas donde aparece. La ``pantalla dividida inteligente'' solicitada por E3 se implementa mediante panel lateral contextual que despliega automáticamente información relevante sin tapar contenido principal, respondiendo literalmente a la solicitud de E2: ``si hago clic en 'JUAN', se despliegue al lado su ficha, sus fotos, sus notas... sin tapar el guion''. El sistema propaga modificaciones: cambiar nombre de personaje actualiza automáticamente todas las referencias, notificando al usuario mediante sistema de alertas.

% ──────────────────────────────────────────────────────────────────
% CONTRIBUCIÓN AL SECTOR
% ──────────────────────────────────────────────────────────────────

\textbf{Contribución al sector audiovisual paraguayo}

Los hallazgos del estudio revelaron que los guionistas paraguayos entrevistados emplean métodos de sistematización consistentes y profesionales, comparables a estándares internacionales establecidos por Field, McKee y otros teóricos consultados. Los tres participantes demostraron dominio de paradigmas estructurales, secuencias metodológicas claras (Logline → Sinopsis → Escaleta), y desarrollo de sistemas propios adaptados a sus necesidades específicas. Sin embargo, las limitaciones tecnológicas identificadas (falta de visualización integrada, dispersión de información, ausencia de vinculación dinámica) obstaculizan la eficiencia de estos procesos creativos ya sistematizados.

DREAMINK, al resolver específicamente las tres necesidades universales identificadas empíricamente en el contexto local, puede optimizar los flujos de trabajo de guionistas paraguayos sin imponer métodos ajenos a sus prácticas actuales. El sistema digitaliza y conecta procesos que los guionistas ya emplean, respetando los fundamentos cognitivos identificados (separación entre fases creativa y editorial, flexibilidad sobre estandarización, visualización panorámica). La herramienta reduce fricción tecnológica permitiendo que guionistas dediquen más tiempo al desarrollo creativo y menos a gestión operativa de elementos dispersos.

La contribución al sector audiovisual paraguayo trasciende la herramienta individual. El desarrollo de DREAMINK constituye caso de software diseñado mediante investigación empírica local, respondiendo a necesidades reales articuladas por profesionales del medio antes que a supuestos importados. Este enfoque metodológico (investigación cualitativa → identificación de necesidades → desarrollo de solución tecnológica) establece precedente para futuros desarrollos de herramientas especializadas basadas en evidencia. Adicionalmente, al tratarse de sistema diseñado por y para guionistas paraguayos, DREAMINK puede incorporar especificidades del contexto productivo local (limitaciones presupuestarias, énfasis en bajo presupuesto mencionado por E3, necesidad de optimización de recursos) que herramientas internacionales no consideran.

Finalmente, los hallazgos evidenciaron que los guionistas locales ya sistematizaron metodológicamente sus procesos, contradiciendo posibles prejuicios sobre informalidad o falta de profesionalización en el sector audiovisual paraguayo. E1, E2 y E3 manifestaron consciencia metodológica, conocimiento teórico sólido (Field, McKee, Snyder) y desarrollo proactivo de soluciones organizativas propias. DREAMINK no ``enseña'' a sistematizar, sino que proporciona infraestructura tecnológica para sistematización que ya existe en la práctica profesional local.

% ──────────────────────────────────────────────────────────────────
% ALINEACIÓN CON PRÁCTICAS PROFESIONALES
% ──────────────────────────────────────────────────────────────────

\textbf{Alineación con prácticas profesionales identificadas}

DREAMINK se fundamenta en el principio de respetar y potenciar prácticas profesionales que los guionistas ya emplean, en lugar de imponer metodologías prescriptivas ajenas a sus flujos de trabajo establecidos. El diseño del sistema se alinea directamente con cinco patrones identificados consistentemente en las tres entrevistas.

Primero, el sistema replica digitalmente la visualización mediante post-its en pared, práctica universal identificada en los tres participantes. La interfaz tipo Kanban con tarjetas arrastrables preserva la lógica espacial y táctil que los guionistas valoran, mientras resuelve limitaciones de portabilidad. Segundo, DREAMINK respeta la secuencia metodológica Logline → Sinopsis → Escaleta que los tres guionistas manifestaron seguir invariablemente, estructurando el flujo de trabajo del sistema según este orden establecido. Tercero, el sistema mantiene flexibilidad mediante campos personalizables en lugar de plantillas rígidas, alineándose con el rechazo universal a fichas estándar que ``matan la creatividad''. Cuarto, DREAMINK reconoce el enfoque híbrido análogo-digital identificado, permitiendo importación de contenido creado análogamente (escaneos de cuadernos, fotos de post-its) para integración digital sin forzar creación directa en computadora durante fase exploratoria. Quinto, el sistema prioriza aspectos psicológicos de personajes mediante campos dedicados a motivaciones, deseos, miedos y ``herida del pasado'', reflejando el enfoque caracterológico que E2 y E3 manifestaron emplear.

Esta alineación con prácticas existentes contrasta con herramientas como Scrivener (rechazada por complejidad) o Final Draft (limitada a formato de guion). DREAMINK no ``enseña'' metodología nueva, sino que digitaliza, unifica y conecta procesos que los guionistas ya dominan pero gestionan mediante soluciones fragmentadas. El sistema actúa como infraestructura tecnológica que soporta sistematización metodológica preexistente, reduciendo fricción operativa sin alterar fundamentos creativos del proceso.

\newpage

% ══════════════════════════════════════════════════════════════════
% FIN DEL CAPÍTULO IV - TEMPLATE
% ══════════════════════════════════════════════════════════════════
% RECORDATORIO FINAL:
%
% Este template debe ser completado con datos reales de las 3 entrevistas.
% Siga el Protocolo de Aplicación (docs/instrumento_entrevista/ProtocoloAplicacion.md)
% Use la Guía de Entrevista (docs/instrumento_entrevista/GuiaEntrevista.md)
% Aplique análisis de contenido cualitativo a las transcripciones reales
%
% Extensión estimada del capítulo completo: 25-34 páginas
% ══════════════════════════════════════════════════════════════════
