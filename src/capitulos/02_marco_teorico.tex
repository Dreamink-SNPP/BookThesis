% ══════════════════════════════════════════════════════════════════
% CAPÍTULO II - MARCO TEÓRICO
% ══════════════════════════════════════════════════════════════════

\clearpage
\thispagestyle{empty}
\vspace*{\fill}
\begin{center}
    {\Large\bfseries CAPÍTULO II - MARCO TEÓRICO}
\end{center}
\vspace*{\fill}
\clearpage

\section*{CAPÍTULO II - MARCO TEÓRICO}
\addcontentsline{toc}{section}{CAPÍTULO II - MARCO TEÓRICO}
\setcounter{section}{2}

\subsection{Antecedentes de la Investigación}

El formato Fountain se consolidó como estándar abierto para escritura de guiones, pero su adopción se limitó a la etapa técnica. Investigaciones recientes (Ávalos, 2023) destacaron que el 78\% de guionistas independientes carecieron de herramientas para integración fluida entre conceptualización y escritura. Proyectos como Trelby y WriterSolo demostraron la viabilidad de soluciones Open Source, pero no abordaron la estructuración dramática previa.

\subsection{Bases Teóricas}

Se expusieron los siguientes conceptos para el marco teórico del trabajo. Tanto para la parte artística de las obras audiovisuales, como las partes técnicas que se usaron en el proyecto.

\subsubsection{Sistematización del Proceso de Pre-escritura de Guiones}

La sistematización del proceso creativo constituyó un área de estudio fundamental en la investigación sobre guionismo y narrativa audiovisual. Según la literatura académica reciente sobre guión transmedia, la sistematización del proceso creativo se entendió como la estructuración metodológica de las fases de desarrollo narrativo, especialmente aquellas que antecedieron a la redacción del guion literario \parencite{guion_transmedia_sistematizacion}.

Este proceso de sistematización cobró particular relevancia en la fase de pre-escritura, entendida como el conjunto de actividades previas a la redacción formal del guion literario. \textcite{field2005} estableció que la organización sistemática de elementos narrativos mediante el paradigma estructural fue fundamental antes de iniciar la escritura del guion. En este sentido, la pre-escritura comprendió la gestión de personajes, locaciones, ideas y estructuras dramáticas que posteriormente configuraron el relato audiovisual.

\textcite{mckee1997}, por su parte, enfatizó la importancia de los procesos de organización narrativa como fundamento de la construcción dramática efectiva. McKee sostuvo que la comprensión profunda de la estructura narrativa y la organización consciente de sus elementos constituyeron pilares esenciales para el desarrollo de historias coherentes y significativas.

El modelo de proceso creativo propuesto por \textcite{koestler1964} identificó tres fases fundamentales que resultaron aplicables al contexto de la pre-escritura de guiones: la fase lógica (definición del problema y planificación), la fase intuitiva (desarrollo creativo y exploración de ideas) y la fase crítica (verificación y refinamiento). Este modelo teórico proporcionó un marco conceptual para comprender cómo los guionistas organizaron y sistematizaron su trabajo creativo.

La literatura profesional contemporánea sobre escritura de guiones reconoció la importancia de la organización en el proceso creativo. Diversos autores destacaron que la estructuración sistemática del material narrativo durante la fase de pre-escritura no solo mejoró la eficiencia del proceso, sino que también contribuyó a la calidad final del producto audiovisual \parencite{finaldraft_organization, masterclass_screenplay_structure}.

En el contexto de la práctica profesional, los métodos de sistematización pudieron incluir diversas herramientas y técnicas: desde el uso de tarjetas de colores para organizar secuencias, hasta la elaboración de documentos como loglines, sinopsis, escaletas y tratamientos. La investigación sobre procesos creativos de guionistas profesionales reveló una amplia variedad de enfoques personalizados, cada uno adaptado a las necesidades específicas del creador y del proyecto \parencite{industrialscripts_writing_process, screencraft_process}.

La estructura dramática, elemento central de la organización narrativa, se conceptualizó como la disposición secuencial de eventos, personajes y conflictos dentro de un marco temporal y causal coherente. \textcite{field1979} propuso el paradigma de tres actos —configuración, confrontación y resolución— como sistema básico de organización, mientras que \textcite{gulino2004} profundizó en la estructuración mediante secuencias como unidades narrativas intermedias.

Fue importante destacar que la sistematización del proceso de pre-escritura no implicó necesariamente rigidez creativa, sino más bien la adopción de métodos organizativos que facilitaron la materialización de ideas narrativas complejas. Como señalaron diversos estudios sobre procesos creativos, la flexibilidad metodológica y la adaptabilidad de los sistemas de organización a las características particulares de cada proyecto constituyeron aspectos fundamentales de la práctica profesional contemporánea \parencite{reedsy_story_structure, masterclass_narrative_structure}.

En el contexto latinoamericano y específicamente hispanohablante, la sistematización del proceso creativo fue abordada desde diversas perspectivas académicas y profesionales, destacándose la importancia de las etapas de inspiración, desarrollo y estructuración como fases diferenciadas del proceso creativo aplicado a la escritura de guiones \parencite{espaciocreacine_proceso, espaciocreacine_inspiracion}.

\subsubsection{Guion Literario}

Un guion literario fue una historia desarrollada mediante imágenes, diálogos y descripciones, ubicada dentro de una estructura dramática definida. Este tipo de guión se consideró la base sobre la que se construyó el guion técnico y adaptativo, según \textcite{field2005} y \textcite{gulino2004}.

\subsubsection{Logline}

Según el autor \textcite{snyder2005}, un logline fue un resumen breve —normalmente de una o dos frases— que presentó el conflicto central de una historia, diseñado para captar la atención del lector o espectador. Proporcionó una visión clara y atractiva del núcleo narrativo del proyecto.

\subsubsection{Storyline}

Según \textcite{trottier2005}, el \textit{storyline} fue la secuencia de eventos que constituyeron la trama de una obra narrativa, como un libro, una película, una obra teatral, entre otros. Esta secuencia organizó las acciones de los personajes, los conflictos y sus consecuencias en una progresión estructurada que permitió comprender el desarrollo de la historia.

A diferencia de la sinopsis o del logline, que fueron formatos de resumen, el storyline reflejó el contenido central del relato, ya que delimitó cómo ocurrieron los hechos, en qué orden, y con qué propósito narrativo \parencite{cambridge_storyline}.

Según el \textit{Cambridge Dictionary}, el término \textit{storyline} se refirió a la historia principal en una obra audiovisual o literaria, como una película, un libro o una obra teatral. Esta definición delimitó el núcleo estructural de la narración, distinguiéndolo de elementos secundarios o subtramas.

En este sentido, el storyline representó el hilo conductor esencial que articuló el relato completo y le otorgó coherencia global \parencite{cambridge_storyline}.

\subsubsection{Sinopsis}

Según \textcite{howard1995}, la \textit{sinopsis} fue un resumen breve y estructurado de una obra narrativa, ya sea audiovisual o literaria, que tuvo como objetivo principal captar la atención del lector, productor o evaluador del proyecto.

Su función fue presentar de manera condensada los elementos centrales de la historia —como los personajes principales, el conflicto esencial, el tono y la progresión dramática— sin desarrollar en profundidad los acontecimientos. Solió utilizarse como herramienta de venta, presentación o evaluación de guiones y proyectos creativos, y fue fundamental en el contexto de pitchings o convocatorias abiertas \parencite{studiobinder_sinopsis}.

\subsubsection{Sinopsis (según fuentes académicas y profesionales)}

La \textit{sinopsis} fue un resumen conciso que describió la trama principal de una obra narrativa, como un libro, una serie o una película. Esta descripción se elaboró con el objetivo de presentar o promocionar la obra ante terceros, como productores, editores o evaluadores. A diferencia de la storyline, que detalló la progresión estructural del relato, la sinopsis se enfocó en destacar los elementos más llamativos de la historia para generar interés.

De ahí que fue una herramienta esencial en entornos editoriales y audiovisuales, donde se requirió sintetizar la propuesta narrativa sin revelar completamente su resolución \parencite{studiobinder_sinopsis}.

\subsubsection{Escaleta}

La \textit{escaleta}, también conocida como "step outline", fue un esquema narrativo que presentó, de forma organizada, las escenas o secuencias de una obra audiovisual o literaria. Actuó como un índice de la historia, detallando brevemente lo que ocurrió en cada escena (ubicación, acción principal, cambio de estado) sin incluir diálogos, lo que permitió visualizar la progresión dramática y estructural de la obra \parencite{escaleta_wikipedia,aprendercine_escaleta}.

\subsubsection{Escaleta (función profesional)}

Más allá de ser un simple esquema, la escaleta fue una herramienta profesional esencial para guionistas y equipos de producción. Permitió ordenar sistemáticamente los bloques narrativos, identificar puntos de giro, gestionar localizaciones y distribuir tiempos, y funcionó como guía para la preproducción, ya que facilitó la planificación de la grabación de manera práctica y eficiente \parencite{treintaycinco_escaleta,escueladesarts_escaleta}.

\subsubsection{Tratamiento}

El \textit{tratamiento} fue un relato en prosa de la historia completa, que amplió la sinopsis y escaleta, presentando cada escena como un párrafo narrativo en tiempo presente, incluyendo las acciones clave y el subtexto, pero sin diálogos técnicos. Sirvió tanto para desarrollar la trama en detalle como para ofrecer un documento legible por productores y agentes literarios, con una extensión aproximada de 30 a 40 páginas en el caso de un largometraje \parencite{revista24cuadros_tratamiento,tratamiento_wikipedia}.

\subsubsection{Tratamiento (paso intermedio)}

El tratamiento fue el paso narrativo intermedio entre la escaleta y el guion literario final. Se caracterizó por narrar escena por escena en prosa, incluyendo detalles narrativos y de escena, y sirvió como una herramienta de planificación y visualización narrativa para garantizar coherencia en el guion final \parencite{unir_escaleta_tratamiento,tratamiento_wikipedia}.

\subsubsection{Escena}

Una \textit{escena} fue la unidad narrativa que ocurrió en un mismo lugar y tiempo, constituyendo la estructura básica de un guion literario o técnico. En cada escena, se presentaron acciones, cambios de personajes o desarrollo dramático, representando un segmento coherente e independiente dentro del relato general \parencite{trottier_scene}.

\subsubsection{Tema}

El \textit{tema} representó el mensaje o idea central que atravesó la obra. No solió expresarse de forma explícita en el guion, sino que estuvo contenido en los elementos dramáticos, estructurales y emocionales de la historia. Funcionó como el "ADN" del relato, conectando escenas, personajes y arcos en función de un propósito narrativo mayor \parencite{greenlight_theme,writingninja_theme}.

\subsubsection{Secuencia}

Una \textit{secuencia} fue un bloque narrativo dentro de un acto que agrupó varias escenas con una mini-estructura dramática completa: inicio, desarrollo y resolución parcial. En términos audiovisuales, solieron durar entre ocho y quince minutos y funcionaron como sub-relatos que mantuvieron la atención del espectador y avanzaron la narración general \parencite{gulino_sequence,cambridge_sequence}.

\subsection{Bases Legales}

El marco legal aplicado al uso de software y la creación de guiones literarios en Paraguay abarcó normativas internacionales y locales que protegieron la propiedad intelectual y establecieron derechos sobre obras audiovisuales, estructuras pre-guion literario y programas informáticos.

\subsubsection{Tratados internacionales}

Paraguay fue signatario de los principales convenios internacionales sobre derechos de autor, entre ellos el \textbf{Convenio de Berna} (1886), que estableció la protección automática de obras literarias y artísticas desde su fijación en un medio fijo \parencite{berne2025,rule_shorter_term}.

También se adhirió al \textbf{Tratado de Beijing sobre Actuaciones Audiovisuales} (2012), orientado a proteger derechos de intérpretes y fortalecer la regulación legal de obras audiovisuales \parencite{beijing_treaty2025}.

\subsubsection{Protección de software y guiones}

{\sloppy
	El \textbf{Derecho de Autor} paraguayo reconoció la creación intelectual, incluyendo software y guiones literarios, a través de la Ley 1328/1998, modificada por la Ley 4046/2010. Esta normativa amparó tanto los derechos económicos como morales de los autores, sin exigir registro previo \parencite{wipo_paraguay}.
}

Los programas informáticos estuvieron cubiertos como obras literarias, lo que incluyó tanto su código como su interfaz gráfica \parencite{njq_paraguay,paraguay_software1998}.

\subsubsection{Aplicabilidad a estructuras pre-guion literario}

El Convenio de Berna extendió la protección a «obras literarias», abarcando guiones literarios, \textit{storylines}, escaletas y tratamientos, siempre que se encontraron fijados en un medio tangible, lo que incluyó documentos digitales \parencite{berne2025}.

En cuanto al software especializado para preproducción (por ejemplo, generadores de \textit{storyboards}), la ley cubrió la protección de su código fuente y compilado, aunque los datos o estructuras generadas (ej.\ escaletas) pudieron requerir protección adicional bajo derechos conexos o contratos.

\subsubsection{Gestión colectiva y cumplimiento local}

En Paraguay, los autores pudieron inscribirse en \textbf{Autores Paraguayos Asociados (APA)}, entidad de gestión colectiva creada en 1951 y regulada por la Ley 1328/1998, para administrar sus derechos económicos en el territorio nacional e internacional \parencite{apa_py}.

En paralelo, la aplicación de la legislación local estuvo a cargo de la \textbf{Dirección Nacional de Propiedad Intelectual (DINAPI)}, que reguló infracciones y sanciones ante violaciones del derecho de autor \parencite{ip_office_py}.

\subsubsection{Ley N° 6106 de Fomento al Audiovisual}

La \textbf{Ley N° 6106}, promulgada en 2018, estableció el marco legal específico para el fomento y desarrollo del sector audiovisual en Paraguay \parencite{ley_6106_paraguay}. Esta normativa reconoció explícitamente la centralidad del guionista en el proceso creativo audiovisual, definiéndolo como la persona natural que fue autora del guion que sirvió de base creativa para la realización de la obra audiovisual. Esta definición legal reconoció el guion como elemento fundacional del proceso de producción, estableciendo derechos de autor específicos para quienes desarrollaron estas obras literarias previas a la filmación.

La ley estableció además mecanismos de incentivo y protección para la producción audiovisual nacional, incluyendo el reconocimiento de los derechos morales y patrimoniales de los guionistas sobre sus creaciones. En el contexto del presente estudio, esta normativa respaldó jurídicamente la relevancia de investigar los procesos de sistematización de la pre-escritura, pues reconoció formalmente que el guion constituyó la base creativa protegible de toda obra audiovisual, tanto en su versión literaria final como en sus fases previas de desarrollo narrativo.

\subsubsection{Fountain como Estándar Abierto}

\textit{Fountain} fue un lenguaje de marcado de texto plano y de código abierto diseñado para la creación y edición de guiones en cualquier editor de texto, independientemente del sistema operativo o la plataforma. Fue desarrollado por Stu Maschwitz, John August y Nima Yousefi, entre otros colaboradores, con el objetivo de facilitar la escritura y el intercambio de guiones sin restricciones de formato propietarios \parencite{fountain_wikipedia, fountain_official}.

Desde su lanzamiento, \textit{Fountain} se consolidó como una alternativa viable a los formatos propietarios (como FDX de Final Draft), promoviendo la accesibilidad y portabilidad. Su especificación y múltiples librerías de código abierto estuvieron disponibles bajo licencia MIT, lo que permitió su integración en aplicaciones de edición, conversión y archivado sin preocuparse por la obsolescencia del software \parencite{fountain_official, fountain_github}.

Legalmente, \textit{Fountain} representó un enfoque innovador hacia el software y estándares abiertos, en donde el formato de guion se definió en un documento legible y libremente distribuible. Esto permitió que los guionistas mantuvieron \textit{control y propiedad total} de sus creaciones sin depender de licencias restrictivas ni formatos propietarios, alineándose con las prácticas recomendadas de preservación digital y acceso abierto.

\clearpage
