% ══════════════════════════════════════════════════════════════════
% INTRODUCCIÓN
% ══════════════════════════════════════════════════════════════════

\section*{Introducción}
\addcontentsline{toc}{section}{Introducción}

La creación de guiones audiovisuales es una actividad profundamente creativa que, paradójicamente, aún depende de métodos manuales y herramientas fragmentadas para estructurar sus componentes narrativos. En la práctica, guionistas independientes y pequeñas productoras enfrentan dificultades significativas al organizar elementos esenciales como personajes, locaciones y secuencias dramáticas. A menudo, recurren al uso de pizarras, notas en papel o archivos dispersos que, si bien permiten cierta flexibilidad, interrumpen el flujo de trabajo cuando es necesario transcribir esta información a programas de escritura técnica como Final Draft, Celtx o Fade In, los cuales no siempre ofrecen soporte para procesos previos ni para formatos abiertos como Fountain.

Frente a este contexto surge DREAMINK: una aplicación web de tipo Open Source diseñada para centralizar la estructuración dramática previa a la escritura del guion literario. Esta herramienta busca integrar la creación de fichas narrativas con exportación directa al formato Fountain, lo que permite mantener la coherencia estructural desde la concepción narrativa hasta la redacción final. A diferencia de soluciones privativas y cerradas, DREAMINK adopta un enfoque minimalista, local y mono-usuario, priorizando la accesibilidad, la transparencia del código y la compatibilidad con estándares abiertos, sin depender de la nube ni exigir conexión constante a internet.

El desarrollo de esta herramienta se apoya en principios de la ingeniería de software, diseño de interfaz centrado en el usuario y metodologías de desarrollo ágil. Además, se enfoca especialmente en quienes más lo necesitan: estudiantes, guionistas emergentes, docentes de narrativa audiovisual y pequeños colectivos de creación. Su objetivo es ofrecer un entorno práctico, seguro y extensible que permita a los creadores enfocarse en lo esencial: contar historias con claridad, estructura y libertad técnica.

En suma, DREAMINK no solo representa una solución funcional a un problema técnico específico, sino también una propuesta pedagógica y cultural que apuesta por el software libre, la interoperabilidad y el empoderamiento creativo. Al unir lo narrativo con lo tecnológico, esta herramienta busca cerrar la brecha entre la idea y el guion, facilitando procesos más fluidos, accesibles y profesionalizados para una nueva generación de narradores audiovisuales.

\newpage
