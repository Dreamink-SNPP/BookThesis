% ══════════════════════════════════════════════════════════════════
% INTRODUCCIÓN
% ══════════════════════════════════════════════════════════════════

\clearpage
\thispagestyle{empty}
\vspace*{\fill}
\begin{center}
    {\Large\bfseries Introducción}
\end{center}
\vspace*{\fill}
\clearpage

\section*{Introducción}
\addcontentsline{toc}{section}{Introducción}

La sistematización del proceso de pre-escritura de guiones constituye una práctica fundamental en el desarrollo de proyectos audiovisuales, determinando la coherencia y viabilidad de las obras antes de su redacción literaria definitiva. Este proceso involucra la organización, estructuración y consolidación de elementos narrativos mediante diversas herramientas y metodologías, las cuales varían según las necesidades y contextos de trabajo de cada guionista.

En el capítulo primero se presenta el marco introductorio, donde se establece el planteamiento del problema de investigación, las preguntas que guían el estudio, los objetivos generales y específicos, así como la justificación, limitaciones y alcance del trabajo.

En el capítulo segundo se desarrolla el marco teórico, el cual comprende los antecedentes de la investigación, las bases teóricas que sustentan el estudio y las bases legales que enmarcan el contexto normativo vinculado al objeto de investigación.

En el capítulo tercero se expone el marco metodológico, donde se describe el tipo, diseño y nivel de investigación, el universo del discurso, la población y muestra, los instrumentos de recolección de datos, así como las técnicas de análisis y procedimientos empleados.

En el capítulo cuarto se aborda el marco analítico, presentando el procesamiento de los datos recogidos, el análisis mediante diversas técnicas, y la presentación y discusión de los resultados obtenidos en el estudio.

En el capítulo quinto se exponen los requerimientos del software, incluyendo la visión general del sistema, las definiciones, acrónimos y abreviaturas, las restricciones del proyecto, el presupuesto estimado, así como los diagramas de actividades, casos de uso, clases y entidad-relación que documentan la arquitectura de la solución propuesta.

Finalmente, se presentan las conclusiones derivadas de la investigación y las recomendaciones para futuros estudios, junto con la bibliografía consultada según normativa APA, así como los anexos y apéndices que complementan el trabajo desarrollado.

\newpage
