% ══════════════════════════════════════════════════════════════════
% APÉNDICES (Parte Post-Textual)
% ══════════════════════════════════════════════════════════════════

\section*{Apéndices}
\addcontentsline{toc}{section}{Apéndices}

\subsection*{Apéndice A: Glosario de Términos}
\addcontentsline{toc}{subsection}{Apéndice A: Glosario de Términos}

\begin{description}[style=nextline]

    \item[Acto:] División narrativa de alto nivel que estructura la obra audiovisual en segmentos principales. Tradicionalmente las obras cinematográficas se organizan en tres actos (configuración, confrontación y resolución), aunque pueden emplearse otras configuraciones según la propuesta narrativa.

    \item[Escaleta:] Esquema narrativo que presenta, de forma organizada, las escenas o secuencias de una obra audiovisual. Actúa como un índice de la historia, detallando brevemente lo que ocurre en cada escena (ubicación, acción principal, cambio de estado) sin incluir diálogos, permitiendo visualizar la progresión dramática y estructural de la obra.

    \item[Escena:] Unidad narrativa mínima que ocurre en un tiempo y espacio continuos. Cada escena se caracteriza por su locación, momento del día y descripción de la acción dramática que se desarrolla en ella.

    \item[Fountain:] Lenguaje de marcado de texto plano y de código abierto diseñado para la creación y edición de guiones en cualquier editor de texto. Desarrollado por Stu Maschwitz, John August y Nima Yousefi, facilita la escritura y el intercambio de guiones sin restricciones de formatos propietarios.

    \item[Guion Literario:] Historia desarrollada mediante imágenes, diálogos y descripciones, ubicada dentro de una estructura dramática definida. Se considera la base sobre la que se construye el guion técnico y adaptativo.

    \item[Locación:] Espacio físico donde se desarrolla una escena. Las locaciones se clasifican en interiores o exteriores y pueden aparecer en múltiples escenas a lo largo de la obra.

    \item[Logline:] Resumen extremadamente conciso de la premisa de una obra audiovisual, típicamente expresado en una o dos oraciones. El logline captura la esencia del conflicto central y los protagonistas principales.

    \item[Pre-escritura:] Conjunto de actividades previas a la redacción formal del guion literario, incluyendo la gestión de personajes, locaciones, ideas y estructuras dramáticas. Comprende la planificación y organización de elementos narrativos antes de iniciar la escritura del guion.

    \item[Secuencia:] Bloque narrativo dentro de un acto que agrupa varias escenas con una mini-estructura dramática completa: inicio, desarrollo y resolución parcial. En términos audiovisuales, suelen durar entre ocho y quince minutos y funcionan como sub-relatos que mantienen la atención del espectador.

    \item[Sinopsis:] Resumen breve y estructurado de una obra narrativa que presenta de manera condensada los elementos centrales de la historia (personajes principales, conflicto esencial, tono y progresión dramática). Se utiliza como herramienta de venta, presentación o evaluación de proyectos creativos.

    \item[Sistematización:] Estructuración metodológica de las fases de desarrollo narrativo, especialmente aquellas que anteceden a la redacción del guion literario. Implica la adopción de métodos organizativos que facilitan la materialización de ideas narrativas complejas.

    \item[Storyline:] Secuencia de eventos que constituyen la trama de una obra narrativa. Refleja el contenido central del relato, delimitando cómo ocurren los hechos, en qué orden, y con qué propósito narrativo. Representa el hilo conductor esencial que articula el relato completo.

    \item[Tema:] Mensaje o idea central que atraviesa la obra. No suele expresarse de forma explícita en el guion, sino que está contenido en los elementos dramáticos, estructurales y emocionales de la historia.

    \item[Tratamiento:] Relato en prosa de la historia completa que amplía la sinopsis y escaleta, presentando cada escena como un párrafo narrativo en tiempo presente. Incluye las acciones clave y el subtexto, pero sin diálogos técnicos. Sirve como documento legible por productores y agentes literarios.

\end{description}

\newpage

\subsection*{Apéndice B: Lista de Acrónimos y Abreviaturas}
\addcontentsline{toc}{subsection}{Apéndice B: Lista de Acrónimos y Abreviaturas}

\begin{description}[style=nextline]

    \item[APA:] Autores Paraguayos Asociados. Entidad de gestión colectiva de derechos de autor creada en 1951 y regulada por la Ley 1328/1998.

    \item[API:] \textit{Application Programming Interface} (Interfaz de Programación de Aplicaciones). Conjunto de definiciones y protocolos que permite la comunicación entre diferentes componentes de software.

    \item[CRUD:] \textit{Create, Read, Update, Delete} (Crear, Leer, Actualizar, Eliminar). Representa las cuatro operaciones básicas de gestión de datos en sistemas de información.

    \item[DINAPI:] Dirección Nacional de Propiedad Intelectual. Entidad paraguaya que regula infracciones y sanciones ante violaciones del derecho de autor.

    \item[ERD:] \textit{Entity-Relationship Diagram} (Diagrama de Entidad-Relación). Representación gráfica de la estructura de datos de un sistema.

    \item[ERS:] Especificación de Requisitos de Software. Documento que detalla de manera formal y completa los requisitos funcionales y no funcionales de un sistema de software.

    \item[FDX:] Formato propietario de archivo utilizado por el software Final Draft para guiones cinematográficos.

    \item[IEEE:] \textit{Institute of Electrical and Electronics Engineers}. Organización profesional que desarrolla estándares internacionales para diversas áreas tecnológicas.

    \item[MIT:] \textit{Massachusetts Institute of Technology}. También se refiere a la licencia de software libre desarrollada por dicha institución.

    \item[PDF:] \textit{Portable Document Format} (Formato de Documento Portátil). Formato de archivo desarrollado por Adobe que preserva la apariencia visual de documentos independientemente del software o hardware utilizado.

\end{description}

\newpage

\subsection*{Apéndice C: Matrices Metodológicas}
\addcontentsline{toc}{subsection}{Apéndice C: Matrices Metodológicas}

\subsubsection*{Matriz de Consistencia}

La matriz de consistencia del presente estudio establece la coherencia entre los elementos fundamentales de la investigación, garantizando la alineación lógica entre el problema identificado, el objetivo planteado y la hipótesis descriptiva del estudio.

\textbf{Problema General:} ¿Cuáles son los métodos de sistematización del proceso de pre-escritura de guiones utilizados por los guionistas del Departamento Central durante el año 2025?

\textbf{Objetivo General:} Describir los métodos de sistematización del proceso de pre-escritura de guiones utilizados por los guionistas del Departamento Central durante el año 2025.

\textbf{Variable de estudio:} Sistematización del proceso de pre-escritura de guiones

\textbf{Objetivos Específicos:}

\begin{enumerate}
    \item Identificar las formas de organización de elementos narrativos (personajes, locaciones e ideas) empleadas por los guionistas durante la fase de pre-escritura.

    \item Caracterizar las herramientas y métodos utilizados por los guionistas para estructurar dramáticamente sus obras antes del guion literario.

    \item Describir las principales dificultades que enfrentan los guionistas en la sistematización del proceso de pre-escritura.
\end{enumerate}

\textbf{Nota metodológica:} El estudio emplea un enfoque cualitativo descriptivo con diseño observacional, aplicado a una muestra de tres guionistas del Departamento Central seleccionados mediante muestreo aleatorio simple. El instrumento de recolección de datos consistió en una guía de entrevista semiestructurada diseñada para explorar las cuatro dimensiones de la variable: (1) organización de elementos narrativos, (2) estructuración dramática, (3) métodos y herramientas de sistematización, y (4) desafíos y dificultades en la sistematización.

\newpage

\subsubsection*{Matriz de Operacionalización de Variables}

La operacionalización de la variable principal del estudio se realizó mediante el diseño de una guía de entrevista semiestructurada que permitió explorar cuatro dimensiones fundamentales:

\textbf{Dimensión 1: Organización de elementos narrativos}

\textit{Indicadores:}
\begin{itemize}
    \item Gestión de personajes
    \item Gestión de locaciones
    \item Gestión de ideas y conceptos
    \item Métodos de registro y documentación
\end{itemize}

\textbf{Dimensión 2: Estructuración dramática}

\textit{Indicadores:}
\begin{itemize}
    \item Organización en actos
    \item Organización en secuencias
    \item Organización en escenas
    \item Paradigmas estructurales utilizados
\end{itemize}

\textbf{Dimensión 3: Métodos y herramientas de sistematización}

\textit{Indicadores:}
\begin{itemize}
    \item Herramientas digitales utilizadas
    \item Métodos análogos (papel, tarjetas, etc.)
    \item Documentos de pre-escritura (logline, sinopsis, escaleta, tratamiento)
    \item Combinaciones de métodos
\end{itemize}

\clearpage

\textbf{Dimensión 4: Desafíos y dificultades en la sistematización}

\textit{Indicadores:}
\begin{itemize}
    \item Dificultades técnicas
    \item Dificultades metodológicas
    \item Limitaciones de herramientas
    \item Necesidades no cubiertas
\end{itemize}

\textbf{Fundamentación teórica:} Las dimensiones propuestas se fundamentan en la literatura académica sobre guionismo y procesos creativos: Field (2005) y McKee (1997) para la organización de elementos narrativos; Field (1979) y Gulino (2004) para la estructuración dramática; literatura contemporánea sobre procesos creativos profesionales (Industrial Scripts, ScreenCraft) para métodos y herramientas; y Shore Scripts (2024) sobre desafíos de guionistas en organización y planificación.

\textbf{Técnica de análisis:} Los datos obtenidos mediante la entrevista fueron analizados mediante análisis de contenido cualitativo, identificando categorías emergentes, patrones comunes entre los guionistas participantes, particularidades en los métodos empleados, y temas recurrentes sobre dificultades y necesidades en la sistematización del proceso de pre-escritura.

\newpage
