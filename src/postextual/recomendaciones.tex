% ══════════════════════════════════════════════════════════════════
% RECOMENDACIONES PARA FUTURAS INVESTIGACIONES (Parte Post-Textual)
% ══════════════════════════════════════════════════════════════════

\section*{Recomendación para futuras investigaciones}
\addcontentsline{toc}{section}{Recomendación para futuras investigaciones}

Con base en los hallazgos y conclusiones del presente estudio, se formulan las siguientes recomendaciones dirigidas a investigadores, desarrolladores de herramientas tecnológicas, instituciones educativas y profesionales del sector audiovisual paraguayo.

\textbf{1. Recomendaciones metodológicas para futuras investigaciones}

Se recomienda ampliar el alcance del estudio mediante la realización de investigaciones con muestras más extensas que incluyan guionistas de otros departamentos de Paraguay, permitiendo identificar particularidades regionales en los métodos de sistematización y validar la universalidad de los patrones identificados en el presente estudio. Asimismo, se sugiere realizar estudios comparativos entre guionistas paraguayos y guionistas de otros países latinoamericanos para identificar convergencias y divergencias en las prácticas de pre-escritura, aportando perspectiva regional sobre sistematización del proceso creativo.

Se recomienda complementar el enfoque cualitativo empleado en el presente estudio con diseños metodológicos mixtos que incorporen análisis cuantitativos mediante encuestas estructuradas aplicadas a muestras representativas del sector, permitiendo cuantificar la frecuencia de uso de herramientas específicas y el nivel de satisfacción con soluciones tecnológicas existentes. Adicionalmente, se sugiere la implementación de estudios longitudinales que documenten la evolución de los métodos de sistematización de los guionistas a lo largo del tiempo, identificando cómo la experiencia profesional modifica las prácticas de pre-escritura.

Dado que el estudio identificó hallazgos emergentes relacionados con dimensiones cognitivas del proceso creativo (fundamentación cognitiva de la preferencia híbrida análogo-digital, importancia de la dimensión háptica en la visualización estructural), se recomienda la realización de investigaciones interdisciplinarias que incorporen perspectivas de ciencias cognitivas, psicología de la creatividad y neurociencia aplicada para profundizar en los fundamentos científicos de las preferencias metodológicas identificadas.

\textbf{2. Recomendaciones para el desarrollo de DREAMINK}

Se recomienda implementar un proceso de validación iterativa del sistema DREAMINK con guionistas profesionales mediante metodología de diseño centrado en el usuario, realizando sesiones de prueba de usabilidad que permitan ajustar la interfaz y funcionalidades según retroalimentación directa de usuarios potenciales. Los tres participantes del presente estudio manifestaron disponibilidad para colaborar como testers beta del sistema.

Se sugiere priorizar en las primeras versiones del sistema las tres funcionalidades críticas identificadas como necesidades universales: visualización integrada tipo Kanban sin scroll, unificación de elementos en ecosistema único, y vinculación dinámica contextual entre elementos narrativos. La implementación gradual de funcionalidades secundarias permitirá mantener la simplicidad y evitar la complejidad excesiva que llevó al rechazo de Scrivener.

Se recomienda desarrollar mecanismos de importación y exportación que respeten el enfoque híbrido análogo-digital identificado como preferencia universal, permitiendo la digitalización de contenido creado análogamente (escaneo de cuadernos, fotos de post-its) y la exportación de documentos en formatos estándar de la industria (PDF, Final Draft, Fountain). Asimismo, se sugiere implementar campos personalizables en lugar de plantillas rígidas, respetando el rechazo universal a estandarización que inhibe creatividad.

Se recomienda considerar la incorporación de funcionalidades colaborativas que faciliten el trabajo conjunto entre guionistas, directores y productores, respondiendo a las necesidades de coordinación en proyectos audiovisuales que involucran múltiples profesionales. La implementación de control de versiones y historial de cambios permitirá documentar la evolución del proceso creativo.

\clearpage

\textbf{3. Recomendaciones para el sector audiovisual paraguayo}

Se recomienda a instituciones educativas que imparten formación en guionismo (universidades, centros de formación audiovisual, talleres especializados) incorporar en sus programas académicos contenidos específicos sobre sistematización del proceso de pre-escritura, reconociendo que esta fase constituye fundamento metodológico esencial para la escritura profesional de guiones. La enseñanza de herramientas y métodos organizativos debe complementar la formación teórica sobre paradigmas estructurales.

Se sugiere a asociaciones profesionales del sector audiovisual (asociaciones de guionistas, gremios de realizadores) organizar talleres y espacios de intercambio de experiencias donde guionistas profesionales compartan sus métodos de sistematización, facilitando la socialización de buenas prácticas y la construcción colectiva de conocimiento metodológico local. El presente estudio evidenció que los guionistas desarrollan sistemas propios de manera aislada; la creación de espacios de intercambio permitiría optimizar estos procesos mediante aprendizaje colaborativo.

Se recomienda a instituciones de fomento audiovisual (fondos concursables, programas de apoyo a la producción) considerar en sus criterios de evaluación de proyectos la calidad de la sistematización del proceso de pre-escritura, reconociendo que una pre-escritura sólida y metodológica constituye indicador de viabilidad y profesionalización del proyecto. La exigencia de documentos estructurados (logline, sinopsis, escaleta) como requisitos de postulación fomenta la adopción de prácticas sistemáticas.

\textbf{4. Líneas de investigación futuras}

Se sugiere investigar específicamente la dimensión háptica del proceso creativo identificada como hallazgo emergente, explorando mediante estudios experimentales si la manipulación física de elementos (post-its, tarjetas) genera diferencias cognitivas significativas respecto a manipulación digital, aportando evidencia científica que fundamente el diseño de interfaces que simulen efectivamente la experiencia táctil.

Se recomienda realizar estudios sobre las diferencias generacionales en los métodos de sistematización, comparando guionistas formados pre-digitalmente con guionistas nativos digitales, identificando si existen patrones diferenciados en la adopción de herramientas tecnológicas y en las preferencias metodológicas según la edad y contexto de formación.

Se sugiere investigar la aplicabilidad de los hallazgos del presente estudio a otros contextos creativos que involucran pre-escritura estructurada (escritura de novelas, diseño de videojuegos narrativos, creación de podcasts de ficción), explorando si las necesidades de visualización integrada, unificación de información y vinculación dinámica constituyen requerimientos transversales a múltiples disciplinas narrativas.

Finalmente, se recomienda realizar estudios de impacto post-implementación de DREAMINK, evaluando mediante diseño cuasi-experimental si el uso del sistema genera mejoras cuantificables en la eficiencia del proceso de pre-escritura (reducción de tiempo, disminución de reescrituras estructurales posteriores) y en la calidad percibida de los guiones producidos, aportando evidencia empírica del valor agregado de la herramienta.

\newpage
