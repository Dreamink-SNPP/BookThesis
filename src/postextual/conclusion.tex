% ══════════════════════════════════════════════════════════════════
% CONCLUSIÓN (Parte Post-Textual)
% ══════════════════════════════════════════════════════════════════

\section*{Conclusión}
\addcontentsline{toc}{section}{Conclusión}

El presente estudio cumplió con el objetivo general de describir los métodos de sistematización del proceso de pre-escritura de guiones utilizados por los guionistas del Departamento Central durante el año 2025, mediante el análisis cualitativo de tres entrevistas semiestructuradas aplicadas a guionistas profesionales con experiencia de 2 a 10 años en el sector audiovisual paraguayo.

En relación al Objetivo Específico 1, se identificó que los guionistas emplean formas de organización de elementos narrativos caracterizadas por el desarrollo de sistemas propios personalizados que priorizan aspectos psicológicos sobre descripciones físicas. El rechazo universal a fichas técnicas tradicionales estandarizadas constituyó hallazgo significativo, evidenciándose que los participantes construyeron métodos adaptativos fundamentados en flexibilidad antes que en estandarización. El uso intensivo del celular para captura espontánea de ideas y la utilización de referencias visuales (fotografías, Pinterest, Google Maps) para personajes y locaciones emergieron como prácticas consistentes entre los tres participantes.

Respecto al Objetivo Específico 2, se caracterizaron las herramientas y métodos utilizados para estructuración dramática, identificándose patrones altamente consistentes: el paradigma de tres actos como base estructural universal, un proceso bifásico de exploración creativa seguida de estructuración rigurosa, y la secuencia metodológica Logline → Sinopsis → Escaleta como fundamento documental. La convergencia total en el uso de post-its o tarjetas físicas en pared para visualización estructural, fundamentada cognitivamente en la necesidad de manipulación táctil y visión panorámica, constituyó hallazgo relevante. La preferencia híbrida análogo-digital fue unánime, justificada por los participantes en que el método análogo favorece creatividad al inhibir edición prematura que ``mata la creatividad''.

En cuanto al Objetivo Específico 3, se describieron tres dificultades principales identificadas universalmente por los tres participantes: falta de visualización integrada de elementos, dispersión de información en múltiples herramientas desconectadas, y ausencia de vinculación dinámica entre elementos narrativos y guion. Estas dificultades no constituyen inconvenientes menores sino carencias fundamentales que afectan la eficacia del proceso creativo, proporcionando base empírica sólida para la justificación de DREAMINK como solución tecnológica integrada.

Los hallazgos validaron empíricamente la vigencia de los paradigmas teóricos de Field (2005), McKee (1997) y Koestler (1964) en el contexto local, evidenciando que los guionistas paraguayos emplean métodos de sistematización consistentes y profesionales comparables a estándares internacionales. Sin embargo, se identificó una brecha significativa entre sistematización metodológica (los guionistas sistematizaron \textit{qué} hacer) y sistematización operativa (carecen de sistemas integrados para \textit{cómo} gestionar eficientemente los elementos durante el proceso).

El estudio proporcionó evidencia empírica que justificó el desarrollo de DREAMINK como herramienta de sistematización diseñada específicamente para resolver las tres necesidades críticas identificadas: visualización integrada mediante interfaz tipo Kanban, unificación de información en ecosistema único con flexibilidad, y vinculación dinámica contextual entre elementos narrativos. El diseño de DREAMINK se fundamentó en necesidades reales articuladas por profesionales locales antes que en supuestos importados, estableciendo precedente metodológico de investigación cualitativa como base para desarrollo tecnológico.

Finalmente, el estudio contribuyó a visibilizar la profesionalización del sector audiovisual paraguayo, contradiciendo posibles prejuicios sobre informalidad. Los participantes demostraron consciencia metodológica, conocimiento teórico sólido y desarrollo proactivo de soluciones organizativas propias, evidenciando que la sistematización ya existe en la práctica profesional local y requiere infraestructura tecnológica que la soporte sin alterar sus fundamentos creativos.

\newpage
