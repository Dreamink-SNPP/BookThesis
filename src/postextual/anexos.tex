% ══════════════════════════════════════════════════════════════════
% ANEXOS (Parte Post-Textual)
% ══════════════════════════════════════════════════════════════════

\section*{Anexos}
\addcontentsline{toc}{section}{Anexos}

\subsection*{Anexo A: Guía de Entrevista Semiestructurada}
\addcontentsline{toc}{subsection}{Anexo A: Guía de Entrevista Semiestructurada}

\subsubsection*{Información General del Instrumento}

\textbf{Título del estudio:} Sistematización del proceso de pre-escritura de guiones en los guionistas del Departamento Central. 2025.

\textbf{Objetivo del instrumento:} Recolectar información cualitativa sobre los métodos, herramientas, prácticas organizativas y dificultades que enfrentan los guionistas durante la fase de pre-escritura de guiones.

\textbf{Tipo de instrumento:} Entrevista semiestructurada con preguntas abiertas y técnicas de sondeo (\textit{probes}).

\textbf{Duración estimada:} 45-60 minutos

\textbf{Población objetivo:} Guionistas cinematográficos activos del Departamento Central, República del Paraguay.

\subsubsection*{Estructura de la Entrevista}

La guía de entrevista se organizó en siete secciones temáticas diseñadas para explorar sistemáticamente las cuatro dimensiones de la variable de estudio:

\textbf{Sección A: Datos Generales (2-3 minutos)}

\textit{Objetivo:} Contextualizar el perfil profesional del entrevistado.

\begin{enumerate}
    \item Nombre completo (opcional, puede usarse seudónimo)
    \item Años de experiencia como guionista
    \item Tipos de proyectos que escribe (ficción, documental, series, cortometrajes, largometrajes, etc.)
    \item Formación en guionismo (autodidacta, talleres, cursos, licenciatura)
\end{enumerate}

\textbf{Sección B: Contexto Profesional (5 minutos)}

\textit{Objetivo:} Establecer rapport y comprender la trayectoria del guionista.

\begin{enumerate}[resume]
    \item ¿Puede contarme brevemente sobre su trayectoria como guionista?
    \begin{itemize}
        \item \textit{Probe:} ¿Cómo comenzó a escribir guiones?
    \end{itemize}
    \item ¿Cuántos proyectos de guion ha desarrollado aproximadamente?
    \begin{itemize}
        \item \textit{Probe:} ¿Cuántos se han producido o están en producción?
    \end{itemize}
    \item ¿Actualmente está trabajando en algún proyecto de guion?
    \begin{itemize}
        \item \textit{Probe:} ¿En qué fase se encuentra ese proyecto?
    \end{itemize}
\end{enumerate}

\textbf{Sección C: Organización de Elementos Narrativos (10-12 minutos)}

\textit{Dimensión 1 --- Responde al Objetivo Específico 1}

\textit{Indicadores:} Gestión de personajes, gestión de locaciones, gestión de ideas/conceptos, métodos de registro y documentación.

\begin{enumerate}[resume]
    \item ¿Cómo organiza los personajes de sus proyectos durante la pre-escritura?
    \begin{itemize}
        \item \textit{Probe:} ¿Qué información registra sobre cada personaje antes de escribir el guion?
        \item \textit{Probe:} ¿Crea fichas de personajes, biografías, perfiles psicológicos?
    \end{itemize}
    \item ¿Cómo gestiona las locaciones de su obra?
    \begin{itemize}
        \item \textit{Probe:} ¿Documenta características específicas de cada locación?
        \item \textit{Probe:} ¿Relaciona las locaciones con los personajes o escenas específicas?
    \end{itemize}
    \item ¿De qué manera organiza y documenta las ideas iniciales de sus proyectos?
    \begin{itemize}
        \item \textit{Probe:} ¿Lleva algún tipo de diario creativo, cuaderno de notas?
        \item \textit{Probe:} ¿Cómo captura ideas que surgen espontáneamente?
    \end{itemize}
    \item ¿Utiliza algún formato o plantilla específica para organizar estos elementos?
    \begin{itemize}
        \item \textit{Probe:} ¿Son formatos propios o utiliza plantillas estándar?
        \item \textit{Probe:} ¿Ha desarrollado su propio sistema de organización?
    \end{itemize}
\end{enumerate}

\textbf{Sección D: Estructuración Dramática (10 minutos)}

\textit{Dimensión 2 --- Responde al Objetivo Específico 2 (parte 1)}

\textit{Indicadores:} Organización en actos, organización en secuencias, organización en escenas, paradigmas estructurales utilizados.

\begin{enumerate}[resume]
    \item ¿Cómo estructura dramáticamente sus historias antes de escribir el guion literario?
    \begin{itemize}
        \item \textit{Probe:} ¿Empieza por la estructura o la historia surge primero?
    \end{itemize}
    \item ¿Utiliza algún paradigma estructural específico? (Ejemplos: 3 actos, 5 actos, secuencias, viaje del héroe, save the cat, etc.)
    \begin{itemize}
        \item \textit{Probe:} ¿Siempre usa el mismo o varía según el proyecto?
    \end{itemize}
    \item ¿En qué momento del proceso define la estructura de actos/secuencias/escenas?
    \begin{itemize}
        \item \textit{Probe:} ¿Lo hace al inicio, durante el desarrollo, o es iterativo?
    \end{itemize}
    \item ¿Cómo visualiza o representa la estructura dramática durante la pre-escritura?
    \begin{itemize}
        \item \textit{Probe:} ¿Usa diagramas, escaletas, tarjetas, mapas mentales?
        \item \textit{Probe:} ¿Prefiere métodos visuales o textuales?
    \end{itemize}
\end{enumerate}

\textbf{Sección E: Métodos y Herramientas (10-12 minutos)}

\textit{Dimensión 3 --- Responde al Objetivo Específico 2 (parte 2)}

\textit{Indicadores:} Herramientas digitales utilizadas, métodos análogos, documentos de pre-escritura, combinaciones de métodos.

\begin{enumerate}[resume]
    \item ¿Qué herramientas (digitales o análogas) utiliza durante la fase de pre-escritura?
    \begin{itemize}
        \item \textit{Probe:} ¿Software específico? (Final Draft, Celtx, Scrivener, Notion, Google Docs, etc.)
        \item \textit{Probe:} ¿Papel, tarjetas, cuadernos, pizarras?
    \end{itemize}
    \item ¿Elabora documentos como logline, sinopsis, escaleta o tratamiento?
    \begin{itemize}
        \item \textit{Probe:} ¿En qué orden los desarrolla?
        \item \textit{Probe:} ¿Considera alguno más importante que otros?
    \end{itemize}
    \item ¿Prefiere métodos digitales, análogos o una combinación de ambos? ¿Por qué?
    \begin{itemize}
        \item \textit{Probe:} ¿Ha cambiado su preferencia con el tiempo?
        \item \textit{Probe:} ¿Qué ventajas/desventajas encuentra en cada uno?
    \end{itemize}
    \item ¿Ha utilizado software específico para organizar su pre-escritura? ¿Cuáles?
    \begin{itemize}
        \item \textit{Probe:} ¿Cómo ha sido su experiencia?
        \item \textit{Probe:} ¿Por qué continúa o dejó de usar esas herramientas?
    \end{itemize}
\end{enumerate}

\textbf{Sección F: Desafíos y Dificultades (8-10 minutos)}

\textit{Dimensión 4 --- Responde al Objetivo Específico 3}

\textit{Indicadores:} Dificultades técnicas, dificultades metodológicas, limitaciones de herramientas, necesidades no cubiertas.

\begin{enumerate}[resume]
    \item ¿Cuáles son las principales dificultades que enfrenta al organizar su pre-escritura?
    \begin{itemize}
        \item \textit{Probe:} ¿Dificultades técnicas, metodológicas, de tiempo, de claridad?
    \end{itemize}
    \item ¿Qué aspectos del proceso considera más desafiantes o caóticos?
    \begin{itemize}
        \item \textit{Probe:} ¿Hay momentos donde siente que pierde el control de la organización?
    \end{itemize}
    \item ¿Las herramientas actuales satisfacen sus necesidades de organización? ¿Por qué?
    \begin{itemize}
        \item \textit{Probe:} ¿Qué funciona bien? ¿Qué no funciona?
    \end{itemize}
    \item ¿Qué funcionalidad o característica le gustaría tener en una herramienta de pre-escritura?
    \begin{itemize}
        \item \textit{Probe:} ¿Algo que haya imaginado pero que no existe aún?
    \end{itemize}
\end{enumerate}

\textbf{Sección G: Cierre (3-5 minutos)}

\textit{Objetivo:} Capturar reflexiones finales y perspectiva global.

\begin{enumerate}[resume]
    \item ¿Hay algo más sobre su proceso de pre-escritura que considere importante mencionar y que no hayamos abordado?
    \item Si tuviera que describir su método personal de sistematización en pocas palabras, ¿cómo lo describiría?
    \begin{itemize}
        \item \textit{Probe:} ¿Diría que es estructurado, flexible, caótico, intuitivo, sistemático?
    \end{itemize}
    \item ¿Le gustaría recibir los resultados de este estudio una vez finalizado?
\end{enumerate}

\clearpage

\subsubsection*{Consideraciones Éticas y Metodológicas}

\begin{itemize}
    \item Se solicitó consentimiento informado a todos los participantes antes de iniciar la entrevista.
    \item Se garantizó la confidencialidad de la identidad de los entrevistados mediante códigos (E1, E2, E3).
    \item Se permitió el uso de seudónimos para aquellos participantes que lo solicitaran.
    \item Las entrevistas fueron grabadas en audio previo consentimiento explícito de los participantes.
    \item Se realizaron transcripciones literales completas de cada entrevista para el análisis de contenido cualitativo.
    \item Se registraron notas de observación sobre lenguaje corporal, tono de voz y emociones durante las entrevistas presenciales.
\end{itemize}

\clearpage

\subsubsection*{Validación del Instrumento}

El instrumento fue validado mediante:

\begin{enumerate}
    \item \textbf{Validación de contenido:} Las preguntas fueron diseñadas en correspondencia directa con las cuatro dimensiones de la matriz de operacionalización de variables, garantizando cobertura completa de la variable de estudio.
    \item \textbf{Fundamentación teórica:} Cada sección se fundamentó en la literatura académica sobre guionismo (Field, McKee, Gulino, Koestler) y en investigaciones contemporáneas sobre procesos creativos.
    \item \textbf{Coherencia metodológica:} La guía de entrevista se alineó directamente con el enfoque cualitativo descriptivo del estudio y con los objetivos específicos planteados.
\end{enumerate}

\newpage
