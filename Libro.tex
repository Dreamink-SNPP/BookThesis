\documentclass[12pt]{article}
\usepackage[margin=1in]{geometry}
\usepackage{fontspec}
\usepackage[spanish]{babel}
\usepackage{titlesec}
\usepackage{titletoc}
\usepackage[hidelinks]{hyperref}

% Configuración del tipo de letra para LuaTeX:
\setmainfont{Times New Roman}

% Configuración para evitar encabezados y pies de página:
\pagestyle{empty}

% Estilos de secciones
\titleformat{\section}[block]
{\normalfont\Large\bfseries\centering}{\thesection}{1em}{}

% Configuración de Índice
\contentsmargin{0cm}
\titlecontents{section}[0em]
{\vspace{0.5em}\bfseries}
{\contentslabel{2em}}
{}
{\titlerule*[0.5pc]{.}\contentspage}

\begin{document}

	\begin{center}
		\vspace*{5cm} % Espacio superior para centrar el contenido.

		\textbf{\large TÉCNICO SUPERIOR EN PROGRAMACIÓN DE APLICACIONES INFORMÁTICA}\\[1em]

		\textbf{\large SISTEMA DE ESTRUCTURACIÓN DRAMÁTICA DE OBRAS AUDIOVISUALES}\\[2em]

		\textbf{TUTOR:} Moisés Ávalos\\[2em]

		\textbf{INTEGRANTES:}\\
		- Alberto Álvarez\\
		- Fernando Cardozo

		\vspace*{5cm} % Espacio inferior para mantener el equilibrio.
	\end{center}

	\clearpage % Pasa a la siguiente página

	% Tabla en el medio de la página
	\begin{center}
		\vspace*{5cm} % Espacio superior para centrar la tabla

		\begin{tabular}{|p{15cm}|}
			\hline
			\begin{minipage}[t]{15cm}
				\centering
				\textbf{SISTEMA DE ESTRUCTURACIÓN DRAMÁTICA DE OBRAS AUDIOVISUALES}\\[1em]

				\textit{Autores:} Fernando Cardozo, Alberto Álvarez\\[1em]

				\textit{Fecha de aprobación del Anteproyecto:} \underline{\hspace{2cm}/\hspace{2cm}/\hspace{2cm}}\\[1em]

				\textit{Tutor:} \underline{\hspace{10cm}}\\[1em]

				\raggedleft
				\textit{Visto Bueno del Tutor:} \underline{\hspace{5cm}}\\[1em]
			\end{minipage} \\
			\hline
			\begin{minipage}[t]{15cm}
				\vspace{8em} % Espacio grande para observaciones
			\end{minipage} \\
			\hline
		\end{tabular}

		\vspace*{5cm} % Espacio inferior para mantener el equilibrio
	\end{center}

	\newpage

	% Justificación
	\section*{Justificación}
	\addcontentsline{toc}{section}{Justificación}
	La creación de guiones audiovisuales es un proceso complejo que exige una organización rigurosa de elementos narrativos como personajes, locaciones y estructuras dramáticas. Sin embargo, la mayoría de los guionistas, especialmente aquellos que trabajan de manera independiente o en equipos pequeños, carecen de herramientas accesibles y adaptadas a sus necesidades reales. Los softwares existentes suelen ser costosos, poco flexibles o no permiten integrar de forma eficiente la conceptualización previa con la escritura técnica en formatos estándar como Fountain, lo que genera una brecha significativa en el flujo de trabajo creativo y técnico.\\

	En este contexto, el desarrollo de una aplicación web Open Source como DREAMINK representa una respuesta necesaria. DREAMINK busca ser una nueva herramienta en el ecosistema de herramientas para escritores audiovisuales, permitiendo la creación y gestión de fichas detalladas de personajes y locaciones, así como su exportación directa al formato Fountain. Esta funcionalidad no solo facilita la transición entre la etapa de diseño narrativo y la redacción del guion, sino que también promueve la adopción de estándares abiertos, evitando la dependencia de plataformas propietarias y costosas.\\
	El carácter Open Source de DREAMINK es un aspecto fundamental de su justificación, ya que democratiza el acceso a tecnología profesional y fomenta la personalización y mejora continua por parte de la comunidad usuaria. Esto es especialmente relevante en contextos donde los recursos económicos y técnicos son limitados, permitiendo que más creadores puedan acceder a herramientas de calidad sin barreras de entrada. Además, la flexibilidad del software abierto posibilita su adaptación a distintos géneros, formatos y necesidades particulares de cada proyecto audiovisual.\\
	Por otro lado, DREAMINK contribuye a optimizar el proceso creativo al reducir la fragmentación entre la conceptualización y escritura, minimizando errores y pérdidas de información. Al centralizar la información relevante de personajes y locaciones, la herramienta favorece una visión integral del guion y facilita la toma de decisiones narrativas, lo que puede traducirse en productos audiovisuales de mayor coherencia y calidad. Esto impacta positivamente tanto en la eficiencia del trabajo individual como en la colaboración dentro de equipos pequeños.\\
	Finalmente, el desarrollo de DREAMINK responde a una necesidad real y concreta en los equipos de trabajo independientes en el área audiovisual contemporánea, donde la agilidad, interoperabilidad y la accesibilidad tecnológica son factores clave para la innovación y competitividad. Al ofrecer una solución específica, flexible y alineada con los estándares actuales, este proyecto no solo atiende una problemática técnica, sino que también impulsa el crecimiento y la profesionalización de los guionistas y creadores audiovisuales en entornos diversos.\\

	\newpage

\end{document}