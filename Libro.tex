\documentclass[12pt]{article}
\usepackage[margin=1in]{geometry}
\usepackage{fontspec}
\usepackage[spanish]{babel}
\usepackage{titlesec}
\usepackage{titletoc}
\usepackage[hidelinks]{hyperref}

% Configuración del tipo de letra para LuaTeX:
\setmainfont{Times New Roman}

% Configuración para evitar encabezados y pies de página:
\pagestyle{empty}

% Estilos de secciones
\titleformat{\section}[block]
{\normalfont\Large\bfseries\centering}{\thesection}{1em}{}

% Configuración de Índice
\contentsmargin{0cm}
\titlecontents{section}[0em]
{\vspace{0.5em}\bfseries}
{\contentslabel{2em}}
{}
{\titlerule*[0.5pc]{.}\contentspage}

\begin{document}

	\begin{center}
		\vspace*{5cm} % Espacio superior para centrar el contenido.

		\textbf{\large TÉCNICO SUPERIOR EN PROGRAMACIÓN DE APLICACIONES INFORMÁTICA}\\[1em]

		\textbf{\large SISTEMA DE ESTRUCTURACIÓN DRAMÁTICA DE OBRAS AUDIOVISUALES}\\[2em]

		\textbf{TUTOR:} Moisés Ávalos\\[2em]

		\textbf{INTEGRANTES:}\\
		- Alberto Álvarez\\
		- Fernando Cardozo

		\vspace*{5cm} % Espacio inferior para mantener el equilibrio.
	\end{center}

	\clearpage % Pasa a la siguiente página

	% Tabla en el medio de la página
	\begin{center}
		\vspace*{5cm} % Espacio superior para centrar la tabla

		\begin{tabular}{|p{15cm}|}
			\hline
			\begin{minipage}[t]{15cm}
				\centering
				\textbf{SISTEMA DE ESTRUCTURACIÓN DRAMÁTICA DE OBRAS AUDIOVISUALES}\\[1em]

				\textit{Autores:} Fernando Cardozo, Alberto Álvarez\\[1em]

				\textit{Fecha de aprobación del Anteproyecto:} \underline{\hspace{2cm}/\hspace{2cm}/\hspace{2cm}}\\[1em]

				\textit{Tutor:} \underline{\hspace{10cm}}\\[1em]

				\raggedleft
				\textit{Visto Bueno del Tutor:} \underline{\hspace{5cm}}\\[1em]
			\end{minipage} \\
			\hline
			\begin{minipage}[t]{15cm}
				\vspace{8em} % Espacio grande para observaciones
			\end{minipage} \\
			\hline
		\end{tabular}

		\vspace*{5cm} % Espacio inferior para mantener el equilibrio
	\end{center}

	\newpage

	% Justificación
	\section*{Justificación}
	\addcontentsline{toc}{section}{Justificación}
	La creación de guiones audiovisuales es un proceso complejo que exige una organización rigurosa de elementos narrativos como personajes, locaciones y estructuras dramáticas. Sin embargo, la mayoría de los guionistas, especialmente aquellos que trabajan de manera independiente o en equipos pequeños, carecen de herramientas accesibles y adaptadas a sus necesidades reales. Los softwares existentes suelen ser costosos, poco flexibles o no permiten integrar de forma eficiente la conceptualización previa con la escritura técnica en formatos estándar como Fountain, lo que genera una brecha significativa en el flujo de trabajo creativo y técnico.\\

	En este contexto, el desarrollo de una aplicación web Open Source como DREAMINK representa una respuesta necesaria. DREAMINK busca ser una nueva herramienta en el ecosistema de herramientas para escritores audiovisuales, permitiendo la creación y gestión de fichas detalladas de personajes y locaciones, así como su exportación directa al formato Fountain. Esta funcionalidad no solo facilita la transición entre la etapa de diseño narrativo y la redacción del guion, sino que también promueve la adopción de estándares abiertos, evitando la dependencia de plataformas propietarias y costosas.\\
	El carácter Open Source de DREAMINK es un aspecto fundamental de su justificación, ya que democratiza el acceso a tecnología profesional y fomenta la personalización y mejora continua por parte de la comunidad usuaria. Esto es especialmente relevante en contextos donde los recursos económicos y técnicos son limitados, permitiendo que más creadores puedan acceder a herramientas de calidad sin barreras de entrada. Además, la flexibilidad del software abierto posibilita su adaptación a distintos géneros, formatos y necesidades particulares de cada proyecto audiovisual.\\
	Por otro lado, DREAMINK contribuye a optimizar el proceso creativo al reducir la fragmentación entre la conceptualización y escritura, minimizando errores y pérdidas de información. Al centralizar la información relevante de personajes y locaciones, la herramienta favorece una visión integral del guion y facilita la toma de decisiones narrativas, lo que puede traducirse en productos audiovisuales de mayor coherencia y calidad. Esto impacta positivamente tanto en la eficiencia del trabajo individual como en la colaboración dentro de equipos pequeños.\\

	Finalmente, el desarrollo de DREAMINK responde a una necesidad real y concreta en los equipos de trabajo independientes en el área audiovisual contemporánea, donde la agilidad, interoperabilidad y la accesibilidad tecnológica son factores clave para la innovación y competitividad. Al ofrecer una solución específica, flexible y alineada con los estándares actuales, este proyecto no solo atiende una problemática técnica, sino que también impulsa el crecimiento y la profesionalización de los guionistas y creadores audiovisuales en entornos diversos.\\

	\newpage

	% Planteamiento del Problema
	\section*{Planteamiento del Problema}
	\addcontentsline{toc}{section}{Planteamiento}

	El proceso de creación de guiones audiovisuales presenta actualmente importantes dificultades en la organización y estructuración de los elementos narrativos esenciales, como personajes y locaciones. Muchos guionistas, especialmente aquellos que trabajan de forma independiente o en equipos pequeños, recurren a métodos manuales -utilizando papel, pizarras y notas sueltas- para conceptualizar y desarrollar sus fichas e historias antes de trasladarlas a un software de guion. Esta práctica, además de ser ineficiente, interrumpe el flujo creativo y genera una duplicación de esfuerzos, ya que la información debe ser transcrita posteriormente a programas especializados que, en la mayoría de los casos, no ofrecen herramientas específicas para la gestión previa de fichas narrativas. \\

	A pesar de la existencia de diversos softwares de escritura de guiones, como Final Draft, Celtx o Fade In, estos suelen ser costosos, poco flexibles o no permiten la integración directa de fichas de personajes y locaciones en formatos abiertos como Fountain. Esta situación genera una brecha significativa entre la etapa de conceptualización y la redacción técnica del guion, limitando la eficiencia y la creatividad del proceso. Además, la falta de soluciones Open Source y adaptadas a distintas necesidades narrativas restringe el acceso de nuevos guionistas y equipos independientes a herramientas profesionales y actualizadas. \\

	El problema se agrava por la ausencia de aplicaciones web que permitan diseñar, estructurar y vincular fichas de personajes y locaciones de manera sencilla, personalizada y compatible con estándares de la industria. La carencia de estas funcionalidades obliga a los creadores a buscar soluciones alternativas o a realizar tareas repetitivas, lo que puede derivar en errores, pérdida de información y una menor calidad en los productos finales. Este vacío tecnológico afecta no solo la productividad, sino también la capacidad de innovación y experimentación en el ámbito audiovisual. \\

	Ante este panorama, surge la necesidad de desarrollar una herramienta accesible, flexible y de código abierto que permita a los guionistas crear, organizar y exportar fichas narrativas al formato Fountain de manera eficiente. Un sistema de estas características facilitaría la integración de las etapas creativas y técnicas del proceso de guion, democratizando el acceso a recursos profesionales y optimizando la producción de obras audiovisuales de calidad. \\

	 \newpage

\end{document}